
%{{{ wish-adapter.c

This new wish-adapter, implemented by Merijn de Jonge, seems good
enough to replace the old one. The old one was part of the ToolBus
distribution from the start, and its long evolution left its mark
on the clearness and conciseness of the code.

\nwfilename{wish-adapter.c.nw}\nwbegincode{1}\sublabel{NWwisH-wisF-1}\nwmargintag{{\nwtagstyle{}\subpageref{NWwisH-wisF-1}}}\moddef{wish-adapter.c*~{\nwtagstyle{}\subpageref{NWwisH-wisF-1}}}\endmoddef\let\nwnotused=\nwoutput{}
\LA{}notice~{\nwtagstyle{}\subpageref{NWwisH-not6-1}}\RA{}
\LA{}includes~{\nwtagstyle{}\subpageref{NWwisH-inc8-1}}\RA{}
\LA{}defines~{\nwtagstyle{}\subpageref{NWwisH-def7-1}}\RA{}
\LA{}global variables~{\nwtagstyle{}\subpageref{NWwisH-gloG-1}}\RA{}
\LA{}function declarations~{\nwtagstyle{}\subpageref{NWwisH-funL-1}}\RA{}

\LA{}error~{\nwtagstyle{}\subpageref{NWwisH-err5-1}}\RA{}
\LA{}wputc~{\nwtagstyle{}\subpageref{NWwisH-wpu5-1}}\RA{}
\LA{}wprintf~{\nwtagstyle{}\subpageref{NWwisH-wpr7-1}}\RA{}
\LA{}print_string~{\nwtagstyle{}\subpageref{NWwisH-priC-1}}\RA{}
\LA{}print_var~{\nwtagstyle{}\subpageref{NWwisH-pri9-1}}\RA{}
\LA{}print_env~{\nwtagstyle{}\subpageref{NWwisH-pri9.2-1}}\RA{}
\LA{}print_term~{\nwtagstyle{}\subpageref{NWwisH-priA-1}}\RA{}
\LA{}print_list~{\nwtagstyle{}\subpageref{NWwisH-priA.2-1}}\RA{}
\LA{}print_args~{\nwtagstyle{}\subpageref{NWwisH-priA.3-1}}\RA{}
\LA{}is_to_tool_comm~{\nwtagstyle{}\subpageref{NWwisH-is*F-1}}\RA{}
\LA{}is_from_tool_comm~{\nwtagstyle{}\subpageref{NWwisH-is*H-1}}\RA{}
\LA{}bytes_in_term~{\nwtagstyle{}\subpageref{NWwisH-bytD-1}}\RA{}
\LA{}rec_monitor~{\nwtagstyle{}\subpageref{NWwisH-recB-1}}\RA{}
\LA{}rec_do~{\nwtagstyle{}\subpageref{NWwisH-rec6-1}}\RA{}
\LA{}rec_eval~{\nwtagstyle{}\subpageref{NWwisH-rec8-1}}\RA{}
\LA{}rec_ack_event~{\nwtagstyle{}\subpageref{NWwisH-recD-1}}\RA{}
\LA{}rec_terminate~{\nwtagstyle{}\subpageref{NWwisH-recD.2-1}}\RA{}
\LA{}handle_input_from_toolbus~{\nwtagstyle{}\subpageref{NWwisH-hanP-1}}\RA{}
\LA{}handle_input_from_wish~{\nwtagstyle{}\subpageref{NWwisH-hanM-1}}\RA{}
\LA{}dummy_check_in_sign~{\nwtagstyle{}\subpageref{NWwisH-dumJ-1}}\RA{}
\LA{}require_fun~{\nwtagstyle{}\subpageref{NWwisH-reqB-1}}\RA{}
\LA{}check_in_sign~{\nwtagstyle{}\subpageref{NWwisH-cheD-1}}\RA{}
\LA{}wish_create~{\nwtagstyle{}\subpageref{NWwisH-wisB-1}}\RA{}
\LA{}wish_start~{\nwtagstyle{}\subpageref{NWwisH-wisA-1}}\RA{}
\LA{}signal_handler~{\nwtagstyle{}\subpageref{NWwisH-sigE-1}}\RA{}
\LA{}disconnect~{\nwtagstyle{}\subpageref{NWwisH-disA-1}}\RA{}
\LA{}shutdown~{\nwtagstyle{}\subpageref{NWwisH-shu8-1}}\RA{}
\LA{}signals_set~{\nwtagstyle{}\subpageref{NWwisH-sigB-1}}\RA{}
\LA{}help~{\nwtagstyle{}\subpageref{NWwisH-hel4-1}}\RA{}
\LA{}cmd_options~{\nwtagstyle{}\subpageref{NWwisH-cmdB-1}}\RA{}
\LA{}main~{\nwtagstyle{}\subpageref{NWwisH-mai4-1}}\RA{}

/*
 * EOF ./serie3/src/wa.c
 */
\nwnotused{wish-adapter.c*}\nwendcode{}\nwbegindocs{2}\nwdocspar

%}}}

%{{{ notice

This new wish-adapter, implemented by Merijn de Jonge, seems good
enough to replace the old one. The old one was part of the ToolBus
distribution from the start, and its long evolution left its mark
on the clearness and conciceness of the code.

\nwenddocs{}\nwbegincode{3}\sublabel{NWwisH-not6-1}\nwmargintag{{\nwtagstyle{}\subpageref{NWwisH-not6-1}}}\moddef{notice~{\nwtagstyle{}\subpageref{NWwisH-not6-1}}}\endmoddef\let\nwnotused=\nwoutput{}
/*
 * Praktikum: Programmeer Omgevingen II
 * File:      ./serie3/src/wa.c
 * Opgave:    Eindopdracht
 * Datum:     Thu Aug 22 1996
 * Auteur(s): Merijn de Jonge   mdejonge   9577483
*/

/*
 * Modified, nowebbed, and documented by Pieter Olivier.
 * Date:     Wed Aug 28 10:09:23 MET DST 1996
 * 
 */

/*
 * Synopsis: wish-adapter [options] -script Script [-script-args A1..An]
 * options:
 *      -help                 print help message
 *      -wish Wish            use Wish as wish executable
 *      -lazy-exec            postpone execution of wish until needed
 * 
 * where N is the toolname
 * and   S is a name of Tcl script to be executed
 * and   ... are arguments to be passed on to wish
 * 
 */

/*
 * notes:
 *    -not very well tested yet
 *
 */
\nwused{\\{NWwisH-wisF-1}}\nwendcode{}\nwbegindocs{4}\nwdocspar

%}}}
%{{{ includes

The wish-adapter is not a big program, and we only need a 
modest number of includes.

\nwenddocs{}\nwbegincode{5}\sublabel{NWwisH-inc8-1}\nwmargintag{{\nwtagstyle{}\subpageref{NWwisH-inc8-1}}}\moddef{includes~{\nwtagstyle{}\subpageref{NWwisH-inc8-1}}}\endmoddef\let\nwnotused=\nwoutput{}
#include <string.h>
#include <unistd.h>
#include <sys/wait.h>
#include <signal.h>
#include <errno.h>
#include <sys/types.h>
#include <TB.h>
\nwused{\\{NWwisH-wisF-1}}\nwendcode{}\nwbegindocs{6}\nwdocspar

%}}}
%{{{ defines

We define some constants, which make it easy to change some defaults.

\nwenddocs{}\nwbegincode{7}\sublabel{NWwisH-def7-1}\nwmargintag{{\nwtagstyle{}\subpageref{NWwisH-def7-1}}}\moddef{defines~{\nwtagstyle{}\subpageref{NWwisH-def7-1}}}\endmoddef\let\nwnotused=\nwoutput{}
#ifndef WISH
#define WISH    "wish"
#endif

#ifndef TBTCL
#define TBTCL   "/usr/local/lib/tb.tcl"
#endif

#define MAX_ARG 50
/*#define DEBUG_ON*/

#define USAGE_MSG \\
   "\\n" \\
   "Synopsis: wa [options]\\n" \\
   "\\n" \\
   "Options are:\\n" \\
   "-help                 print this message\\n" \\
   "-wish Wish            use Wish as wish executable\\n" \\
   "-lazy-exec            postpone execution of wish until needed\\n" \\
   "-script Name          use Name as Tcl script for wish\\n" \\
   "-script-args A1 ...   use A1 ... as arguments for the wish execution\\n"
\nwused{\\{NWwisH-wisF-1}}\nwendcode{}\nwbegindocs{8}\nwdocspar

%}}}
%{{{ global variables

Now we define some global variables, and declare two external variables.

\nwenddocs{}\nwbegincode{9}\sublabel{NWwisH-gloG-1}\nwmargintag{{\nwtagstyle{}\subpageref{NWwisH-gloG-1}}}\moddef{global variables~{\nwtagstyle{}\subpageref{NWwisH-gloG-1}}}\endmoddef
static char*  progname             = WISH;
static char*  script               = NULL;
static char*  script_args[MAX_ARG];
static int    script_nr_args       = 0;
static pid_t  pgid                 = -1;
static FILE*  to_wish              = NULL;
static FILE*  from_wish            = NULL;
static TBbool connected            = TBfalse;
static TBbool lazy_exec            = TBfalse;
static char*  name                 = NULL;
static char   backslashes[16]      = "\\\\\\\\\\\\\\\\\\\\\\\\\\\\\\\\\\\\\\\\\\\\\\\\\\\\\\\\\\\\\\\\";

extern term_list* tool_in_sign; /* for check_in_sign */
extern char*      tool_name;    /* idem */
\nwused{\\{NWwisH-wisF-1}}\nwendcode{}\nwbegindocs{10}\nwdocspar

%}}}
%{{{ function declarations

\nwenddocs{}\nwbegincode{11}\sublabel{NWwisH-funL-1}\nwmargintag{{\nwtagstyle{}\subpageref{NWwisH-funL-1}}}\moddef{function declarations~{\nwtagstyle{}\subpageref{NWwisH-funL-1}}}\endmoddef
static void error(const char* msg);
static void wputc(int c);
static void wprintf(char* fmt, ...);
static void print_string(char* s, int n);
static void print_var(term* e);
static void print_env(env* e);
static void print_term(term* t);
static void print_list(term_list* l, char* left, char* sep, char* right);
static TBbool is_to_tool_comm(const char* s );
static TBbool is_from_tool_comm(char* s );
static int bytes_in_term(term* t);
static TBbool rec_monitor(term* e, term** out);
static TBbool rec_do(term* e, term** out);
static TBbool rec_eval(term* e, term** out);
static TBbool rec_ack_event(term* e, term** out);
static TBbool rec_terminate(term* e, term** out);

static term*  handle_input_from_toolbus(term* e);
static term*  handle_input_from_wish(term* e);
static term*  dummy_check_in_sign(term* t);

static void require_fun(char* fname, term_list* fargs);
static void check_in_sign();
static void wish_create();
static void wish_start();
static void signal_handler( int sig );
static void disconnect();
static void shutdown( int ret );
static void signals_set();
static void help();
static void cmd_options( int argc, char* argv[] );
\nwused{\\{NWwisH-wisF-1}}\nwendcode{}\nwbegindocs{12}\nwdocspar

%}}}

%{{{ static void error(const char* msg)

The {\tt error} function prints errors to standard out.

\nwenddocs{}\nwbegincode{13}\sublabel{NWwisH-err5-1}\nwmargintag{{\nwtagstyle{}\subpageref{NWwisH-err5-1}}}\moddef{error~{\nwtagstyle{}\subpageref{NWwisH-err5-1}}}\endmoddef\let\nwnotused=\nwoutput{}
static void error( const char* msg )
\{
   disconnect();
   fprintf(stderr, "%s: %s : %s\\n", progname, msg, strerror(errno));
   shutdown(1);
\}
\nwused{\\{NWwisH-wisF-1}}\nwendcode{}\nwbegindocs{14}\nwdocspar

%}}}
%{{{ static void wputc(int c)

Print a character to wish. Duplicate it on
the stderr stream when {\tt DEBUG_ON} is defined.

\nwenddocs{}\nwbegincode{15}\sublabel{NWwisH-wpu5-1}\nwmargintag{{\nwtagstyle{}\subpageref{NWwisH-wpu5-1}}}\moddef{wputc~{\nwtagstyle{}\subpageref{NWwisH-wpu5-1}}}\endmoddef\let\nwnotused=\nwoutput{}
static void wputc( int c )
\{
   fputc( c, to_wish );
   #ifdef DEBUG_ON
      fputc( c, stderr );
   #endif
\}
\nwused{\\{NWwisH-wisF-1}}\nwendcode{}\nwbegindocs{16}\nwdocspar

%}}}
%{{{ static void wprintf(char *fmt, ...)

Print a formatted string to wish. Duplicate it to stderr
when {\tt DEBUG_ON} is defined.

\nwenddocs{}\nwbegincode{17}\sublabel{NWwisH-wpr7-1}\nwmargintag{{\nwtagstyle{}\subpageref{NWwisH-wpr7-1}}}\moddef{wprintf~{\nwtagstyle{}\subpageref{NWwisH-wpr7-1}}}\endmoddef\let\nwnotused=\nwoutput{}
static void wprintf(char* fmt, ...)
\{
  va_list ap;
  va_start(ap, fmt);
  vfprintf(to_wish, fmt, ap);
  #ifdef DEBUG_ON
    vfprintf(stderr, fmt, ap);
  #endif
  va_end(ap);
\}
\nwused{\\{NWwisH-wisF-1}}\nwendcode{}\nwbegindocs{18}\nwdocspar

%}}}
%{{{ static void print_string(char* s, int n)

{\tt print_string} writes a string to the {\tt to\_wish} channel. 
at most $n$ characters are written.
when non-printable characters are found in the string,
their octal values are written.

\nwenddocs{}\nwbegincode{19}\sublabel{NWwisH-priC-1}\nwmargintag{{\nwtagstyle{}\subpageref{NWwisH-priC-1}}}\moddef{print_string~{\nwtagstyle{}\subpageref{NWwisH-priC-1}}}\endmoddef\let\nwnotused=\nwoutput{}
static void print_string(char* s, int n)
\{
  wputc('"');
  while(n--)
  \{
    if(isprint(*s)) \{
      switch (*s) \{
        case '\\\\':
        case '"':
        case '\{':
        case '\}':
        case '[':
        case ']':
/*      case ' ':*/
        case ';':
        case '$':       wputc('\\\\');
        default:        wputc(*s);
      \}
    \} else \{
        wputc(*s);
/*      wprintf("\\\\%03o", (unsigned int)*s);*/
    \}
    s++;
  \}
  wputc('"');
\}
\nwused{\\{NWwisH-wisF-1}}\nwendcode{}\nwbegindocs{20}\nwdocspar

%}}}
%{{{ static void print_var(term* e)

{\tt print_var} prints variables to the {\tt to\_wish} channel.

\nwenddocs{}\nwbegincode{21}\sublabel{NWwisH-pri9-1}\nwmargintag{{\nwtagstyle{}\subpageref{NWwisH-pri9-1}}}\moddef{print_var~{\nwtagstyle{}\subpageref{NWwisH-pri9-1}}}\endmoddef\let\nwnotused=\nwoutput{}
static void print_var(term* e)
\{
   char* txt;
   txt = get_txt(var_sym(e));
   print_string(txt, strlen(txt));
   wputc(':');
   print_term(var_type(e));
   if(var_result(e))
     wputc('?');
\}
\nwused{\\{NWwisH-wisF-1}}\nwendcode{}\nwbegindocs{22}\nwdocspar

%}}}
%{{{ static void print_env(env* e)

{\tt print_env} prints environment terms to the {\tt to\_wish} channel.

\nwenddocs{}\nwbegincode{23}\sublabel{NWwisH-pri9.2-1}\nwmargintag{{\nwtagstyle{}\subpageref{NWwisH-pri9.2-1}}}\moddef{print_env~{\nwtagstyle{}\subpageref{NWwisH-pri9.2-1}}}\endmoddef\let\nwnotused=\nwoutput{}
static void print_env(env* e)
\{
   TBbool first = TBtrue;
   char* txt;
   
   wputc('"');
   while(e != NULL)
   \{
      assert(is_env(e));
      if(first)
        first = TBfalse;
      else
        wputc(' ');
      txt = get_txt(env_sym(e));
      print_string(txt, strlen(txt));
      wputc(' ');
      print_term(env_val(e));
      e = env_next(e);
   \}
   wputc('"');
\}
\nwused{\\{NWwisH-wisF-1}}\nwendcode{}\nwbegindocs{24}\nwdocspar

%}}}

%{{{ static void print_term(term* t)

Print a term to the {\tt to\_wish} channel.

\nwenddocs{}\nwbegincode{25}\sublabel{NWwisH-priA-1}\nwmargintag{{\nwtagstyle{}\subpageref{NWwisH-priA-1}}}\moddef{print_term~{\nwtagstyle{}\subpageref{NWwisH-priA-1}}}\endmoddef\let\nwnotused=\nwoutput{}
static void print_term(term* t)
\{
   switch(tkind(t))
   \{
      case t_str:
         print_string(str_val(t), strlen(str_val(t)));
         break;
      case t_bstr:
         print_string(bstr_val(t), bstr_len(t));
         break;
      case t_bool:
         if(bool_val(t) == TBtrue) \{
            wprintf("true");
         \} else \{
            wprintf("false");
         \}
         break;
      case t_int:
         wprintf("%d", int_val(t));
         break;
      case t_real:
         wprintf("%f", real_val(t));
         break;
      case t_var:
         print_var(t);
         break;
      case t_placeholder:
         wputc('<');
         print_term(placeholder_type(t));
         wputc('>');
         break;
      case t_appl:
         wprintf("%s", get_txt(fun_sym(t)));
         if(fun_args(t) != NULL)
            print_list(fun_args(t), "(", ",", ")");
         break;
      case t_list:
         print_list(t, "\{", " ", "\}");
         break;
      case t_env:
         print_env(t);
         break;
   \}
\}
\nwused{\\{NWwisH-wisF-1}}\nwendcode{}\nwbegindocs{26}\nwdocspar

%}}}
%{{{ static void print_list(term_list* l, char* left, char* sep,char* right,int depth)

Print a list to the {\tt to\_wish} channel. A custom separator is used.

\nwenddocs{}\nwbegincode{27}\sublabel{NWwisH-priA.2-1}\nwmargintag{{\nwtagstyle{}\subpageref{NWwisH-priA.2-1}}}\moddef{print_list~{\nwtagstyle{}\subpageref{NWwisH-priA.2-1}}}\endmoddef\let\nwnotused=\nwoutput{}
static void print_list(term_list* l, char* left, char* sep, char* right)
\{
   int i;
   wprintf(left);
   for(i=1; i<=list_length(l); i++) \{
      if(i>1) \{
         wprintf(sep);
      \}
      print_term(list_index(l, i));
   \}
   wprintf(right);
\}
\nwused{\\{NWwisH-wisF-1}}\nwendcode{}\nwbegindocs{28}\nwdocspar

%}}}
%{{{ static void print_args(term_list *args)

Print a list of arguments.

\nwenddocs{}\nwbegincode{29}\sublabel{NWwisH-priA.3-1}\nwmargintag{{\nwtagstyle{}\subpageref{NWwisH-priA.3-1}}}\moddef{print_args~{\nwtagstyle{}\subpageref{NWwisH-priA.3-1}}}\endmoddef\let\nwnotused=\nwoutput{}
static void print_args(term_list *args)
\{
  while(args) \{
    if(!is_list(list_first(args)))
      wprintf("\{");
    print_term(list_first(args));
    if(!is_list(list_first(args)))
      wprintf("\}");
    wputc(' ');
    args = list_next(args);
  \}
\}
\nwused{\\{NWwisH-wisF-1}}\nwendcode{}\nwbegindocs{30}\nwdocspar

%}}}
%{{{ static TBbool is_to_tool_comm(const char* s)

Check if a function name indicates toolbus-to-tool communication.

\nwenddocs{}\nwbegincode{31}\sublabel{NWwisH-is*F-1}\nwmargintag{{\nwtagstyle{}\subpageref{NWwisH-is*F-1}}}\moddef{is_to_tool_comm~{\nwtagstyle{}\subpageref{NWwisH-is*F-1}}}\endmoddef\let\nwnotused=\nwoutput{}
static TBbool is_to_tool_comm( const char* s )
\{
   return streq( s, "snd-eval" )      || 
          streq( s, "snd-cancel" )    ||
          streq( s, "snd-do" )        || 
          streq( s, "snd-ack-event" ) ||
          streq( s, "snd-terminate" );
\}
\nwused{\\{NWwisH-wisF-1}}\nwendcode{}\nwbegindocs{32}\nwdocspar

%}}}
%{{{ static TBbool is_from_tool_comm( char* s )

Check if a function name indicates tool-to-toolbus communication.

\nwenddocs{}\nwbegincode{33}\sublabel{NWwisH-is*H-1}\nwmargintag{{\nwtagstyle{}\subpageref{NWwisH-is*H-1}}}\moddef{is_from_tool_comm~{\nwtagstyle{}\subpageref{NWwisH-is*H-1}}}\endmoddef\let\nwnotused=\nwoutput{}
static TBbool is_from_tool_comm( char* s )
\{
   return streq( s, "rec-value" ) ||  
          streq( s, "rec-event" ) ||
          streq( s, "rec-disconnect" );
\}
\nwused{\\{NWwisH-wisF-1}}\nwendcode{}\nwbegindocs{34}\nwdocspar

%}}}
%{{{ static int bytes_in_term(term *t)

Calculates the number of bytes occumpied by a term when it is send
to the ToolBus.

\nwenddocs{}\nwbegincode{35}\sublabel{NWwisH-bytD-1}\nwmargintag{{\nwtagstyle{}\subpageref{NWwisH-bytD-1}}}\moddef{bytes_in_term~{\nwtagstyle{}\subpageref{NWwisH-bytD-1}}}\endmoddef\let\nwnotused=\nwoutput{}
static int bytes_in_term( term* t )
\{ 
   int length;
   
   length = 8; /* LENSPEC */
   length += strlen( TBsprintf( "%t", t ) );
   return length;
\}
\nwused{\\{NWwisH-wisF-1}}\nwendcode{}\nwbegindocs{36}\nwdocspar

%}}}
%{{{ static TBbool rec_monitor(term* e, term** out)

{\tt rec\_monitor} handles monitor events.

\nwenddocs{}\nwbegincode{37}\sublabel{NWwisH-recB-1}\nwmargintag{{\nwtagstyle{}\subpageref{NWwisH-recB-1}}}\moddef{rec_monitor~{\nwtagstyle{}\subpageref{NWwisH-recB-1}}}\endmoddef\let\nwnotused=\nwoutput{}
static TBbool rec_monitor( term* e, term** out )
\{
  \LA{}rec_monitor variables~{\nwtagstyle{}\subpageref{NWwisH-recL-1}}\RA{}

  *out = NULL;

  \LA{}dissect rec-monitor term~{\nwtagstyle{}\subpageref{NWwisH-disO-1}}\RA{}

  wprintf("if [catch \{ ");

  \LA{}dissect coordinates~{\nwtagstyle{}\subpageref{NWwisH-disJ-1}}\RA{}
   
   if(streq(AtFun, "create")) \{
     \LA{}handle create~{\nwtagstyle{}\subpageref{NWwisH-hanD-1}}\RA{}
   \} else  if(streq(AtFun, "rec-connect")) \{
     \LA{}handle rec-connect~{\nwtagstyle{}\subpageref{NWwisH-hanI-1}}\RA{}
   \}
   else if(is_to_tool_comm(AtFun)) \{
     \LA{}handle toolbus-to-tool communication~{\nwtagstyle{}\subpageref{NWwisH-hana-1}}\RA{}
   \}
   else if(is_from_tool_comm(AtFun)) \{
     \LA{}handle tool-to-toolbus communication~{\nwtagstyle{}\subpageref{NWwisH-hana.2-1}}\RA{}
   \}

   if(pid1 > 0) \{
     \LA{}update variables~{\nwtagstyle{}\subpageref{NWwisH-updG-1}}\RA{}
     \LA{}update subscriptions~{\nwtagstyle{}\subpageref{NWwisH-updK-1}}\RA{}
     \LA{}update notes~{\nwtagstyle{}\subpageref{NWwisH-updC-1}}\RA{}
   \}
   
   if( pid2 > 0 ) \{
     \LA{}handle proc-to-proc communication~{\nwtagstyle{}\subpageref{NWwisH-hanX-1}}\RA{}
   \}
   
   if(streq(AtFun, "endlet")) \{
     \LA{}handle endlet~{\nwtagstyle{}\subpageref{NWwisH-hanD.2-1}}\RA{}
   \} else \{
     wprintf("monitor_atom %d %s %s %d %d %d %d\\n",
                        pid1, AtFun, filename, blino, bpos, elino, epos);
   \}

   wprintf("\} msg] \{ TBerror $msg \}\\n");
   fflush(to_wish);

   return TBtrue;
\}
\nwused{\\{NWwisH-wisF-1}}\nwendcode{}\nwbegindocs{38}\nwdocspar

%{{{ rec_monitor variables

The {\tt rec\_monitor} function needs an awfull lot of variables.
We might want to split up this function in the future.

\nwenddocs{}\nwbegincode{39}\sublabel{NWwisH-recL-1}\nwmargintag{{\nwtagstyle{}\subpageref{NWwisH-recL-1}}}\moddef{rec_monitor variables~{\nwtagstyle{}\subpageref{NWwisH-recL-1}}}\endmoddef
  int        pid1;
  int        pid2;
  int        blino;
  int        bpos;
  int        elino;
  int        epos;
  int        tid;
  int        dir;
  term*      Env;
  term*      Subs;
  term*      Notes;
  term*      AtArgs;
  term*      Coords;
  char*      AtFun;
  term_list* ts;
  term_list* pair;
  term_list* args;
  term*      var;
  term*      pe;
  char*      mon_point;
  char*      name;
  char*      filename;
\nwused{\\{NWwisH-recB-1}}\nwendcode{}\nwbegindocs{40}\nwdocspar

%}}}
%{{{ dissect rec-monitor term

Now we analyze the rec-monitor term, and check if it has the right format.

\nwenddocs{}\nwbegincode{41}\sublabel{NWwisH-disO-1}\nwmargintag{{\nwtagstyle{}\subpageref{NWwisH-disO-1}}}\moddef{dissect rec-monitor term~{\nwtagstyle{}\subpageref{NWwisH-disO-1}}}\endmoddef
  if( TBmatch( e, "rec-monitor(%f(%d,%f,%t,%t,%t,%t,%t,%d,%t))",
        &mon_point,
        &pid1,
        &AtFun,
        &AtArgs,
        &Coords,
        &Env,
        &Subs,
        &Notes,
        &pid2,
        &pe ) == TBfalse )
    return TBfalse;
\nwused{\\{NWwisH-recB-1}}\nwendcode{}\nwbegindocs{42}\nwdocspar

%}}}
%{{{ dissect coordinates

Analyze the coordinats in {\tt Coords}. If {\tt Coords} is {\tt NULL},
then we are in the {\tt INIT} process, of which we do not have the
source code.

\nwenddocs{}\nwbegincode{43}\sublabel{NWwisH-disJ-1}\nwmargintag{{\nwtagstyle{}\subpageref{NWwisH-disJ-1}}}\moddef{dissect coordinates~{\nwtagstyle{}\subpageref{NWwisH-disJ-1}}}\endmoddef
  if(Coords == NULL) \{
      filename = "INIT";
      blino = elino = bpos = epos = 0;
  \} else \{
    if(TBmatch(Coords, "[%s,%d,%d,%d,%d]", 
                &filename, &blino, &bpos, &elino, &epos) == TBfalse )
         TBmsg( "**** coords do not match\\n" );
  \}
\nwused{\\{NWwisH-recB-1}}\nwendcode{}\nwbegindocs{44}\nwdocspar

%}}}
%{{{ handle create

Handle the {\tt create} primitive. This amounts to calling {\tt create\_proc}
with the right arguments.

\nwenddocs{}\nwbegincode{45}\sublabel{NWwisH-hanD-1}\nwmargintag{{\nwtagstyle{}\subpageref{NWwisH-hanD-1}}}\moddef{handle create~{\nwtagstyle{}\subpageref{NWwisH-hanD-1}}}\endmoddef
  if(TBmatch(AtArgs, "[%f(%l), %t]", &name, &args, &var) == TBfalse)
    TBmsg( "**** args of create do not match\\n" );
  TBprintf(to_wish, "create_proc %t %s\\n", list_get(Env, var), name);
\nwused{\\{NWwisH-recB-1}}\nwendcode{}\nwbegindocs{46}\nwdocspar

%}}}
%{{{ handle rec-connect

Handle the {\tt rec-connect} primitive. Just like the {\tt create} primitive,
we call {\tt create\_tool}.

\nwenddocs{}\nwbegincode{47}\sublabel{NWwisH-hanI-1}\nwmargintag{{\nwtagstyle{}\subpageref{NWwisH-hanI-1}}}\moddef{handle rec-connect~{\nwtagstyle{}\subpageref{NWwisH-hanI-1}}}\endmoddef
  if(TBmatch(AtArgs, "[%t]", &var ) == TBfalse)
    TBmsg( "**** args of rec-connect do not match: %t\\n", AtArgs );
  if(TBmatch(get_list_as_env(var, Env), "%f(%d)", &name,&tid) == TBfalse) \{
    TBmsg( "**** rec-connect: value of var does not match: var=%t, env=%t", 
                                                                    var, Env);
    TBmsg( "**** value_list(var, Env) = %t\\n", get_list_as_env(var, Env));
  \}
  TBprintf( to_wish, "create_tool %d %s\\n", tid, name );
\nwused{\\{NWwisH-recB-1}}\nwendcode{}\nwbegindocs{48}\nwdocspar

%}}}
%{{{ handle toolbus-to-tool communication

Handle one of the toolbus-to-tool communication primitives
by calling {\tt proc\_tool\_comm} with the right arguments.

\nwenddocs{}\nwbegincode{49}\sublabel{NWwisH-hana-1}\nwmargintag{{\nwtagstyle{}\subpageref{NWwisH-hana-1}}}\moddef{handle toolbus-to-tool communication~{\nwtagstyle{}\subpageref{NWwisH-hana-1}}}\endmoddef
  if(TBmatch(AtArgs, "[%t, %l]", &var, &args) == TBfalse)
    TBmsg( "**** args of tool comm do not match:%t\\n", AtArgs );
  if(TBmatch(get_list_as_env(var, Env ), "%f(%d)", &name, &tid) == TBfalse)
    TBmsg("**** to_tool: value of var does not match: var=%t, env=%t",var,Env);
  TBprintf(to_wish, "proc_tool_comm %d %d %d\\n", pid1,tid,bytes_in_term(args));
\nwused{\\{NWwisH-recB-1}}\nwendcode{}\nwbegindocs{50}\nwdocspar

%}}}
%{{{ handle tool-to-toolbus communication

Handle one of the tool-to-toolbus communication primitives
by calling {\tt tool\_proc\_comm} with the right arguments.

\nwenddocs{}\nwbegincode{51}\sublabel{NWwisH-hana.2-1}\nwmargintag{{\nwtagstyle{}\subpageref{NWwisH-hana.2-1}}}\moddef{handle tool-to-toolbus communication~{\nwtagstyle{}\subpageref{NWwisH-hana.2-1}}}\endmoddef
  if(TBmatch( AtArgs, "[%t, %l]", &var, &args) == TBfalse)
    TBmsg("**** args of tool comm do not match:%t\\n", AtArgs);
  if(TBmatch(get_list_as_env(var, Env), "%f(%d)", &name, &tid) == TBfalse)
    TBmsg("**** from_tool: val of var doesn't match: var=%t, env=%t", var,Env);
  TBprintf(to_wish, "tool_proc_comm %d %d %d\\n", pid1,tid,bytes_in_term(args));
\nwused{\\{NWwisH-recB-1}}\nwendcode{}\nwbegindocs{52}\nwdocspar

%}}}
%{{{ update variables

Give the viewer a chance to update its shadow variables.

\nwenddocs{}\nwbegincode{53}\sublabel{NWwisH-updG-1}\nwmargintag{{\nwtagstyle{}\subpageref{NWwisH-updG-1}}}\moddef{update variables~{\nwtagstyle{}\subpageref{NWwisH-updG-1}}}\endmoddef
  for(ts = Env; ts != NULL ; ts = next(ts)) \{
    pair = first(ts);
    if(get_txt(var_sym(first(pair)))[0] != '_')
    TBprintf(to_wish, "update_var %d \{%t\} \}}
%{{{ update subscriptions/notes

Let the viewer update its subscriptions/notes.

\nwenddocs{}\nwbegincode{55}\sublabel{NWwisH-updK-1}\nwmargintag{{\nwtagstyle{}\subpageref{NWwisH-updK-1}}}\moddef{update subscriptions~{\nwtagstyle{}\subpageref{NWwisH-updK-1}}}\endmoddef
  for(ts = Subs; ts != NULL; ts = next(ts))
    TBprintf(to_wish, "update_subs %d \{%t\}\\n", pid1, first(ts));
\nwused{\\{NWwisH-recB-1}}\nwendcode{}\nwbegindocs{56}\nwdocspar

\nwenddocs{}\nwbegincode{57}\sublabel{NWwisH-updC-1}\nwmargintag{{\nwtagstyle{}\subpageref{NWwisH-updC-1}}}\moddef{update notes~{\nwtagstyle{}\subpageref{NWwisH-updC-1}}}\endmoddef
  for(ts = Notes; ts != NULL; ts = next(ts))
    TBprintf(to_wish, "update_notes %d \}}
%{{{ handle proc-to-proc communication

Two processes are communicating, let the viewer know by calling
the procedure {\tt proc\_proc\_comm}.

\nwenddocs{}\nwbegincode{59}\sublabel{NWwisH-hanX-1}\nwmargintag{{\nwtagstyle{}\subpageref{NWwisH-hanX-1}}}\moddef{handle proc-to-proc communication~{\nwtagstyle{}\subpageref{NWwisH-hanX-1}}}\endmoddef
  dir = streq(AtFun, "snd-msg") ? 1 : -1;
  TBprintf(to_wish, "proc_proc_comm %d %d %d\\n", pid1, pid2, dir);
\nwused{\\{NWwisH-recB-1}}\nwendcode{}\nwbegindocs{60}\nwdocspar

%}}}
%{{{ handle endlet

On encountering an {\tt endlet}, we need to remove a number of
variables that are no longer valid.

\nwenddocs{}\nwbegincode{61}\sublabel{NWwisH-hanD.2-1}\nwmargintag{{\nwtagstyle{}\subpageref{NWwisH-hanD.2-1}}}\moddef{handle endlet~{\nwtagstyle{}\subpageref{NWwisH-hanD.2-1}}}\endmoddef
  if(TBmatch(AtArgs, "[%l]", &args) == TBfalse)
    TBmsg("**** args of end_let do not match\\n");
  for( ; args != NULL; args = next(args))
    TBprintf(to_wish, "delete_var %d \{%t\}\\n", pid1, first(args));
  TBprintf(to_wish, "\} msg] \{ TBerror $msg \}\\n");
  *out = TBmake( "snd-continue(%d)", pid1);
  return TBtrue;
\nwused{\\{NWwisH-recB-1}}\nwendcode{}\nwbegindocs{62}\nwdocspar

%}}}

%}}}
%{{{ static TBbool rec_do( term* e, term** out )

Build the arguments and call the Tcl procedure.

\nwenddocs{}\nwbegincode{63}\sublabel{NWwisH-rec6-1}\nwmargintag{{\nwtagstyle{}\subpageref{NWwisH-rec6-1}}}\moddef{rec_do~{\nwtagstyle{}\subpageref{NWwisH-rec6-1}}}\endmoddef\let\nwnotused=\nwoutput{}
static TBbool rec_do(term* e, term** out)
\{
   char* fname;
   term_list* fargs;
   
   *out = NULL;
   
   /* rec-do */
   if(TBmatch( e, "rec-do(%f(%l))", &fname, &fargs) == TBfalse)
      return TBfalse;
   wprintf("if [catch \}}
%{{{ static TBbool rec_eval( term* e, term** out )

Build the arguments and call the Tcl procedure.

\nwenddocs{}\nwbegincode{65}\sublabel{NWwisH-rec8-1}\nwmargintag{{\nwtagstyle{}\subpageref{NWwisH-rec8-1}}}\moddef{rec_eval~{\nwtagstyle{}\subpageref{NWwisH-rec8-1}}}\endmoddef\let\nwnotused=\nwoutput{}
static TBbool rec_eval( term* e, term** out )
\{
   char* fname;
   term_list* fargs;
   
   *out = NULL;
   
   /* rec-eval */
   if(TBmatch(e, "rec-eval(%f(%l))", &fname, &fargs) == TBfalse)
     return TBfalse;
   wprintf("if [catch \}}
%{{{ static TBbool rec_ack_event( term* e, term** out )

Build the arguments and call the Tcl procedure {\tt rec-ack-event}.

\nwenddocs{}\nwbegincode{67}\sublabel{NWwisH-recD-1}\nwmargintag{{\nwtagstyle{}\subpageref{NWwisH-recD-1}}}\moddef{rec_ack_event~{\nwtagstyle{}\subpageref{NWwisH-recD-1}}}\endmoddef\let\nwnotused=\nwoutput{}
static TBbool rec_ack_event(term* e, term** out)
\{
   term* farg;
   
   *out = NULL;
   
   /* rec-ack-event */
   if(TBmatch(e, "rec-ack-event(%t)", &farg) == TBfalse)
      return TBfalse;
   wprintf("if [catch \{rec-ack-event \{");
   print_term(farg);
   wprintf("\}\} msg] \{TBerror $msg\}\\n" );
   fflush(to_wish);
   return TBtrue;
\}
\nwused{\\{NWwisH-wisF-1}}\nwendcode{}\nwbegindocs{68}\nwdocspar

%}}}
%{{{ static TBbool rec_terminate( term* e, term** out )

On receiving a {\tt rec-terminate} request, we first
call the Tcl procedure {\tt rec-terminate}, and then
quit.

\nwenddocs{}\nwbegincode{69}\sublabel{NWwisH-recD.2-1}\nwmargintag{{\nwtagstyle{}\subpageref{NWwisH-recD.2-1}}}\moddef{rec_terminate~{\nwtagstyle{}\subpageref{NWwisH-recD.2-1}}}\endmoddef\let\nwnotused=\nwoutput{}
static TBbool rec_terminate( term* e, term** out )
\{
   term* farg;
   
   *out = NULL;
   
   /* rec-terminate */
   if(TBmatch(e, "rec-terminate(%t)", &farg) == TBfalse)
      return TBfalse;
   wprintf("if [catch \{rec-terminate \{");
   print_term(farg);
   wprintf("\}\} msg] \{TBerror $msg\}; exit\\n" );
   fflush(to_wish);
   connected = TBfalse;
   shutdown(0);
   
   /* not reached */
   return TBtrue;
\}
\nwused{\\{NWwisH-wisF-1}}\nwendcode{}\nwbegindocs{70}\nwdocspar

%}}}

%{{{ term *handle_input_from_toolbus(term *e)

Handle input from the ToolBus.

\nwenddocs{}\nwbegincode{71}\sublabel{NWwisH-hanP-1}\nwmargintag{{\nwtagstyle{}\subpageref{NWwisH-hanP-1}}}\moddef{handle_input_from_toolbus~{\nwtagstyle{}\subpageref{NWwisH-hanP-1}}}\endmoddef\let\nwnotused=\nwoutput{}
term* handle_input_from_toolbus( term* e )
\{
   term* result;
   
   if( rec_ack_event( e, &result ) ||
       rec_terminate( e, &result ) ||
         rec_monitor( e, &result ) ||
            rec_eval( e, &result ) ||
              rec_do( e, &result ) )
      return result;

   /* error ! */
   TBmsg("Ignored: %t\\n", e);
   return NULL;
\}
\nwused{\\{NWwisH-wisF-1}}\nwendcode{}\nwbegindocs{72}\nwdocspar

%}}}
%{{{ term* handle_input_from_wish(term *e) 

Handle input from wish.

\nwenddocs{}\nwbegincode{73}\sublabel{NWwisH-hanM-1}\nwmargintag{{\nwtagstyle{}\subpageref{NWwisH-hanM-1}}}\moddef{handle_input_from_wish~{\nwtagstyle{}\subpageref{NWwisH-hanM-1}}}\endmoddef\let\nwnotused=\nwoutput{}
term* handle_input_from_wish( term* e ) 
\{
   char* msg;

   /* wish-error */
   if(TBmatch(e, "wish-error( %s )", &msg)) \{
      TBmsg("wish-error: %s\\n", msg);
      return NULL;
   \}

   /* snd-disconnect */
   if(TBmatch(e, "snd-disconnect")) \{
      disconnect();
      shutdown( 0 );
   \}
      
   return e;
\}
\nwused{\\{NWwisH-wisF-1}}\nwendcode{}\nwbegindocs{74}\nwdocspar

%}}}

%{{{ term *dummy_check_in_sign(term *t)

\nwenddocs{}\nwbegincode{75}\sublabel{NWwisH-dumJ-1}\nwmargintag{{\nwtagstyle{}\subpageref{NWwisH-dumJ-1}}}\moddef{dummy_check_in_sign~{\nwtagstyle{}\subpageref{NWwisH-dumJ-1}}}\endmoddef\let\nwnotused=\nwoutput{}
term *dummy_check_in_sign( term* t )
\{
   return NULL;
\}
\nwused{\\{NWwisH-wisF-1}}\nwendcode{}\nwbegindocs{76}\nwdocspar

%}}}
%{{{ static void require_fun( char* fname, term_list* fargs )

Call the Tcl procedure {\tt TBrequire}, to check if a certain
function is present.

\nwenddocs{}\nwbegincode{77}\sublabel{NWwisH-reqB-1}\nwmargintag{{\nwtagstyle{}\subpageref{NWwisH-reqB-1}}}\moddef{require_fun~{\nwtagstyle{}\subpageref{NWwisH-reqB-1}}}\endmoddef\let\nwnotused=\nwoutput{}
static void require_fun( char* fname, term_list* fargs )
\{
   wprintf("TBrequire %s %s %d\\n", tool_name, fname, list_length(fargs));
   fflush(to_wish);
\}
\nwused{\\{NWwisH-wisF-1}}\nwendcode{}\nwbegindocs{78}\nwdocspar

%}}}
%{{{ static void check_in_sign()

Check the input signature of a tool.

\nwenddocs{}\nwbegincode{79}\sublabel{NWwisH-cheD-1}\nwmargintag{{\nwtagstyle{}\subpageref{NWwisH-cheD-1}}}\moddef{check_in_sign~{\nwtagstyle{}\subpageref{NWwisH-cheD-1}}}\endmoddef\let\nwnotused=\nwoutput{}
static void check_in_sign()
\{
   char*      atf;
   char       pat[128];
   term_list* arg;
   term_list* reqs;

   reqs = tool_in_sign;
   
   /* construct match pattern, e.g. ``%f( <calc>,%l )'' */
   sprintf( pat, "%%f( <%s>, %%l )", tool_name );

   for( ; reqs != NULL; reqs = next( reqs ) )
   \{
      if( TBmatch( first( reqs ), pat, &atf, &arg ) )
      \{
         if( streq( atf, "rec-do") || streq( atf, "rec-eval" ) )
            require_fun( get_txt( fun_sym( first( arg ) ) ), 
                         fun_args( first( arg ) ) );
         else
         if( streq( atf, "rec-ack-event" ) )
            require_fun( "rec-ack-event", arg );
         else
         if( streq( atf, "rec-terminate" ) )
            require_fun( "rec-terminate", arg );
         else
            TBmsg( "check_in_sign: skipped %t\\n", first( reqs ) );
      \}
      else
         TBmsg( "check_in_sign: skipped %t\\n", first( reqs ) );
   \}
\}
\nwused{\\{NWwisH-wisF-1}}\nwendcode{}\nwbegindocs{80}\nwdocspar

%}}}

%{{{ static void wish_create()

Create the wish child.

\nwenddocs{}\nwbegincode{81}\sublabel{NWwisH-wisB-1}\nwmargintag{{\nwtagstyle{}\subpageref{NWwisH-wisB-1}}}\moddef{wish_create~{\nwtagstyle{}\subpageref{NWwisH-wisB-1}}}\endmoddef\let\nwnotused=\nwoutput{}
static void wish_create()
\{
   pid_t pid;
   int   pipe_tw[2]; /* toolbus 2 wish */
   int   pipe_wt[2]; /* wish 2 toolbus */
   
   if( pipe( pipe_tw ) < 0 )
      error( "pipe" );
   if( pipe( pipe_wt ) < 0 )
      error( "pipe" );

   pid = fork();
   switch( pid )
   \{
      case 0:
         if(setpgid(0, 0) == -1)
            error( "sepgid");
         dup2(pipe_tw[0], 0);
         dup2(pipe_wt[1], 1);
         
         close(pipe_tw[1]);
         close(pipe_wt[0]);
         
         execlp(progname, progname, "-name", name, NULL);
         fprintf(stderr, "%s: execlp: %s\\n", progname, strerror(errno));
         exit(1);
      case -1:
         error("fork");
      default:
         if(pgid == -1)
            pgid = pid;
         to_wish = fdopen(pipe_tw[1], "w");
         from_wish = fdopen(pipe_wt[0], "r");
         close(pipe_tw[0]);
         close(pipe_wt[1]);
         TBaddTermPort( pipe_wt[0], handle_input_from_wish );
         return;
   \}
\}
\nwused{\\{NWwisH-wisF-1}}\nwendcode{}\nwbegindocs{82}\nwdocspar

%}}}
%{{{ static void wish_start()

Get the tool going by sourcing the library file and the tool script.

\nwenddocs{}\nwbegincode{83}\sublabel{NWwisH-wisA-1}\nwmargintag{{\nwtagstyle{}\subpageref{NWwisH-wisA-1}}}\moddef{wish_start~{\nwtagstyle{}\subpageref{NWwisH-wisA-1}}}\endmoddef\let\nwnotused=\nwoutput{}
static void wish_start()
\{
   int i;

   wprintf("if ![file exists %s] "
                      "\{puts stderr \\"%s not found\\"; exit 1\}\\n", TBTCL, TBTCL );
   wprintf("if ![file exists %s] \{puts stderr \\"%s not found\\"; exit 1\}\\n",
                                                        script, script);
   
   wprintf("source %s\\n", TBTCL);
   wprintf("set argv \{") ;
   for(i=0; i<script_nr_args; i++) \{
     wprintf("%s ", script_args[i]);
   \}
   wprintf("\}\\n");
   wprintf("set argc %d\\n", script_nr_args);
   wprintf("set TB_LIBDIR %s\\n", LIBDIR);
   wprintf("source %s\\n", script);
   check_in_sign();
\}
\nwused{\\{NWwisH-wisF-1}}\nwendcode{}\nwbegindocs{84}\nwdocspar

%}}}
%{{{ static void signal_handler( int sig )

The function {\tt signal\_handler} is called when a signal is
received.

\nwenddocs{}\nwbegincode{85}\sublabel{NWwisH-sigE-1}\nwmargintag{{\nwtagstyle{}\subpageref{NWwisH-sigE-1}}}\moddef{signal_handler~{\nwtagstyle{}\subpageref{NWwisH-sigE-1}}}\endmoddef\let\nwnotused=\nwoutput{}
static void signal_handler( int sig )
\{
   int status;
   int child_exit = -1;

   switch( sig )
   \{
      case SIGCHLD:
         if( waitpid( -pgid, &status, WNOHANG ) == -1 )
            error( "waitpid" );
         /* only save child's exit when it was not killed */
         if( WIFEXITED(status) )
            child_exit = WEXITSTATUS( status );
         disconnect();
         exit( child_exit );
         break;
      default:
         disconnect();
         shutdown( 1 );
   \}
\}
\nwused{\\{NWwisH-wisF-1}}\nwendcode{}\nwbegindocs{86}\nwdocspar

%}}}
%{{{ static void disconnect()

Disconnect from the ToolBus, by sending a {\tt snd-disconnect}.

\nwenddocs{}\nwbegincode{87}\sublabel{NWwisH-disA-1}\nwmargintag{{\nwtagstyle{}\subpageref{NWwisH-disA-1}}}\moddef{disconnect~{\nwtagstyle{}\subpageref{NWwisH-disA-1}}}\endmoddef\let\nwnotused=\nwoutput{}
static void disconnect()
\{
   if(connected)
      TBsend(TBmake("snd-disconnect"));
   connected = TBfalse;
\}
\nwused{\\{NWwisH-wisF-1}}\nwendcode{}\nwbegindocs{88}\nwdocspar

%}}}
%{{{ static void shutdown(int ret)

Kill the wish child and exit.

\nwenddocs{}\nwbegincode{89}\sublabel{NWwisH-shu8-1}\nwmargintag{{\nwtagstyle{}\subpageref{NWwisH-shu8-1}}}\moddef{shutdown~{\nwtagstyle{}\subpageref{NWwisH-shu8-1}}}\endmoddef\let\nwnotused=\nwoutput{}
static void shutdown( int ret )
\{
   sleep(1);
   
   if(pgid != -1)
      kill(-pgid, SIGKILL);
   
   exit(ret);
\}
\nwused{\\{NWwisH-wisF-1}}\nwendcode{}\nwbegindocs{90}\nwdocspar

%}}}
%{{{ static void signals_set()

Catch the most important signals, and redirect them to {\tt signal\_handler}.

\nwenddocs{}\nwbegincode{91}\sublabel{NWwisH-sigB-1}\nwmargintag{{\nwtagstyle{}\subpageref{NWwisH-sigB-1}}}\moddef{signals_set~{\nwtagstyle{}\subpageref{NWwisH-sigB-1}}}\endmoddef\let\nwnotused=\nwoutput{}
static void signals_set()
\{
   struct sigaction act;
   
   act.sa_handler = signal_handler;
   act.sa_flags   = SA_RESTART;

   sigemptyset( &act.sa_mask );
   
   sigaction( SIGINT,  &act, NULL );
   sigaction( SIGTERM, &act, NULL );
   sigaction( SIGHUP,  &act, NULL );
   sigaction( SIGQUIT, &act, NULL );
   sigaction( SIGCHLD, &act, NULL );  
\}
\nwused{\\{NWwisH-wisF-1}}\nwendcode{}\nwbegindocs{92}\nwdocspar

%}}}
%{{{ static void help()

Print a help (usage) message.

\nwenddocs{}\nwbegincode{93}\sublabel{NWwisH-hel4-1}\nwmargintag{{\nwtagstyle{}\subpageref{NWwisH-hel4-1}}}\moddef{help~{\nwtagstyle{}\subpageref{NWwisH-hel4-1}}}\endmoddef\let\nwnotused=\nwoutput{}
static void help()
\{
   fprintf( stderr, USAGE_MSG );
\}
\nwused{\\{NWwisH-wisF-1}}\nwendcode{}\nwbegindocs{94}\nwdocspar

%}}}
%{{{ static void cmd_options( int argc, char* argv[] )

Analyze the command line arguments.

\nwenddocs{}\nwbegincode{95}\sublabel{NWwisH-cmdB-1}\nwmargintag{{\nwtagstyle{}\subpageref{NWwisH-cmdB-1}}}\moddef{cmd_options~{\nwtagstyle{}\subpageref{NWwisH-cmdB-1}}}\endmoddef\let\nwnotused=\nwoutput{}
static void cmd_options( int argc, char* argv[] )
\{
   int i;

   i = 0;

   while( i < argc )
   \{
      if( streq( argv[i], "-help" ) )
         help();
      if( streq( argv[i], "-TB_TOOL_NAME" ) )
         name = argv[++i];
      if( streq( argv[i], "-script" ) )
         script = argv[++i];
      if( streq( argv[i], "-lazy-exec" ) )
         lazy_exec = TBtrue;
      if( streq( argv[i], "-wish" ) )
         progname = argv[++i];
      if( streq( argv[i], "-script-args" ) )
      \{
         i++;
         for( ; i < argc; i++ )
            script_args[script_nr_args++] = argv[i];
      \}
      i++;
   \}
\}
\nwused{\\{NWwisH-wisF-1}}\nwendcode{}\nwbegindocs{96}\nwdocspar

%}}}
%{{{ void main(int argc, char* argv[])

Get things going. After analyzing the arguments and setting the signals,
we make contact with the ToolBus and start the wish child.
After this, the ToolBus eventloop takes care of the rest.

\nwenddocs{}\nwbegincode{97}\sublabel{NWwisH-mai4-1}\nwmargintag{{\nwtagstyle{}\subpageref{NWwisH-mai4-1}}}\moddef{main~{\nwtagstyle{}\subpageref{NWwisH-mai4-1}}}\endmoddef\let\nwnotused=\nwoutput{}
void main( int argc, char* argv[] )
\{
   cmd_options(argc, argv);
   signals_set();
   
   TBinit(name, argc, argv, handle_input_from_toolbus, dummy_check_in_sign);
   connected = TBtrue;

   if(script == NULL) \{
      error("Missing -script argument\\n");
      exit(1);
   \}

   while(lazy_exec && !TBpeek())
      sleep(1);

   wish_create();
   wish_start();
   signals_set();
   
   TBeventloop();
\}
\nwused{\\{NWwisH-wisF-1}}\nwendcode{}

\nwixlogsorted{c}{{bytes_in_term}{NWwisH-bytD-1}{\nwixu{NWwisH-wisF-1}\nwixd{NWwisH-bytD-1}}}%
\nwixlogsorted{c}{{check_in_sign}{NWwisH-cheD-1}{\nwixu{NWwisH-wisF-1}\nwixd{NWwisH-cheD-1}}}%
\nwixlogsorted{c}{{cmd_options}{NWwisH-cmdB-1}{\nwixu{NWwisH-wisF-1}\nwixd{NWwisH-cmdB-1}}}%
\nwixlogsorted{c}{{defines}{NWwisH-def7-1}{\nwixu{NWwisH-wisF-1}\nwixd{NWwisH-def7-1}}}%
\nwixlogsorted{c}{{disconnect}{NWwisH-disA-1}{\nwixu{NWwisH-wisF-1}\nwixd{NWwisH-disA-1}}}%
\nwixlogsorted{c}{{dissect coordinates}{NWwisH-disJ-1}{\nwixu{NWwisH-recB-1}\nwixd{NWwisH-disJ-1}}}%
\nwixlogsorted{c}{{dissect rec-monitor term}{NWwisH-disO-1}{\nwixu{NWwisH-recB-1}\nwixd{NWwisH-disO-1}}}%
\nwixlogsorted{c}{{dummy_check_in_sign}{NWwisH-dumJ-1}{\nwixu{NWwisH-wisF-1}\nwixd{NWwisH-dumJ-1}}}%
\nwixlogsorted{c}{{error}{NWwisH-err5-1}{\nwixu{NWwisH-wisF-1}\nwixd{NWwisH-err5-1}}}%
\nwixlogsorted{c}{{function declarations}{NWwisH-funL-1}{\nwixu{NWwisH-wisF-1}\nwixd{NWwisH-funL-1}}}%
\nwixlogsorted{c}{{global variables}{NWwisH-gloG-1}{\nwixu{NWwisH-wisF-1}\nwixd{NWwisH-gloG-1}}}%
\nwixlogsorted{c}{{handle create}{NWwisH-hanD-1}{\nwixu{NWwisH-recB-1}\nwixd{NWwisH-hanD-1}}}%
\nwixlogsorted{c}{{handle endlet}{NWwisH-hanD.2-1}{\nwixu{NWwisH-recB-1}\nwixd{NWwisH-hanD.2-1}}}%
\nwixlogsorted{c}{{handle proc-to-proc communication}{NWwisH-hanX-1}{\nwixu{NWwisH-recB-1}\nwixd{NWwisH-hanX-1}}}%
\nwixlogsorted{c}{{handle rec-connect}{NWwisH-hanI-1}{\nwixu{NWwisH-recB-1}\nwixd{NWwisH-hanI-1}}}%
\nwixlogsorted{c}{{handle toolbus-to-tool communication}{NWwisH-hana-1}{\nwixu{NWwisH-recB-1}\nwixd{NWwisH-hana-1}}}%
\nwixlogsorted{c}{{handle tool-to-toolbus communication}{NWwisH-hana.2-1}{\nwixu{NWwisH-recB-1}\nwixd{NWwisH-hana.2-1}}}%
\nwixlogsorted{c}{{handle_input_from_toolbus}{NWwisH-hanP-1}{\nwixu{NWwisH-wisF-1}\nwixd{NWwisH-hanP-1}}}%
\nwixlogsorted{c}{{handle_input_from_wish}{NWwisH-hanM-1}{\nwixu{NWwisH-wisF-1}\nwixd{NWwisH-hanM-1}}}%
\nwixlogsorted{c}{{help}{NWwisH-hel4-1}{\nwixu{NWwisH-wisF-1}\nwixd{NWwisH-hel4-1}}}%
\nwixlogsorted{c}{{includes}{NWwisH-inc8-1}{\nwixu{NWwisH-wisF-1}\nwixd{NWwisH-inc8-1}}}%
\nwixlogsorted{c}{{is_from_tool_comm}{NWwisH-is*H-1}{\nwixu{NWwisH-wisF-1}\nwixd{NWwisH-is*H-1}}}%
\nwixlogsorted{c}{{is_to_tool_comm}{NWwisH-is*F-1}{\nwixu{NWwisH-wisF-1}\nwixd{NWwisH-is*F-1}}}%
\nwixlogsorted{c}{{main}{NWwisH-mai4-1}{\nwixu{NWwisH-wisF-1}\nwixd{NWwisH-mai4-1}}}%
\nwixlogsorted{c}{{notice}{NWwisH-not6-1}{\nwixu{NWwisH-wisF-1}\nwixd{NWwisH-not6-1}}}%
\nwixlogsorted{c}{{print_args}{NWwisH-priA.3-1}{\nwixu{NWwisH-wisF-1}\nwixd{NWwisH-priA.3-1}}}%
\nwixlogsorted{c}{{print_env}{NWwisH-pri9.2-1}{\nwixu{NWwisH-wisF-1}\nwixd{NWwisH-pri9.2-1}}}%
\nwixlogsorted{c}{{print_list}{NWwisH-priA.2-1}{\nwixu{NWwisH-wisF-1}\nwixd{NWwisH-priA.2-1}}}%
\nwixlogsorted{c}{{print_string}{NWwisH-priC-1}{\nwixu{NWwisH-wisF-1}\nwixd{NWwisH-priC-1}}}%
\nwixlogsorted{c}{{print_term}{NWwisH-priA-1}{\nwixu{NWwisH-wisF-1}\nwixd{NWwisH-priA-1}}}%
\nwixlogsorted{c}{{print_var}{NWwisH-pri9-1}{\nwixu{NWwisH-wisF-1}\nwixd{NWwisH-pri9-1}}}%
\nwixlogsorted{c}{{rec_ack_event}{NWwisH-recD-1}{\nwixu{NWwisH-wisF-1}\nwixd{NWwisH-recD-1}}}%
\nwixlogsorted{c}{{rec_do}{NWwisH-rec6-1}{\nwixu{NWwisH-wisF-1}\nwixd{NWwisH-rec6-1}}}%
\nwixlogsorted{c}{{rec_eval}{NWwisH-rec8-1}{\nwixu{NWwisH-wisF-1}\nwixd{NWwisH-rec8-1}}}%
\nwixlogsorted{c}{{rec_monitor}{NWwisH-recB-1}{\nwixu{NWwisH-wisF-1}\nwixd{NWwisH-recB-1}}}%
\nwixlogsorted{c}{{rec_monitor variables}{NWwisH-recL-1}{\nwixu{NWwisH-recB-1}\nwixd{NWwisH-recL-1}}}%
\nwixlogsorted{c}{{rec_terminate}{NWwisH-recD.2-1}{\nwixu{NWwisH-wisF-1}\nwixd{NWwisH-recD.2-1}}}%
\nwixlogsorted{c}{{require_fun}{NWwisH-reqB-1}{\nwixu{NWwisH-wisF-1}\nwixd{NWwisH-reqB-1}}}%
\nwixlogsorted{c}{{shutdown}{NWwisH-shu8-1}{\nwixu{NWwisH-wisF-1}\nwixd{NWwisH-shu8-1}}}%
\nwixlogsorted{c}{{signal_handler}{NWwisH-sigE-1}{\nwixu{NWwisH-wisF-1}\nwixd{NWwisH-sigE-1}}}%
\nwixlogsorted{c}{{signals_set}{NWwisH-sigB-1}{\nwixu{NWwisH-wisF-1}\nwixd{NWwisH-sigB-1}}}%
\nwixlogsorted{c}{{update notes}{NWwisH-updC-1}{\nwixu{NWwisH-recB-1}\nwixd{NWwisH-updC-1}}}%
\nwixlogsorted{c}{{update subscriptions}{NWwisH-updK-1}{\nwixu{NWwisH-recB-1}\nwixd{NWwisH-updK-1}}}%
\nwixlogsorted{c}{{update variables}{NWwisH-updG-1}{\nwixu{NWwisH-recB-1}\nwixd{NWwisH-updG-1}}}%
\nwixlogsorted{c}{{wish-adapter.c*}{NWwisH-wisF-1}{\nwixd{NWwisH-wisF-1}}}%
\nwixlogsorted{c}{{wish_create}{NWwisH-wisB-1}{\nwixu{NWwisH-wisF-1}\nwixd{NWwisH-wisB-1}}}%
\nwixlogsorted{c}{{wish_start}{NWwisH-wisA-1}{\nwixu{NWwisH-wisF-1}\nwixd{NWwisH-wisA-1}}}%
\nwixlogsorted{c}{{wprintf}{NWwisH-wpr7-1}{\nwixu{NWwisH-wisF-1}\nwixd{NWwisH-wpr7-1}}}%
\nwixlogsorted{c}{{wputc}{NWwisH-wpu5-1}{\nwixu{NWwisH-wisF-1}\nwixd{NWwisH-wpu5-1}}}%
\nwbegindocs{98}\nwdocspar

%}}}
   

\nwenddocs{}
