
\section{\label{PYintact}Python interactive interpreter simulator}

The Python interpreter will not enter interactive mode when it's input
comes from a pipe but it will first read a complete script (until EOF)
and then execute it.  However, for the Python adapter we need an
interactive interpreter since each function call needs to be evaluated
directly after it is send by the \TB.

The solution is the use of this script which simulates the Python
interactive interpreter by repeatedly reading a line from standard
input and executing it.
\nwfilename{PYintact.py.nw}\nwbegincode{1}\sublabel{NWPYiE-PYiB-1}\nwmargintag{{\nwtagstyle{}\subpageref{NWPYiE-PYiB-1}}}\moddef{PYintact.py~{\nwtagstyle{}\subpageref{NWPYiE-PYiB-1}}}\endmoddef\let\nwnotused=\nwoutput{}\nwstartdeflinemarkup\nwprevnextdefs{\relax}{NWPYiE-PYiB-2}\nwenddeflinemarkup
# -----------------------------------------------------------------------
# PYintact.py -- Python script which simulates interactive interpreter
#
# Author:       Leon Moonen,
# Time-stamp:   <95/08/30 19:11:35 leon>
# -----------------------------------------------------------------------

import sys
# add the path of the TB module to the module load-path
# this is changed to the right path during "make install" 
# (generally the path is 'root_of_ToolBus/lib')
# <PO>: Removed this line, in favor of environment variable mechanism
#       (because of the much cleaner installation procedure)
# sys.path.append('@prefix@/lib')
import TB
import string
import types
import __main__
\nwalsodefined{\\{NWPYiE-PYiB-2}\\{NWPYiE-PYiB-3}\\{NWPYiE-PYiB-4}}\nwnotused{PYintact.py}\nwendcode{}\nwbegindocs{2}\nwdocspar
The \txttt{firstof} function is used by the \txttt{catch} function
to get the first element of a sequence (in the case of an exception
value tuple, this first element contains the error description).
\nwenddocs{}\nwbegincode{3}\sublabel{NWPYiE-PYiB-2}\nwmargintag{{\nwtagstyle{}\subpageref{NWPYiE-PYiB-2}}}\moddef{PYintact.py~{\nwtagstyle{}\subpageref{NWPYiE-PYiB-1}}}\plusendmoddef\let\nwnotused=\nwoutput{}\nwstartdeflinemarkup\nwprevnextdefs{NWPYiE-PYiB-1}{NWPYiE-PYiB-3}\nwenddeflinemarkup
def firstof(seq):
        if (type(seq) == types.TupleType):
                return seq[0]
        elif (type(seq) == types.ListType):
                return seq[0]
        else:
                return seq
\nwendcode{}\nwbegindocs{4}\nwdocspar
The \txttt{catch} function is used my the main loop to catch all
possible exceptions (i.e. errors) while execution a line and
convert and return them to the Python adapter.
\nwenddocs{}\nwbegincode{5}\sublabel{NWPYiE-PYiB-3}\nwmargintag{{\nwtagstyle{}\subpageref{NWPYiE-PYiB-3}}}\moddef{PYintact.py~{\nwtagstyle{}\subpageref{NWPYiE-PYiB-1}}}\plusendmoddef\let\nwnotused=\nwoutput{}\nwstartdeflinemarkup\nwprevnextdefs{NWPYiE-PYiB-2}{NWPYiE-PYiB-4}\nwenddeflinemarkup
def catch(line):
        try:
                exec line in vars(__main__)
        except SystemExit:
                sys.exit()
        except:
                cmd = string.strip(line)
                TB.error("'%s: %s' while executing '%s'" %\\
                         (sys.exc_type, firstof(sys.exc_value), cmd))
\nwendcode{}\nwbegindocs{6}\nwdocspar
The main part of the program repeatedly reads a line from standard
input (whether or not that is a pipe) and executes it using the
\txttt{catch} function.
\nwenddocs{}\nwbegincode{7}\sublabel{NWPYiE-PYiB-4}\nwmargintag{{\nwtagstyle{}\subpageref{NWPYiE-PYiB-4}}}\moddef{PYintact.py~{\nwtagstyle{}\subpageref{NWPYiE-PYiB-1}}}\plusendmoddef\let\nwnotused=\nwoutput{}\nwstartdeflinemarkup\nwprevnextdefs{NWPYiE-PYiB-3}{\relax}\nwenddeflinemarkup
# main:
eof = 0
while not eof:
        try:
                line = sys.stdin.readline()     
                if (line == ''):
                        eof = 1
                else:
                        catch(line)
        except KeyboardInterrupt:
                eof = 1;
\nwendcode{}

\nwixlogsorted{c}{{PYintact.py}{NWPYiE-PYiB-1}{\nwixd{NWPYiE-PYiB-1}\nwixd{NWPYiE-PYiB-2}\nwixd{NWPYiE-PYiB-3}\nwixd{NWPYiE-PYiB-4}}}%
\nwbegindocs{8}\nwdocspar

\nwenddocs{}
