\noindent
\normalsize
 Compute the function $F$. Note the use of the result parameter {\tt Res}.

\small
\begin{verbatim}
process F(Z1 : real, Z2 : real, Z3 : real, Z4 : real, Res : real?) is
  let CdTdX2 : real
  in
     CdTdX2 := 0.01 .                                     %% arbitrary value for (c dt/dx)^2
     Res := radd(rsub(rmul(2.0, Z1), Z2),                 %% 2z1 - z2 +
                 rmul(CdTdX2,                             %%     (c dt/dx)^2 *
                     radd(rsub(Z3, rmul(2.0, Z1)), Z4)))  %%     (z3 - 2z1 + z4)
  endlet

\end{verbatim}
\noindent
\normalsize
 Process {\tt P} describes the behaviour of sample point {\tt I} with
 left neighbour {\tt L} and right neighbour {\tt R}.
 The amplitude in point {\tt I} at time $t - \Delta t$ and $t$
 is, respectively {\tt D} and {\tt E}.
 The current amplitude in point {\tt I} is written to display tool {\tt Tid}.
 The global behaviour of {\tt P} is:
 \begin{itemize}
 \item Receive the amplitudes of both neighbours.
 \item Send the amplitude {\tt E} to both neighbours.
 \item Compute the new amplitude {\tt E} at $t + \Delta t$ using
 auxiliary process {\tt F} defined above.
 \item Repeat these steps.
 \end{itemize}

\small
\begin{verbatim}
process P(Tid : display, L : int, I : int, R : int, Dstart : real, Estart : real) is
  let AL : real, AR : real, D : real, D1 : real, E : real
  in
     D := Dstart .
     E := Estart .
     ( (  rec-msg(L, I, AL?)        %% receive amplitude of left neighbour
       || rec-msg(R, I, AR?)        %% receive amplitude of right neighbour
       || snd-msg(I, L, E)          %% send own amplitude to left neighbour
       || snd-msg(I, R, E)          %% send own amplitude to right neighbour
       ||    %% update own amplitude on the display
             snd-do(Tid, update(I, E))
       ) .
       D1 := E .
       F(E, D, AL, AR, E?) .
       D := D1
     ) * delta
  endlet

\end{verbatim}
\noindent
\normalsize
 Define the processes at the end points. {\tt I} is the index
 of the end point, {\tt NB} is its immediate neighbour.

\small
\begin{verbatim}
process Pend(Tid : display, I : int, NB : int) is
  let W : real
  in
   ( rec-msg(NB, I, W?) || snd-msg(I, NB, 0.0) || snd-do(Tid, update(I, 0.0))) * delta
  endlet

\end{verbatim}
\noindent
\normalsize
 Construct the processes {\tt Pend}$_0$, {\tt P}$_1$, ..., {\tt P}$_{N-1}$, {\tt Pend}$_N$.

\small
\begin{verbatim}
process MakeWave(N : int) is
  let Tid : display, Id : int, I : int, L : int, R : int
  in
     execute(display, Tid?) .         %% create the display
     snd-do(Tid, mk-wave(N)) .        %% make an N point wave
     create(Pend(Tid, 0, 1), Id?).    %% create left end point
     L := sub(N,1) .
     create(Pend(Tid, N, L), Id?) .   %% create right end point
     I := 1 .                         %% create the P's in between
     if less(I, N) then
        L := sub(I, 1) . R := add(I, 1) .
        create(P(Tid, L, I, R, 1.0, 1.0), Id?) .
        I := add(I, 1)
     fi *
     shutdown("end") delay(sec(60))
  endlet

\end{verbatim}
\noindent
\normalsize
 Define the {\tt display} tool.

\small
\begin{verbatim}
tool display is { command = "wish-adapter -wish /usr/local/bin/wish -script ui-wave.tcl" }

\end{verbatim}
\noindent
\normalsize
 Define the initial \TB\ configuration.

\small
\begin{verbatim}
toolbus(MakeWave(8))
\end{verbatim}
\normalsize
