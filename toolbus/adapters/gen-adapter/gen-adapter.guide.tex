%%
%%    ToolBus -- The ToolBus Application Architecture
%%    Copyright (C) 1998-2000  Stichting Mathematisch Centrum, Amsterdam, 
%%                             The  Netherlands.
%%
%%    This program is free software; you can redistribute it and/or modify
%%    it under the terms of the GNU General Public License as published by
%%    the Free Software Foundation; either version 2 of the License, or
%%    (at your option) any later version.
%%
%%    This program is distributed in the hope that it will be useful,
%%    but WITHOUT ANY WARRANTY; without even the implied warranty of
%%    MERCHANTABILITY or FITNESS FOR A PARTICULAR PURPOSE.  See the
%%    GNU General Public License for more details.
%%
%%    You should have received a copy of the GNU General Public License
%%    along with this program; if not, write to the Free Software
%%    Foundation, Inc., 59 Temple Place, Suite 330, Boston, MA  02111-1307 USA
%%

\section{\label{ToolsFromUnix}Using arbitrary Unix commands as tool}

Using arbitrary Unix commands as \TB\ tool is achieved by the {\tt gen-adapter}
to be explained in Section~\ref{gen-adapter}.
An example of its is use is given in Section~\ref{Ex-pipe}.


\subsection{\label{gen-adapter}{\tt gen-adapter}}

\paragraph{Synopsis.} Execute an arbitrary Unix command as tool.

\paragraph{Example.} {\tt gen-adapter -cmd ls -l}

\paragraph{Specific arguments.}
\begin{itemize}
\item {\tt -addnewline}: always add a newline character to the standard input for the command.
\item {\tt -keepnewline}: keep the last newline character
in the output generated by the command. Without this argument, 
the last newline character is always removed.
\item \texttt{-string-output}: return the output of the command as string
(type: \texttt{<str>}). When no output format options has been specified,
\texttt{-string-output} is used.
\item {\tt -binary-output}: return the output of the command
as a binary string (type: {\tt <bstr>}).
\item \texttt{-term-output}: return the output of the command as term (type:
\texttt{<term>}).
\end{itemize}

\paragraph{Communication.}  \hspace{-0.3cm}\footnote{Communication is described
from the point of view of the \TB, i.e., {\tt snd-} and {\tt rec-}
mean, respectively, send by \TB\ and receive by \TB.}

\begin{itemize}
\item {\tt snd-eval($Tid$, cmd($Cmd$, input($Str$))}: execute the Unix command
\begin{quote}
        {\tt $Cmd$ < $Str$}
\end{quote}
i.e., execute $Cmd$ with $Str$ as standard input. The output of this command execution
is captured and will be returned by default as string value (see below).
Other output formats can be selected using command line options.
$Tid$ is a tool identifier 
(as produced by {\tt execute} or {\tt rec-connect}) for an instance of the {\tt gen-adapter}.

\item {\tt snd-eval($Tid$, cmd($Cmd$, input($Bstr$))}: same as above, except that
a binary string is used as standard input for $Cmd$.

\item {\tt rec-value($Tid$,output($Res$))}: the return value for a previous evaluation request.
$Res$ is a string containing the output produced by the command execution.
By default \texttt{gen-adapter} returns the output of the executed command
as ordinary string. Other output formats can be specified using command line
options. {\bf Note:} in the current implementation of {\tt gen-adapter}
there is an arbitrary limit (10000) on the {\em size} of the output produced
by the command.

\item {\tt snd-terminate($Tid$, $A_1$)}: terminate execution of {\tt gen-adapter}.
\end{itemize}

\subsection{\label{Ex-pipe}Example: pipe communication between two Unix commands}
Suppose we want to count how many words there are in a listing of the
current file directory. At the Unix level, this can be achieved by
\begin{verbatim}
   ls -l | wc
\end{verbatim}
where ``{\tt ls -l}'' produces the directory listing and
``{\tt wc -w}'' counts the number of words in this listing.
The same effect is achieved by the script given in Figure~\ref{fig:pipe.tb}.

\begin{figure}[tb]
\rule{\textwidth}{0.5mm}
\input{pipe.tb.tex}
  \caption{{\tt pipe.tb}: Executing the pipe line {\tt ls -l | wc} in a script.}
  \label{fig:pipe.tb}
\rule{\textwidth}{0.5mm}
\end{figure}
