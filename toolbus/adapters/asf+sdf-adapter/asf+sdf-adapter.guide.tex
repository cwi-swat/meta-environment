\section{\label{ToolsInASFSDF}Writing tools in \ASFSDF}

Writing \TB\ tools in \ASFSDF\ is greatly simplified by the adapters described in
this section:
\begin{itemize}
\item {\tt parser-adapter} (Section~\ref{parser-adapter}) is a universal parser 
interface primarily intended for use in the \ASFSDF\ Meta-environment.

\item {\tt asf+sdf-adapter} (Section~\ref{asf+sdf-adapter}) is a
universal adapter for compiled \ASFSDF\ specifications.

\end{itemize}

\paragraph{Note.} The connection of tools written in \ASFSDF\ to the \TB\ is
still under development. The descriptions of the adapters below may be
subject to change.

\subsection{\label{parser-adapter}{\tt parser-adapter}}

\paragraph{Synopsis.} Execute a parser as tool. For each request a new parser is executed.

\paragraph{Example.} {\tt parser-adapter -cmd parse-pascal}

\paragraph{Specific arguments.} None.

\paragraph{Communication.} \hspace{-0.3cm}\footnote{Communication is described
from the point of view of the \TB, i.e., {\tt snd-} and {\tt rec-}
mean, respectively, send by \TB\ and receive by \TB.}

\begin{itemize}
\item {\tt snd-eval($Tid$, input($S$))}: parse the string $S$.
$Tid$ is a tool identifier 
(as produced by {\tt execute} or {\tt rec-connect}) for an instance of the {\tt parser-adapter}.

\item {\tt rec-value($Tid$, output($G$?))}: the return value for a previous parse request
reporting successful completion of the parse.
$G$ is a term of type {\tt gel(<bstr>)} and contains a binary representation (in \GEL)
of the parse tree.

\item {\tt rec-value($Tid$,syntax-error($Line$?, $Pos$?, $Msg$?))}: the return value
for a previous parse request reporting a syntax error in the input text.
$Line$ and $Pos$ are coordinates in the text where the syntax error was found.
$Msg$ is a string describing the error.

\item {\tt rec-value($Tid$,error($Msg$?))}: the return value for a previous parse request
reporting a general error situation.

\item {\tt snd-terminate($Tid$, $A_1$)}: terminate execution of {\tt parser-adapter}.
\end{itemize}

\noindent The actual parse command to be executed should write on the standard error stream
error messages of the form
\begin{verbatim}
      syntax error at line %d, pos %d: %[^\n]\n
\end{verbatim}
and exit with non-zero status in case of a syntax error,
otherwise it should write a representation of the parse tree to its standard output.

\subsection{\label{asf+sdf-adapter}{\tt asf+sdf-adapter}}

\paragraph{Synopsis.} Execute a compiled \ASFSDF\ specification as tool.
Also provide a variable mechanism, to keep big terms in the adapter and
lighten the communication burden for the \TB.

\paragraph{Example.} {\tt asf+sdf-adapter -result gel -cmd Booleans}

\paragraph{Specific arguments.}
\begin{itemize}
\item {\tt -result {\em Kind}}: determines the format of the output term.
Possible values for {\em Kind} are: {\tt  gel} (produce a term in \GEL\ format),
or {\tt term} (produce a \TB\ term).

\item {\tt -details {\em String}}: define details of the mapping between
\TB\ terms and \ASFSDF\ terms. The mapping is defined by a list of triples
of the form
\begin{quote}
$Dir~~~~~TBpat~~~~~SDFdef$
\end{quote}
where
\begin{itemize}
\item $Dir$: gives the {\em direction} in which the mapping applies:
{\tt in} (input only, i.e., from \TB\ to specification),
{\tt out} (output only, i.e., from specification back to \TB),
{\tt inout} (both directions).
\item $TBpat$: is a term pattern (compare Section~\ref{Patterns}) defining
a function symbol for a \TB\ term. For {\tt out} or {\tt inout} mapppings,
only the [{\tt \%t}] directive is allowed. The following \emp{directives} are
allowed for {\tt in}put only mappings:
  \begin{itemize}
    \item [{\tt \%t}]: corresponds to a normal \TB\ term that should be handled
          recursively.
    \item [{\tt \%b}]: corresponds to a binary string that is taken to be
          in gel format.
    \item [{\tt \%l}]: corresponds to a toolbus list that is translated into
          a \ASFSDF\ list constructor.
    \item [{\tt \%s}]: corresponds to a toolbus string that is translated into
          a \ASFSDF\ lexical constructor.
  \end{itemize}
  The pattern should only be one level deep, i.e., all argument positions
  should be pattern directives.
  In adition to this, a directive of the form [{\tt \%c@var-name}] (with {\tt c}
  one of the characters {\tt t, b, l, or s} as described above) means
  that the corresponding argument is not present in the input term, but
  shoud be taken from variable {\tt var-name}.

  \item $SDFdef$: the literal text of the corresponding context-free function
  declaration from the \ASFSDF\ definition with which the function symbol
  defined in the previous point should correspond.
  \end{itemize}
\end{itemize}
Example use of a mapping for Boolean terms and strings:
\begin{verbatim}
tool string is { 
command = "asf+sdf-adapter -cmd bgelio spec/Editor -r1 -w1"
details = <<
inout   true            Booleans: "true" -> BOOL
inout   false           Booleans: "false" -> BOOL
in      and(%t,%t)      Booleans: and ( BOOL , BOOL ) -> BOOL
in	str-con(%s)     STR-CON
in	string(%t)      Strings: STR-CON -> STRING
in	correct(%t@s,%t) Strings: equal ( STRING , STRING ) -> BOOL
>>

process CHECK-PWD is
let
  Str : string, S : str
in
  execute(string, Str?) .
  snd-do(Str, set-variable("s", "top-secret")) .
  read("Type a password", S?) .
  snd-eval(Str, snd-eval(input(correct(string(str-con(S)))))) .
  ( rec-value(Str, output(true)) .
    printf("Password is correct.\n")
    +
    rec-value(Str, output(false)) .
    printf("You are an intruder!, your termination is imminent.\n")
  )
endlet

toolbus(CHECK-PWD)
\end{verbatim}
{itemize}

\paragraph{Communication.} \hspace{-0.3cm}\footnote{Communication is described
from the point of view of the \TB, i.e., {\tt snd-} and {\tt rec-}
mean, respectively, send by \TB\ and receive by \TB.}


\begin{itemize}
  \item {\tt snd-eval($Tid$, input($T$))}: rewrite the term $T$.
        There are two cases:
  \begin{itemize}
    \item $T$ is of type {\tt gel(<bstr>)} and contains a term in binary form (\GEL)
          to be reduced;
    \item otherwise, $T$ is a \TB\ term to be reduced.
  \end{itemize}
  $Tid$ is a tool identifier 
  (as produced by {\tt execute} or {\tt rec-connect}) for an instance of 
  the {\tt asf+sdf-adapter}.

  \item {\tt snd-do($Tid$, store(input($T$), $S$)}: rewrite the term $T$, 
        and store the result in variable $S$.
        There are here also two cases:
  \begin{itemize}
    \item $T$ is of type {\tt gel(<bstr>)} and contains a term in binary form (\GEL)
          to be reduced;
    \item otherwise, $T$ is a \TB\ term to be reduced.
  \end{itemize}

  \item {\tt rec-value($Tid$, output($T$?))}: the normal form computed for 
        a previous rewriting request. There are here also two cases:
  \begin{itemize}
    \item $T$ is of type {\tt gel(<bstr>)} and contains a term in binary form
          representing the normal form;
    \item otherwise, $T$ is a \TB\ term representing the normal form.
  \end{itemize}

  \item {\tt rec-value($Tid$, error($Msg$?))}: the return value for a previous
        rewriting request reporting a general error described by the string 
        $Msg$.

  \item {\tt snd-do(set-variable($S$, $T$))}: assign value $T$ to variable
        $S$.

  \item {\tt snd-eval(get-variable($S$))}: retrieve the contents of
        variable $S$.

  \item {\tt rec-value(variable($S$, $T$))}: the value of a variable
        requested by a previous retrieval request.

  \item {\tt snd-terminate($Tid$, $A_1$)}: terminate execution of {\tt asf+sdf-adapter}.
\end{itemize}

\paragraph{Implementation note} The current implementation is still feeble
in the following respects:
\begin{itemize}
\item The syntax checking of the triples in {\tt details} is marginal.
\item In the $SDFdef$ layout symbols are unfortunately {\bf essential}.
For instance, spaces should surround the parentheses and comma in
the definition of {\tt and}. In addition, quoting should be
{\bf exactly} as in the specification, e.g., {\tt true} and
{\tt "true"} are considered to be different.
\item Error handling is crude, in particular when a function
symbol appears in a normal form for which a mapping is missing.
\end{itemize}

\subsection{On the status of \GEL} The implementation of the \ASFSDF\ meta-environment
is currently in a transition phase regarding, among many other things, the use of \GEL.
Older components use a version of \GEL\  that encodes trees as ASCII texts, while
newer components use a newer version of \GEL\  that uses a much more concise binary
encoding. As a result, both versions have to coexist for some time. Inventory:
\begin{itemize}
\item all generated parsers use ASCII \GEL.
\item all compiled specifications use ASCII \GEL.
\item the {\tt asf+sdf-adapter} already uses binary \GEL.
\end{itemize}
To bridge this gap, several conversion programs are available:
\begin{itemize}
\item {\tt gel2a}: convert binary \GEL\ to ASCII \GEL.
\item {\tt gel2b}: convert ASCII \GEL\ to binary \GEL.
\item {\tt bgelo}: execute a program that generates ASCII \GEL\ and converts its output
to binary \GEL.
\item {\tt bgelio}: execute a program that reads and writes ASCII \GEL: first convert 
its input to binary, then execute the program, and finally convert its output
to binary \GEL.
\end{itemize}

\noindent The practical consequences of this are:
\begin{itemize}
\item Execute a parser, e.g., {\tt pascal} as follows: {\tt bgelo pascal}.
\item Execute a compiled \ASFSDF\ specification, e.g., {\tt Booleans} as follows:
{\tt bgelio Booleans r1 w1}.
\item The conversion programs {\tt bgelo} and {\t begelio} are located in the
same directory as all other programs related to the \TB.
\item Make sure that the conversion programs {\tt gel2a} and {\tt gel2b} are also
on your search path. Normally they reside in the ``{\tt gel}'' directory.
\end{itemize}
