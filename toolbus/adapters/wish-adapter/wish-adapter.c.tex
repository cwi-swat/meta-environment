\nwfilename{wish-adapter.c.nw}\nwbegincode{1}\sublabel{NWwisH-wisF-1}\nwmargintag{{\nwtagstyle{}\subpageref{NWwisH-wisF-1}}}\moddef{wish-adapter.c*~{\nwtagstyle{}\subpageref{NWwisH-wisF-1}}}\endmoddef\let\nwnotused=\nwoutput{}\nwstartdeflinemarkup\nwenddeflinemarkup
/* -*-C-*-
 * wish-adapter -- Generic adapter for ToolBus <-> wish connection
 *
 * Synopsis: wish-adapter [options] -script Script [-script-args A1..An]
 * options:
 *      -help                 print help message
 *      -wish Wish            use Wish as wish executable
 *      -lazy-exec            postpone execution of wish until needed
 *
 * where N is the toolname
 * and   S is a name of Tcl script to be executed
 * and   ... are arguments to be passed on to wish
 *
 * Author: Paul Klint, March, 1994 & Februari 1995

 * May 24, Pieter Olivier
 * - added -lazy-exec option
 * - added -wish option
 * March 1994, Pieter Olivier
 * - added support for binary strings and lists
 * April 5, Paul Klint
 * - clean up of terminate/disconnect code
 * - monitor code surrounded by a catch construct
 *
 * Purpose: pass term coming from ToolBus to wish, e.g.,
 *
 *    do(mk-button(3))  ==> mk-button 3
 *
 *
 * Architecture:
 *
 *         ================== ToolBus
 *           |           ^
 *           v           |    (sockets)
 *         +---------------+
 *         | wish-adapter  |
 *         +---------------+
 *           |           ^
 *           |           |
 *           |  +------+ |
 *           +->| wish |-+    (standard input/output)
 *              +------+
 *
 */

#include "TB.h"
#include <unistd.h>
#include <signal.h>

FILE *to_wish;  /* file descriptor connected to std input of wish */
int pgid;       /* pid group of the wish children */

/* Some utility functions */
void pr_type_wish(type *tp);
void pr_term_list_wish(const term_list *tl, char sep);
void pr_env_wish(const env *e);
void pr_string_wish(char *s, int len);
term *handle_input_from_wish(term *e);
void connect_to_wish(char *script, char *name, TBcallbackTerm handler);

/* arguments passed to script */

char *def_args[] = \{NULL\};
char **script_args = def_args;

/* Variables needed for lazy execution */
TBbool lazy_exec = TBfalse;
char *name = "wish-adapter";
char *script = NULL;
char *wish_exec = "wish";

/* Signal handler */
static void signal_handler( int sig )
\{
  if(pgid != -1) \{
    TBsend(TBmake("snd-disconnect"));
    sleep(1);
    kill(-pgid, sig);
  \}
  exit(0);
\}

TBbool is_to_tool_comm(char *s)
\{
  return streq(s, "snd-eval") || streq(s, "snd-cancel") || 
    streq(s, "snd-do") || streq(s, "snd-ack-event") ||
    streq(s, "snd-terminate");
\}

TBbool is_from_tool_comm(char *s)
\{
  return streq(s, "rec-value") ||  streq(s, "rec-event") || streq(s, "rec-disconnect");
\}

term *dummy_check_in_sign(term *t)
\{
  return NULL;
\}

require_fun(char *fname, term_list *fargs)
\{
  TBprintf(to_wish, "TBrequire %s %s %d\\n", tool_name, fname, list_length(fargs));
\}

check_in_sign()
\{ char *atf, *tn;
  term *tid;
  term_list *arg;
  extern term_list *tool_in_sign;
  term_list *reqs = tool_in_sign;
  char pat[128];

  /* construct match pattern, e.g. ``%f(<calc>,%l)'' */
  sprintf(pat, "%%f(<%s>, %%l)", tool_name); 

  for(; reqs; reqs = next(reqs))\{
    if(TBmatch(first(reqs), pat, &atf, &arg))\{
        if(streq(atf, "rec-do") || streq(atf, "rec-eval"))
          require_fun(get_txt(fun_sym(first(arg))), fun_args(first(arg)));
        else if(streq(atf, "rec-ack-event"))
          require_fun("rec-ack-event", arg);
        else if(streq(atf, "rec-terminate"))
          require_fun("rec-terminate", arg);
        else
          TBmsg("check_in_sign: skipped %t\\n", first(reqs));
  \} else
        TBmsg("check_in_sign: skipped %t\\n", first(reqs));      
  \}
\}

int bytes_in_term(term *t)
\{
  return 8 /* LENSPEC */ + strlen(TBsprintf("%t", t));
\}

char *wish_type_string(tkind kind)
\{
  static char *types[] =
    \{ "term", "bool", "int", "real", "str", "bstr", "var",
      "placeholder", "appl", "env", "list"
    \};

  if(kind >= t_term && kind <= t_list)
    return types[kind];

  return "unknown";
\}

void print_escaped_char(char c)
\{
  /* In a string, we just escape the character. */
  char Buf[6];
  int i;
  
  sprintf(Buf, "\\\\%o", ((unsigned int)c) & 0xFF);
  for(i=0; i<strlen(Buf); i++)
    fputc(Buf[i], to_wish);
\}

void printn_wish(const char *s, int n)
\{
  static TBbool instring = TBfalse;
  static TBbool prev_escaped = TBfalse;
  static int escaped = TBfalse;

  while(n)\{
    if(*s == '"') \{
      putc('\\\\', to_wish);
      if(!escaped) \{
        if(instring) \{
          instring = TBfalse;
          prev_escaped = TBfalse;
        \} else
          instring = TBtrue;
      \}
    \}
    escaped = TBfalse;
    if(instring)
      \{
        if(*s == '\\\\' || *s == '[' || *s == ']')
          fputc('\\\\', to_wish);
        prev_escaped = TBfalse;
      \}

    if(isprint(*s))
      \{
        /* We don't want an octal digit after an escaped number! */
        if(prev_escaped && *s >= '0' && *s <= '7')
          print_escaped_char(*s++);
        else \{
          if(*s == '\\\\' && !instring) \{
            escaped = TBtrue;
            fputc('\\\\', to_wish);
          \} 
          fputc(*s++, to_wish);
        \}
        prev_escaped = TBfalse;
      \}
    else
      \{
        /* We have a non-printable character */
        print_escaped_char(*s++);
        prev_escaped = TBtrue;
      \}

    n--;
  \}
\}

void pr_term_wish(term *t)
\{
  char cbuf[100], *ftxt;

  /* TBprintf(stderr, "Term: %t\\n", t);*/
  if(!t)\{
    printn_wish("\{\}",2);
    return;
  \}
  switch(tkind(t))
    \{
    case t_str:
      pr_string_wish(str_val(t), strlen(str_val(t)));
      break;

    case t_bstr:
      pr_string_wish(bstr_val(t), bstr_len(t));
      break;

    case t_bool:
      if(bool_val(t)) 
        printn_wish("true", 4);
      else
        printn_wish("false", 4);
      break;

    case t_int:
      sprintf(cbuf, "%d", int_val(t));
      printn_wish(cbuf, strlen(cbuf));
      break;

    case t_real:
      sprintf(cbuf, "%f", real_val(t));
      printn_wish(cbuf, strlen(cbuf));
      break;

    case t_var: /* Can't occur ??? */
      ftxt = get_txt(var_sym(t));
      printn_wish(ftxt, strlen(ftxt));
      pr_type_wish(var_type(t));
      if(var_result(t))
        printn_wish("?",1);
      break;

    case t_placeholder:
      printn_wish("<",1);
      pr_term_wish(placeholder_type(t));
      printn_wish(">",1);
      break;

    case t_appl:
      ftxt = get_txt(fun_sym(t));
      printn_wish(ftxt, strlen(ftxt));
      if(fun_args(t) != NULL || isupper(ftxt[0]))\{
        printn_wish("(", 1);
        pr_term_list_wish(fun_args(t), ',');
        printn_wish(")", 1);
      \}
      break;

    case t_list:
      printn_wish("\{", 1); 
      pr_term_list_wish(t, ' ');
      printn_wish("\}", 1);
      break;

    case t_env:
      pr_env(t);
      break;
    \}
\}

void pr_string_wish(char *s, int len)
\{
  fputc('\{', to_wish);
/*  printn_wish("\\"", 1);*/
  printn_wish(s, len);
  fputc('\}', to_wish);
/*  printn_wish("\\"", 1);*/
\}

void pr_type_wish(type *tp)
\{
  printn_wish(":", 1); 
  pr_term_wish(tp);
\}

void pr_term_list_wish(const term_list *tl, char sep)
\{
  TBbool first = TBtrue;

  for( ; tl; tl = next(tl))\{
    assert(is_list(tl));    
    if(first)
      first = TBfalse;
    else
      printn_wish(&sep, 1);
    pr_term_wish(first(tl));
  \}
\}

void pr_env_wish(const env *e)
\{
  TBbool first = TBtrue;
  char *ftxt;

  printn_wish("\{", 1);
  for( ; e; e = env_next(e))\{
    assert(is_env(e));    
    if(first)
      first = TBfalse;
    else
      printn_wish(" ", 1);
    printn_wish("\{", 1);
    ftxt = get_txt(env_sym(e));
    printn_wish(ftxt, strlen(ftxt));
    printn_wish(" ", 1);
    pr_term_wish(env_val(e));
    printn_wish("\}",1);
  \}
  printn_wish("\}", 1);
\}


print_escaped(term *t)
\{
  char *s = TBsprintf("%t", t);
  int c;
  TBbool instring = TBfalse;

  fputc('\{', to_wish);
  while(*s)\{
    c = *s++;
    switch(c)\{
    case '\{': case '\}':
      if(!instring)
        fputc('\\\\', to_wish);
      break;
    case '"':      
      if(instring)
        instring = TBfalse;
      else
        instring = TBtrue;
      break;
    case '\\\\':
      fputc('\\\\', to_wish);
      if(*s)
        fputc(*s++, to_wish);
      continue;
    default:;
    \}
    fputc(c, to_wish);
  \}
  fputc('\}', to_wish);
  fputc(' ', to_wish);
\}


term *handle_input_from_toolbus(term *e)
\{
  term *trm;  

  int pid1, pid2;
  term *Env, *Subs, *Notes, *AtArgs, *Coords;
  char *AtFun;
  char *fname;
  term_list *fargs;
  term *farg;
  term *pe;
  char *mon_point;
  
  /*TBmsg("received %t from toolbus\\n", e);*/
  if(TBmatch(e, "rec-monitor(%f(%d,%f,%t,%t,%t,%t,%t,%d,%t))",
                    &mon_point,
             &pid1, &AtFun, &AtArgs, &Coords,
             &Env, &Subs, &Notes, &pid2, &pe))
    \{

      term_list *ts = Env, *pair, *args;
      term *var;
      char *name, *tname;
      char *filename;
      int blino, bpos, elino, epos;

      /* all monitor commands are generated inside a big catch construct */
      TBprintf(to_wish, "if [catch \{ ");

      if(!Coords)\{
        filename = "INIT"; blino=elino=bpos=epos=0;
      \} else if(!TBmatch(Coords, "[%s,%d,%d,%d,%d]", 
                         &filename, &blino, &bpos, &elino, &epos))
        TBmsg("**** coords do not match\\n");

      if(streq(AtFun, "create"))\{                   /* create process */
/* TBmsg("create\\n"); */
        if(!TBmatch(AtArgs, "[%f(%l), %t]", &name, &args, &var))
          TBmsg("**** args of create do not match\\n");

        TBprintf(to_wish, "create_proc %t %s\\n", list_get(Env, var), name);
      \} else if(streq(AtFun, "rec-connect"))\{         /* create tool */  
        int tid;
/* TBmsg("rec-connect\\n");       */  
        if(!TBmatch(AtArgs, "[%t]", &var))
          TBmsg("**** args of rec-connect do not match: %t\\n", AtArgs);
        if(!TBmatch(get_list_as_env(var, Env), "%f(%d)", &name, &tid))\{
          TBmsg("**** rec-connect: value of var does not match: var=%t, env=%t", var, Env);
          TBmsg("**** value_list(var, Env) = %t\\n", get_list_as_env(var, Env));
        \}

        TBprintf(to_wish, "create_tool %d %s\\n", tid, name);
      \} else if(is_to_tool_comm(AtFun))\{            /* any comm to a tool */
        int tid;
/* TBmsg("to_tool\\n"); */
        if(!TBmatch(AtArgs, "[%t, %l]", &var, &args))
          TBmsg("**** args of tool comm do not match:%t\\n", AtArgs);
        if(!TBmatch(get_list_as_env(var, Env), "%f(%d)", &name, &tid))
          TBmsg("**** to_tool: value of var does not match: var=%t, env=%t", var, Env);

        TBprintf(to_wish, "proc_tool_comm %d %d %d\\n", pid1, tid, bytes_in_term(args)); 
      \} else if(is_from_tool_comm(AtFun))\{            /* any comm from a tool */
        int tid;
/* TBmsg("from_tool\\n");*/
        if(!TBmatch(AtArgs, "[%t, %l]", &var, &args))
          TBmsg("**** args of tool comm do not match:%t\\n", AtArgs);
        if(!TBmatch(get_list_as_env(var, Env), "%f(%d)", &name, &tid))
          TBmsg("**** from_tool: value of var does not match: var=%t, env=%t", var, Env);
 
        TBprintf(to_wish, "tool_proc_comm %d %d %d\\n", pid1, tid, bytes_in_term(args)); 
      \}
            
      if(pid1 > 0)\{
        for(ts = Env; ts; ts = next(ts))\{
          pair = first(ts);
        if(get_txt(var_sym(first(pair)))[0] != '_')             
          TBprintf(to_wish, "update_var %d \{%t\} \{%t\}\\n", 
                   pid1, first(pair), first(next(pair)));
        \}
        for(ts = Subs;ts; ts=next(ts))\{
          TBprintf(to_wish, "update_subs %d \{%t\}\\n", pid1, first(ts));  
        \}
        for(ts = Notes; ts; ts=next(ts))\{
          TBprintf(to_wish, "update_notes %d \{%t\}\\n", pid1, first(ts));  
        \}
      \}
   
      if(pid2 > 0)\{
        int dir = streq(AtFun, "snd-msg") ? 1 : -1;
        TBprintf(to_wish, "proc_proc_comm %d %d %d\\n", pid1, pid2, dir);
      \}

      if(streq(AtFun, "endlet"))\{
        if(!TBmatch(AtArgs, "[%l]", &args))
          TBmsg("**** args of end_let do not match\\n");
        for( ; args; args = next(args))\{
          TBprintf(to_wish, "delete_var %d \{%t\}\\n", pid1, first(args));
        \}
        /* complete the surrounding catch construct */
        TBprintf(to_wish, "\} msg] \{ TBerror $msg \}\\n");
        return TBmake("snd-continue(%d)", pid1);
      \} else    
        TBprintf(to_wish, "monitor_atom %d %s %s %d %d %d %d\\n", 
                 pid1, AtFun, filename, blino, bpos, elino, epos);

    /* complete the surrounding catch construct */
    TBprintf(to_wish, "\} msg] \{ TBerror $msg \}\\n");
/* TBmsg("wish-adapter returns\\n"); */
      return NULL;   
    \} else if(TBmatch(e, "rec-do(%f(%l))", &fname, &fargs) ||
              TBmatch(e, "rec-eval(%f(%l))", &fname, &fargs))\{
      TBprintf(to_wish, "if [catch \{%s ", fname);
      for( ; fargs ; fargs = next(fargs))
        \{
          pr_term_wish(first(fargs));
          printn_wish(" ", 1);
        \}
      TBprintf(to_wish, "\} msg] \{ TBerror $msg \}\\n");
      return NULL;
    \} else if(TBmatch(e, "rec-ack-event(%t)", &farg))\{
      TBprintf(to_wish, "if [catch \{rec-ack-event \{");
      pr_term_wish(farg);
      TBprintf(to_wish, "\}\} msg] \{ TBerror $msg \}\\n");

/*      TBprintf(to_wish, "if [catch \{rec-ack-event \{%t\}\} msg] \{ TBerror $msg \}\\n", farg);*/
      return NULL;
    \} else if(TBmatch(e, "rec-terminate(%t)", &farg))\{
      TBprintf(to_wish, "if [catch \{rec-terminate \{%t\}\} msg] \{ TBerror $msg \};exit\\n", farg);
      sleep(1);
      kill(-pgid, SIGKILL);   /* the wish child */ 
      exit(0);
    \}
    TBmsg("Ignored: %t\\n", e);
    return NULL;
\}

/* NOTE: in the above code the arguments of ack-event and terminate should be escaped! */
 
term *handle_input_from_wish(term *e)
\{
  char *msg;     
           
  /*TBmsg("handle_input_from_wish(%t)\\n", e);*/
  if(TBmatch(e, "wish-error(%s)", &msg))\{
    TBmsg("wish-error: %s\\n", msg);
    return NULL;
  \} else if(TBmatch(e,"snd-disconnect"))\{
    TBsend(e);
    /* <PO>: pid bug fixed! (thanks to Merijn de Jonge) */
    kill(-pgid, SIGKILL);   /* the wish child */
    sleep(1);  /* make sure any incoming data will be consumed */
    exit(0);
  \}
  return e;
\}

void connect_to_wish(char *script, char *name, TBcallbackTerm handler)
\{
  int pid;
  int ui2wish[2];
  int wish2ui[2];
  int old_stdin, old_stdout, from_wish;
  int n_script_args;
  char **p;

  if(pipe(ui2wish) < 0 || pipe(wish2ui) < 0)\{
    TBmsg("Can't create pipes"); exit(1);
  \}
  old_stdin = dup(0);
  old_stdout = dup(1);

  to_wish = fdopen(ui2wish[1], "w");
  from_wish = wish2ui[0];

  close(0); close(1);

  if(dup(ui2wish[0]) < 0 || dup(wish2ui[1]) < 0)\{
    TBmsg("Can't dup (1)\\n"); exit(1);
  \}
  if((pid = fork()))\{
    /* ui: the parent */
    if(pid < 0)\{
      TBmsg("Can't fork\\n"); exit(1);
    \}
    pgid = pid;
    setpgid(pid, pgid);
    close(0); close(1);
    dup(old_stdin);
    dup(old_stdout);
    close(old_stdin);
    close(old_stdout);
    TBprintf(to_wish, "source %s\\n", TBTCL);
    TBprintf(to_wish, "set argv \{");
    n_script_args = 0;
    for(p = script_args; *p; p++)\{
      n_script_args++;   
      TBprintf(to_wish, "%s ", *p);
    \}  
    TBprintf(to_wish, "\}\\nset argc %d\\n", n_script_args);
    TBprintf(to_wish, "source %s\\n", script);
    check_in_sign();
    TBaddTermPort(from_wish, handler);
  \} else \{
    /* wish: the child */
    if(execlp(wish_exec, name, NULL) < 0)\{
      TBmsg("Can't execute wish\\n");
      exit(1);
    \}
  \}
\}

void help(void)
\{
  char * str =
"\\n\\
Synopsis: wish-adapter [options]\\n\\
\\n\\
Options are:\\n\\
-help                 print this message\\n\\
-wish Wish            use Wish as wish executable\\n\\
-lazy-exec            postpone execution of wish until needed\\n\\
-script Name          use Name as Tcl script for wish\\n\\
-script-args A1 ...   use A1 ... as arguments for the wish execution\\n";
  fprintf(stderr, str);
\}

void main(int argc, char *argv[])
\{
  int i = 1;
  FILE *f;

  while(i < argc)\{
    if(streq(argv[i], "-help"))\{
      help();
    \} else if(streq(argv[i], "-TB_TOOL_NAME"))\{
      name = argv[i+1]; i += 1;
    \} else if(streq(argv[i], "-script"))\{
      script = argv[i+1]; i += 1;
    \} else if(streq(argv[i], "-lazy-exec"))\{
      lazy_exec = TBtrue;
    \} else if(streq(argv[i], "-wish"))\{
      wish_exec = argv[++i];
    \} else if(streq(argv[i], "-script-args"))\{
      script_args = &argv[i+1];
      break;
    \}
    i++;
  \}
  TBinit(name, argc, argv, handle_input_from_toolbus, dummy_check_in_sign);

  signal(SIGINT,  signal_handler);
  signal(SIGTERM, signal_handler);
  signal(SIGHUP,  signal_handler);
  signal(SIGQUIT, signal_handler);
 
  if(!script)
    err_fatal("Missing -script argument");

  if((f = fopen(TBTCL, "r")))
    fclose(f);
  else
    err_sys_fatal("Can't open tcl script: %s", TBTCL);
  if((f = fopen(script, "r")))
    fclose(f);
  else
    err_sys_fatal("Can't open tcl script: %s\\n", script);
    
  while(lazy_exec && !TBpeek())
    sleep(1);

  connect_to_wish(script, name, handle_input_from_wish);

  TBeventloop();
\}
\nwnotused{wish-adapter.c*}\nwendcode{}

\nwixlogsorted{c}{{wish-adapter.c*}{NWwisH-wisF-1}{\nwixd{NWwisH-wisF-1}}}%
\nwbegindocs{2}\nwdocspar
\nwenddocs{}
