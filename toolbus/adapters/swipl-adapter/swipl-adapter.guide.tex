
%
% The real SWI-Prolog section starts here.
%
\section{Writing tools in SWI-Prolog}

Writing \TB\ tools in SWI-Prolog is greatly simplified by the
{\tt swipl-adapter} described in this section.

\subsection{\tt swipl-adapter}

\paragraph{Synopsis.} Execute a SWI-Prolog program as a tool. For each request
a query is sent to the prolog program and the result sent back to the \TB\.

\paragraph{Example.} {\tt swipl-adapter -binary a.out}

\paragraph{Specific arguments.}
\begin{itemize}
\item {\tt -binary}: The SWI-Prolog compiled binary to send queries to.
If this flag is not present, the queries will be sent to the standard
SWI-Prolog system. If all you need is SWI-Prolog built-ins, you don't have
to specify a binary.
\end{itemize}

\paragraph{Communication.} \hspace{-0.3cm}\footnote{Communication is described
from the point of view of the \TB, i.e., {\tt snd-} and {\tt rec-}
mean, respectively, send by \TB\ and receive by \TB.}

\begin{itemize}
\item {\tt snd-eval($Tid$, query($Query$))} Send the query $Query$ to the
prolog system. All \TB\ data types can be used.

Where one would put a variable (an identifier starting with a capital) in
SWI-Prolog, one should put the atom {\tt pl-var} here.
This tells the {\tt swipl-adapter} and by way of the {\tt swipl-adapter}
the SWI-Prolog system that the value should be treated as an unbound variable,
as if the user entered a variable on the SWI-Prolog prompt.

\item {\tt rec-value($Tid$, result($Success$, $Query$))} Receive the result of
the query $Query$ to the prolog system. $Success$ will equal {\tt true} if the
query succeeded (SWI-Prolog would say "Yes.") and will equal
{\tt false} if the query failed (SWI-Prolog would say "No.").

If $Success$ equals {\tt true}, $Query$ contains the query that was sent using
{\tt snd-eval}, but unbound variables ({\tt pl-var} atoms in the original
query) have now been assigned a value.

For example, sending {\tt snd-value($Tid$, query(plus(2, 3, pl-var)))}
will return\\
{\tt rec-value($Tid$, result(true, plus(2, 3, 5)))}.

\item {\tt rec-value($Tid$, error($Msg$))} Receive and error message from
the {\tt swipl-adapter}. These errors should not occur under normal
circumstances and usually mean that the {\tt swipl-adapter} has encountered
an unknown \TB\ or SWI-Prolog data type.
\end{itemize}

The SWI-Prolog system has been integrated into the {\tt swipl-adapter}. When
it is initialized it is given the {\tt -x} flag with the binary to execute.
After that each query is sent to the SWI-Prolog system to evaluate.

The binary that is needed should be compiled by the SWI-Prolog system by
giving the system the {\tt -c} flag and the filename to compile. It will compile
the prolog program and the output will be put in a file aptly named
{\tt a.out}. To get the output in another file use the {\tt -o} switch.

\subsection{An example}

As an example we will demonstrate how to use the {\tt swipl-adapter} to
calculate the successor of a character. The following steps should be
followed:

\begin{itemize}
\item Write the SWI-Prolog code. The result is shown in
Figure~\ref{fig:succ.pl}.
\item Write the \T\ script. The result is shown in
Figure~\ref{fig:succ.tb}.
\item Compile the prolog code; {\tt swipl -o succ.pl.o -c succ.pl}
\item Run the system; {\tt toolbus succ.tb}
This will produce the output shown in
Figure~\ref{fig:succ.output}
\end{itemize}

\begin{figure}
\rule{\textwidth}{0.5mm}
\begin{verbatim}
volgend(a,b).
volgend(b,c).
volgend(c,d).
volgend(d,e).
volgend(e,f).
\end{verbatim}

	\caption{{\tt succ.pl}: the successor tool in SWI-Prolog}
	\label{fig:succ.pl}
\rule{\textwidth}{0.5mm}
\end{figure}

\begin{figure}
\rule{\textwidth}{0.5mm}
\begin{verbatim}
process SUCC is
    let P : prolog,
        SUCC : term
    in
        execute(prolog, P?) .

        snd-eval(P, query(volgend(a,pl-var))) .
        rec-value(P, result(true, volgend(a, SUCC?))) .
        printf("The successor of a is %t.\n", SUCC) .

        snd-eval(P, query(volgend(b,c))) .
        rec-value(P, result(true, volgend(b,c))) .
        printf("The successor of b really is c.\n") .

        snd-eval(P, query(volgend(a,f))) .
        rec-value(P, result(false, volgend(a,f))) .
        printf("The successor of d is not f.\n")

    endlet

tool prolog is {command = "swipl-adapter -binary succ.pl.o"}

toolbus(SUCC)
\end{verbatim}

	\caption{{\tt succ.tb}: the successor \T\ script}
	\label{fig:succ.tb}
\rule{\textwidth}{0.5mm}
\end{figure}

\begin{figure}
\rule{\textwidth}{0.5mm}
\begin{verbatim}
The successor of a is b.
The successor of b really is c.
The successor of d is not f.
\end{verbatim}

	\caption{The output of the \T\ script}
	\label{fig:succ.output}
\rule{\textwidth}{0.5mm}
\end{figure}

\end{document}
