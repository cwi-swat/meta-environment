
\section{\label{ToolsInSwipl}Writing tools in SWI-Prolog}

You can write \TB\ tools in Prolog using a \TB -aware version of
SWI-Prolog.
How to build such a \TB -aware version of SWI-Prolog is explained in
Section \ref{build-swipl}. Section \ref{using-swipl} explains how
to connect your Prolog programs to the \TB.

\subsection{\label{build-swipl}Building a \TB -aware SWI-Prolog interpreter.}

Before adding \TB\ support to SWI-Prolog, you first have to retrieve and
install SWI-Prolog, version 2.7.8 or later.
If you need more information about SWI-Prolog, or more
specific information about installing SWI-Prolog, you can visit the
\emp{SWI-Prolog} at 
{\tt http://swi.psy.uva.nl/usr/jan/SWI-Prolog.html. In this document
we assume you have succesfully installed SWI-Prolog and are only interested
in adding \TB\ support.

You first need to change the first two lines of the file 
{\tt adapters/swipl-adapter/Makefile.in}, because this is the location
where SWI-Prolog can be found. You also might need to change the
{\tt EXTRA\_LIBS} variable in {\tt Makefile.in}, depending on your 
system.

Reconfigure the ToolBus to include
SWI-Prolog support: {\tt ./configure --with-swipl} (in the main ToolBus
directory).
Now type {\tt make install} and keep your fingers crossed.

The program {\tt tbswipl}, which is installed in the ToolBus bin
directory, can be used to compile your SWI-Prolog programs into
executables. These executables only contain the byte-compiled Prolog
program and a reference to the {\tt tbswipl} program that can interpret
such programs.

The shell script {\tt swipl-adapter}, located in the ToolBus bin directory,
is used by your \TB\ scripts to start
the \TB -aware SWI-Prolog executable with the right arguments. 

\subsection{\label{compiling-swipl}Compiling SWI-Prolog programs using
{\tt tbswipl}}

\paragraph{Synopsis.} Compile a SWI-Prolog program.

\paragraph{Example.} {\tt tbswipl -o testpl -c test.pl}

\paragraph{Specific argmuments.}
\begin{itemize}
  \item {\tt tbswipl -help}: Display help message;
  \item {\tt tbswipl -v}:    Display version information;
  \item {\tt swipl [options]}: Start an interactive SWI-Prolog session.;
  \item {\tt swipl [options] [-o output] -c file ...}:
	Compile a SWI-Prolog program into byte code;
  \item {\tt swipl [options] [-o output] -b bootfile -c file ...}:
        Boot compilation. For system maintenance only.
\end{itemize}

\paragraph{Options.}
\begin{itemize}
 \item {\tt -x state}: Start from state (must be first);
 \item {\tt -[LGTAP]kbytes}: Specify {Local,Global,Trail,Argument,Lock} stack sizes;
 \item {\tt -B}: Small stack sizes to prepare for boot;
 \item {\tt -t toplevel}: Toplevel goal;
 \item {\tt -g goal}: Initialisation goal;
 \item {\tt -f file}: Initialisation file;
 \item {\tt [+/-]tty}: Allow tty control;
 \item {\tt -O}: Optimised compilation.
\end{itemize}
Other options can be found in the SWI-Prolog manual.

\subsection{\label{using-swipl}Using the SWI-Prolog adapter: 
{\tt swipl-adapter}}

\paragraph{Example.} {\tt swipl-adapter -program testpl}

\paragraph{Specific arguments.}
\begin{itemize}
\item {\tt -program $Name$}: Use $Name$ as the Prolog byte-code program
to execute.
\item {\tt -goal $Goal$}: Use $Goal$ as the toplevel goal to execute.
This toplevel goal usually initializes the ToolBus connection.
\end{itemize}

\paragraph{Communication.} \hspace{-0.3cm}\footnote{Communication is described
from the point of view of the \TB, i.e., {\tt snd-} and {\tt rec-}
mean, respectively, send by \TB\ and receive by \TB.}

\begin{itemize}
\item {\tt snd-do($Tid$, $Fun$($A_1$, ..., $A_n$))}: perform the Prolog
query
{\tt $Fun$($cid$,$A_1$, ..., $A_n$)}. Here $Tid$ is a tool identifier
(as produced by {\tt execute} or {\tt rec-connect}) for an instance of the {\tt
python-adapter}, and $cid$ is the connection id for this tools as returned
by {\tt tb\_new\_connection}.
\item {\tt snd-eval($Tid$, $Fun$($A_1$, ..., $A_n$))}: perform the Prolog
query
{\tt $Fun$($cid$, $A_1$, ..., $A_n$}. $Tid$ and $cid$ as above.
If the query succeeds, any uninstantiated variables in $A_1$ to $A_n$ are
replaced with their (first) solution, and the complete term is returned
using {\tt rec-value($Tid$, $Fun$($A_1$, ..., $A_n$}.
If the query fails, the term {\tt rec-value($Tid$, no)}
is returned.
\item {\tt rec-value($Tid$,$Res$)}: the return value for a previous evaluation request.
\item {\tt rec-event($Tid$, $A_1$, ..., $A_n$)}: event generated by Prolog.
\item {\tt snd-ack-event($Tid$, $A_1$)}: acknowledgement of
a previously generated event. Perform the Prolog query
{\tt rec\_ack\_event($cid$, $A_1$)}.
\item {\tt snd-terminate($Tid$, $A_1$)}: terminate execution of the
Prolog tool. Perform the Prolog query
{\tt rec\_terminate($cid$, $A_1$)}.
\end{itemize}


\subsection{\label{Prolog-predicates}Predefined SWI-Prolog predicates}
The following Prolog predicates are predefined and can be used freely in
Prolog programs executed by a \TB -aware SWI-Prolog byte-code interpreter.

\begin{itemize}

\item {\tt tb\_tool(-$Tool$)}: $Tool$ is unified with the default tool name.
      This is the tool name that is given as the {\tt -TB\_TOOL\_NAME $name$}
      argument to the {\tt swipl-adapter}.

\item {\tt tb\_host(-$Host$)}: $Host$ is unified with the default host.
      This is the host name that is given as the {\tt -TB\_HOST $host$}
      argument to the {\tt swipl-adapter}.

\item {\tt tb\_port(-$Port$)}: $Port$ is unified with the default port.
      This is the port number that is given as the {\tt -TB\_PORT $port$}
      argument to the {\tt swipl-adapter}.

\item {\tt tb\_tid(-$Tid$)}: $Tid$ is unified with the tool id.
      This is the port number that is given as the {\tt -TB\_TOOL\_ID $tid$}
      argument to the {\tt swipl-adapter}.

\item {\tt tb\_new\_connection(+$Tool$, +$Host$, +$Port$, +$Tid$, -$Cid$)}: 
      Create a new tool instance. $Tool$ and $Host$ are
      strings, $Port$, $Tid$, and $Cid$ are integers. 
      $Host$ can also be the {\tt []}, in which
      case the local host is always used. $Cid$ is unified with the 
      connection id of this new connection.

\item {\tt tb\_connect(+$Cid$)}: Create the actual connection with the \TB,
      or fail when this cannot be done.

\item {\tt tb\_eventloop}: Start the ToolBus eventloop. Note that
      this predicate does not terminate!

\item {\tt tb\_send(+$Cid$, +$Term$)}: Send a term to the ToolBus.

\end{itemize}

\subsection{\label{Ex-hello.pl}The hello example in Prolog: {\tt hello.pl}}

Writing the hello demo in Prolog requires three steps:
\begin{itemize}
\item Write the required Prolog code {\tt hello.pl}. 
      The result is shown in Figure~\ref{fig:hello.pl}.
\item Compile the Prolog program:
\begin{verbatim}
> tbswipl -o hellopl -c hello.pl
hello.pl boot compiled, 0.00 sec, 2,032 bytes.
>
\end{verbatim}
\item Write the required ToolBus script {\tt hello.pl.tb}.
      The result is shown in Figure~\ref{fig:hello.pl.tb}.
\end{itemize}

\begin{figure}
\rule{\textwidth}{0.5mm}
\input{../demos/hello/hello.pl.tex}
  \caption{{\tt hello.pl}: the hello tool in Prolog}
  \label{fig:hello.pl}
\rule{\textwidth}{0.5mm}
\end{figure}

\begin{figure}
\rule{\textwidth}{0.5mm}
\small
\begin{verbatim}
process HELLO is
  let H : hello,               %% H will represent the hello tool
      S : str                  %% S is a string valued variable
  in
      execute(hello, H?) .     %% Execute hello, H gets a tool id as value
      snd-eval(H, get-text(S?)) .  %% Request a text from the hello tool
      rec-value(H, get-text(S?)) . %% Receive it, S gets the text as value
      printf(S)                %% Print it    
  endlet

tool hello is {command = "swipl-adapter -program hellopl -goal toolbus" }

toolbus(HELLO)
\end{verbatim}
\normalsize

  \caption{{\tt hello.pl}: the hello tool in Prolog}
  \label{fig:hello.pl}
\rule{\textwidth}{0.5mm}
\end{figure}
