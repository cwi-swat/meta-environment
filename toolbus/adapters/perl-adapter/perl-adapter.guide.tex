%%
%%    ToolBus -- The ToolBus Application Architecture
%%    Copyright (C) 1998-2000  Stichting Mathematisch Centrum, Amsterdam, 
%%                             The  Netherlands.
%%
%%    This program is free software; you can redistribute it and/or modify
%%    it under the terms of the GNU General Public License as published by
%%    the Free Software Foundation; either version 2 of the License, or
%%    (at your option) any later version.
%%
%%    This program is distributed in the hope that it will be useful,
%%    but WITHOUT ANY WARRANTY; without even the implied warranty of
%%    MERCHANTABILITY or FITNESS FOR A PARTICULAR PURPOSE.  See the
%%    GNU General Public License for more details.
%%
%%    You should have received a copy of the GNU General Public License
%%    along with this program; if not, write to the Free Software
%%    Foundation, Inc., 59 Temple Place, Suite 330, Boston, MA  02111-1307 USA
%%

\section{\label{ToolsInPerl}Writing tools in Perl}

Writing \TB\ tools in Perl is greatly simplified by the {\tt perl-adapter}
to be explained in Section~\ref{perl-adapter}.
Next, a small set of predefined Perl functions is described that are
always loaded by the {\tt perl-adapter} and can be used in any Perl script
(Section~\ref{Perl-functions}).
Finally, we present in Section~\ref{Ex-hello.perl} the Perl version
of the {\tt hello} tool.

\subsection{\label{perl-adapter}{\tt perl-adapter}}

\paragraph{Synopsis.} Execute a Perl script as tool.

\paragraph{Example.} {\tt perl-adapter -script hello.perl}

\paragraph{Specific arguments.}
\begin{itemize}
\item {\tt -script}: The Perl script to be executed.
\end{itemize}

\paragraph{Communication.} \hspace{-0.3cm}\footnote{Communication is described
from the point of view of the \TB, i.e., {\tt snd-} and {\tt rec-}
mean, respectively, send by \TB\ and receive by \TB.}

\begin{itemize}
\item {\tt snd-eval($Tid$, $Fun$($A_1$, ..., $A_n$)}: perform the Perl subroutine call
{\tt do $Fun$($A_1$,  ..., $A_n$)}. Here $Tid$ is tool identifier 
(as produced by {\tt execute} or {\tt rec-connect}) for an instance of the {\tt perl-adapter}.
\item {\tt rec-value($Tid$,$Res$)}: the return value for a previous evaluation request.
\item {\tt rec-event($Tid$, $A_1$, ..., $A_n$)}: event generated by Perl.
\item {\tt snd-ack-event($Tid$, $A_1$)}: acknowledgement of
a previously generated event.
\item {\tt snd-terminate($Tid$, $A_1$)}: terminate execution of perl-adapter.
\end{itemize}

\noindent The command {\tt perl} is executed once, an initial Perl script
is read, and all further requests are directed to this incarnation
of {\tt perl}. A small set of Perl procedures is available for
unpacking and packing \TB\ terms (see below).

\subsection{\label{Perl-functions}Predefined Perl functions}
The following Perl functions are predefined and can be used freely in Perl
script executed via the perl-adapter:
\begin{itemize}
\item {\tt TBstring $Str$\/}: converts a Perl string to a \TB\ string by
surrounding it with double quotes and escaping double quotes occurring
inside $Str$.

\item {\tt PERLstring $Str$\/}: converts a \TB\ string into  a Perl string
by removing surrounding double quotes.

\item {\tt TBerror $Msg$\/}: constructs an error message that can be send back
to the \TB.

\item {\tt TBsend $Trm$\/}: send $Trm$ back to the \TB.

\end{itemize}

\subsection{\label{Ex-hello.perl}The hello example in Perl: {\tt hello.perl}}

Writing the hello tool in Perl requires two steps:
\begin{itemize}
\item Write the required Perl code {\tt hello.perl}. The result is shown in Figure~\ref{fig:hello.perl}.
\item Replace {\tt hello}'s tool definition in {\tt hello2.tb} by:
\begin{verbatim}
      tool hello is {command = "perl-adapter -script hello.perl"}
\end{verbatim}
\end{itemize}


\begin{figure}
\rule{\textwidth}{0.5mm}
\input{hello.perl.tex}
  \caption{{\tt hello.perl}: the hello tool in Perl}
  \label{fig:hello.perl}
\rule{\textwidth}{0.5mm}
\end{figure}
