\documentstyle[11pt,a4]{article}
\setlength{\textwidth}{15cm}
\setlength{\oddsidemargin}{0.46cm}
\setlength{\evensidemargin}{0.46cm}
\begin{document}
\bibliographystyle{alpha}
\title{\bf An algebraic specification of a TRS}
\author{Frank Tip (tip@cwi.nl)}
\date{June 1992}
\maketitle
%
\section{Modules}

The syntax of a TRS is defined in module {\tt TRS-syntax}. Matching and
instantiation of terms are defined in modules {\tt TRS-match} and
{\tt TRS-instantiate}, respectively. In module {\tt TRS-rewrite}, term
rewriting is specified. Module {\tt TRS-check} contains a static checker
for term rewriting systems. Modules {\tt Layout}, {\tt Booleans},
{\tt Integers} and {\tt Identifiers} are borrowed from {\tt Pico} and
contain some basic data types. {\tt Lisp} and {\tt Output} define hybrid
functions for printing intermediate terms and error messages, and for
writing information to a ``log'' file. Lastly, module {\tt Annotations}
defines annotations of symbols of terms.

\section{Syntax}

In module TRS-syntax, the syntax of a TRS is defined. Terms are defined in
an inductive manner. Constants (see module {\tt Identifiers}) and variables
are terms. Moreover, function symbols applied to the appropriate number of
terms construct terms. Associated with each symbol in a term may be a list
of zero or more annotations, defined in module {\tt Annotations}.
Variables always start with \$, in order to distinguish them from
function symbols and constants.

According to the following rule in the specification:
\begin{verbatim}
  "trs" SIG-SECTION RR-SECTION TERM-SECTION       -> TRS
\end{verbatim}
a TRS consists of the keyword {\tt trs}, a signature section (sort
{\tt SIG-SECTION}), a rewrite rule section (sort {\tt RR-SECTION}), and a
term section {\tt sort TERM-SECTION}). The signature section consists of
the keyword {\tt signature} followed by a list of one or more signature
elements (sort {\tt SIG-ELEM}), each consisting of a function symbol
followed by a ``/'' and its number of arguments (e.g. f/2).
The rewrite rule section contains a list of rewrite rules separated by
commas and preceded by the keyword {\tt rewrite-rules}. A rewrite rule
is a {\tt TAG} between square brackets followed by a term, an arrow ({\tt ->})
and another term. Conditional rules are not (yet) supported. Finally, the
term section consists of the keyword {\tt term} followed by a term.
Below, an example TRS is shown (see file bool.trs):
\begin{verbatim}
trs
  signature
    AND/2, OR/2, NOT/1, TRUE/0, FALSE/0
  rewrite-rules
    [B1] AND (TRUE, $X)         -> $X,
    [B2] AND (FALSE, $X)        -> FALSE,
    [B3] OR(TRUE, $X)           -> TRUE,
    [B4] OR(FALSE, $X)          -> $X,
    [B5] NOT(TRUE)              -> FALSE,
    [B6] NOT(FALSE)             -> TRUE
  term
    AND(
      AND(
        NOT(AND(FALSE,TRUE)),
        NOT(TRUE)
      ),
      OR(FALSE,TRUE)
    )
\end{verbatim}

\section{Checking a TRS}

In order to avoid errors, a TRS can be checked by applying the function
{\tt chk} of module {\tt TRS-check} to it. The following errors will be
detected:
\begin{itemize}
  \item
    The signature contains the same element more than once.
  \item
    A left-hand side of a rule is a single variable.
  \item
    A right-hand side of a rule contains variables which do not occur
    in the left-hand side.
  \item
    A rewrite rule contains function symbols which do not appear in
    the signature.
  \item
    The term contains function symbols which do not appear in
    the signature.
\end{itemize}
Hybrid functions \cite{ASDFman92} are used to emit error messages during
the checking of the TRS (see module {\tt Output}).

\section{Term Rewriting}

\subsection{Matching}

\begin{verbatim}
     match "(" TERMS ";" TERMS ";" BINDINGS ")"   -> BINDINGS
\end{verbatim}
The function {\tt match} in module {\tt TRS-match} defines the matching of
closed terms against open terms.
Note that this function takes two {\it lists of} terms as arguments---in
this way, the function can be expresssed in terms of itself. {\tt match}
returns either a list of variable bindings (sort {\tt BINDINGS}), or the
special value {\tt fail} in case the matching fails.

\subsection{Instantiation}

\begin{verbatim}
     inst "(" TERMS ";" BINDINGS ")"              -> TERMS
\end{verbatim}
The function {\tt inst} in module {\tt TRS-instantiate} takes as arguments a
list of open terms, and a list of variable bindings.
It computes the list of instantiated terms. Observe that in the resulting
term, variable bindings are annotated with the annotation {\tt nf}, indicating
that a subterm is in normal form. This assumption can only be made when an
innermost strategy is used for rewriting.

\subsection{Rewriting}

\begin{verbatim}
     rewrite "(" TRS ")"                          -> TERM
\end{verbatim}
The function {\tt rewrite} in module {\tt TRS-rewrite} specifies the
rewriting of the term in the term section according to the rewrite rules
in the rewrite rules section. Two mutually recursive functions, {\tt try}
and {\tt traverse} define a leftmost-innermost rewriting process. State
information is represented {\it explicitly}: a list of subterms is
maintained together with the current path from the root of the term. (A
path consists of a sequence of natural numbers indicating argument positions
in function symbols.) All rewrite rules a stored in a {\it table}; there is
one entry for each signature element which contains the list of rules for that
function.

In equation {\tt Rew}, the table is constructed and we start by traversing
down from the root of the term. Traversal proceeds downwards until either
a constant, or a normal form is reached, then the direction changes to
``up''. When the direction is up, a rewrite step is attempted ({\tt Tr4}).

As mentioned, function {\tt try} tries a rewrite step. In equation {\tt Tr1},
the matching fails and the next rule is considered. In {\tt Tr2} matching
succeeds, the rewrite step is performed, the state is updated, information
is written to the log file, and we traverse downwards in the contractum.
{\tt Tr3} handles the case where no match can be found, and a ``right
brother'' can be traversed. In {\tt Tr4}, the last argument of the function
did not match, and we go up in the term, by traversing the ``parent''.
Finally, in {\tt Tr5} all subterms have been considered and the normal
form is reached.

\subsection{Executing the specification}

Type ``ctasdf'' in a directory containing the specification. Module
{\tt TRS-syntax} will be loaded automatically. Type a TRS yourself, or
load one of the \*.trs files in an editor over module {\tt TRS-syntax}.
\begin{itemize}
  \item bool.trs : evaluation of boolean expressions
  \item calc.trs : simple succ---0 integer arithmetic
  \item tc.trs : a typechecker for a small Pico-like language
  \item ev.trs : an evaluator for a small Pico-like language
\end{itemize}
Selection of the buttons {\tt check TRS} and {\tt rewrite} will result in
loading the appropriate modules, and checking/rewriting, respectively.
Note that, as a side effect, intermediate results are {\it appended} to a file
``asdf.out''.

\section{Remarks}

\begin{itemize}
 \item
   There is no support yet for conditional TRSs. In the ASF+SDF system,
   CTRSs are evaluated as {\it join} systems \cite{Klo91}; negative
   conditions are evaluated by matching the resulting normal forms;
   positive conditions may introduce variables in at most one side. Such
   ``new'' variables are bound by matching.
 \item
   Executing non-terminating TRSs will of course not terminate.
 \item
   Lists (i.e., functions with a variable arity) are not supported.
 \item
   The traverse-try algorithm for innermost rewriting is a bit obscure.
 \item
   The specification defines a term rewriting system (TRS), as described in
   \cite{Klo91}. Implementation issues of TRSs are discussed in \cite{Wal91}.
\end{itemize}

\bibliography{/ufs/gipe/lib/tex/bib}
\end{document}
