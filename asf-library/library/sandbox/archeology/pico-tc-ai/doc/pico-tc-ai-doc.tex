%%%%%%%%%%%%%%%%%%%%%%%%%%%%%%%%%%%%%%%%%%%%%%%%%%%%%%%%%%%%%%%%%%%%%%%%%%%%

\documentstyle[verbatimfiles,11pt,a4]{article}

\input{/ufs/gipe/src/ct/centaur/sources/ASDF/tolatex/ASF+SDF.tex}

\newcommand{\Q}{"}
\newcommand{\spec}[1]{{\tt #1}}
\newcommand{\specparent}[1]{{\rm #1}}
\newcommand{\specvariable}[1]{{\it #1}}
\newcommand{\specnum}[1]{{$#1$}}
\newcommand{\commasym}{,}
\newcommand{\colonsym}{:}
\newcommand{\idcommalistp}{\{<ID> ","\}+}
\newcommand{\Qmul}{"*"}
\newcommand{\Qslash}{"/"}
\newcommand{\idsemilist}{\{ID ";"\}+}
\newcommand{\idsemilistp}{\{ID ";"\}+}
\newcommand{\idcommalists}{\{ID ","\}*}
\newcommand{\expcommalists}{\{EXP ","\}*}
\newcommand{\Lpar}{(}
\newcommand{\Rpar}{)}
\newcommand{\idincommalistp}{\{<ID-in-,-list> ","\}+}
\newcommand{\idincommalist}{<ID-in-,-list>}
\newcommand{\MATH}[1]{$#1$}
\newcommand{\arrowstar}{{\huge $\rightarrow^*$}}
\newcommand{\reductionarrow}{{\huge $\rightarrow$}}
\newcommand{\append}{\^{ }}



\newcommand{\asdf}{\mbox{ASF{\tt +}SDF}}
\newcommand{\clax}{\mbox{CLaX}}

\newenvironment{Split}{\begin{tabular}[t]{@{}l@{}}}{\end{tabular}}


\setlength{\textwidth}{15cm}
\setlength{\oddsidemargin}{0.46cm}
\setlength{\evensidemargin}{0.46cm}
\begin{document}
\bibliographystyle{alpha}
\title{\bf Type-checking revisited: Modular Error-handling}
\author{{\em T.B. Dinesh}, \\
        CWI, \\
        P.O. Box 4079, 1009 AB Amsterdam, The Netherlands\\
        e-mail: {\tt dinesh@cwi.nl}}
\maketitle
%
%\input{abstract}


\begin{abstract}

We study alternative ways in which the declarative
knowledge available in algebraic specifications of type-checkers
for a language, can be utilized in generated programming environments.
Algebraic specification formalisms are often based
on initial algebra semantics, in which it is easy to
introduce the so called {\em non-standard values}.
Using an {\em abstract-interpretation} style in such a
setting yields natural and concise type-checking specifications.
A separate module can then process the output
of such a type-checker to generate human-readable messages.

We use the type-checker specification of a toy language (Pico)
to illustrate the utility of the suggested style. 
We then discuss how this style facilitates {\em origin-tracking},
available in the \asdf\ system, to 
automatically identify the source of errors in a Pico program.

\end{abstract}

\section{An Introduction}

Interactive programming environments can be generated from
formal language specifications. An algebraic specification
formalism can be used for generating such programming
environments since term rewriting can be used to execute
these specifications. Thus, given a specification for a
programming language, many customised tools for this
language can be generated.

In this paper we discuss the specification of static semantics
(type-checking) for a programming language. We concentrate
on discussing a specification style that appears to 
benefit not only modularization of type-checker specification,
but also can be used to generate an {\em error reporting} tool.
The specification style advocates concise and abstract
specifications. A significant advantage of this style of
specification over the ``classical'' specification styles
is that the specification of error propagation need not be knitted 
explicitly as part of the type-checker specification 
and can be handled in a modular manner.

We use Pico, a simple language of while-programs, to illustrate
our specification style. Since Pico is a very small language,
we are able to discuss the specification style and its effects in detail.
A program in Pico consists of declarations followed by statements. 
The syntax and a brief description of Pico is in the appendix.
A straight-forward specification of a Pico type-checker
using \asdf\ can be found in Chapter~9 of \cite{BHK89},
which concentrates on keeping the specification
simple and thus does not specify the handling of error cases. 
Thus, type-checking a program results in {\tt true} if the program
is valid and {\tt false} otherwise.

This specification also concentrates on
avoiding any specification over ``non-standard'' values.
{\em Non-standard} values are introduced when a new function,
that is not total, is defined over an existing sort.
In an initial algebra semantics,
the definition of a non total function introduces new elements
for the ``undefined cases'' (in a partial algebra sense).
These additional elements, are known as ``non-standard values''.
Here we rewrite the specification of a type-checker for Pico
so that the non-standard values serve as error messages.
We call these {\em structured errors}, since the components of
these errors form a structure that could be separately
analyzed by an error-handling module\footnote{
Note that ``partial algebras'' and ``error algebras'' which
attempt to make clean initial algebra semantics, by putting
non-standard values in one equivalence class and allow error
propagation (error algebras) will not be useful for specifying
a type-checker in a natural manner, since knowing that type-check
failed without knowing why and where is not interesting for
a programmer.}.

% \begin{verbatim}
% Arie: This is related to 1) Partial Algebras (Riechel? 87) and
% 2) Error Algebras (Goguen 77). Both attempt to make "clean"
% initial algebra semantics, by putting the non-standard elements
% in one equivalence class (and allow for error propagation in the
% error algebras).
% \end{verbatim}

\section{\label{AlgSpec}Algebraic specifications and term rewriting systems}

%something about algebraic specifications,

Algebraic specifications are typically used as formalisms to
specify abstract data types. An abstract data type (ADT) is an
{\em algebra}, i.e., a collection of {\em carrier sets}
and associated {\em functions}. 
In an algebraic specification formalism, the properties of the data type to
be specified have to be expressed in terms of {\em equations}. 

ASF is an algebraic specification formalism that allows modular
construction of specifications \cite{BHK89}. 
A {\em basic} ASF module consists of
\begin{itemize}
\item 
      A set of {\em sorts} and {\em function} declarations.
      Together these constitute the so-called {\em signature} of
      the module. Sorts are names of the carrier sets and functions
      correspond to functions in the algebra described by the module.
\item 
      A set of {\em Variable declarations}, which together with the
      signature define a language of terms.
\item 
      A set of {\em equations} over the terms defined by the language.
\end{itemize}
An ADT specification describes many algebras, all
of them satisfying the equations it contains -- the {\em models} of an ADT.
The model closest to ordinary programming practice is the so-called
{\em initial} algebra. Characteristic properties of initial algebra 
are 1) no junk (only contains objects expressible using its signature)
and 2) no confusion (all equational properties can be deduced from
its equations).
In ASF, a typical specification is a sequence of modules. A module
can be normalized in the context of the specification in which it 
belongs by eliminating all imports. The semantics of a module is
the initial algebra of its normal form.

%and their implementation as term rewriting systems

An interesting consequence of algebraic specifications is that
they can be executed if they can be interpreted as {\em term
rewriting systems} (TRS). For a specification to be a TRS, it
must obey two rules:
\begin{itemize}
\item 
      No variable may occur on the right-hand side of an equation
      which does not occur on the left-hand side.
\item
      The left-hand side of an equation cannot be a sole variable.
\end{itemize}

%basic terminology: redex, contractum, substitutions, contexts

In such a specification (TRS), an equation $s = t$ is interpreted
as a rewrite rule $s\rightarrow t$, meaning that $s$
can be rewritten to $t$. A term is said to be in its {\em normal form}
if it cannot be further rewritten.
A term containing variables is called {\em open},
otherwise it is called a {\em ground term}. A {\em substitution}
assigns values (terms) to the variables in term. An {\em instantiation}
of an open term is obtained by substituting, for each of its variables,
an assigned value.

If $s = t$ is an equation in ASF, it is interpreted as a reduction
rule $s \rightarrow t$. If $s^\phi$ is an instantiation of $s$ using 
substitution $\phi$, then $s^\phi \rightarrow t^\phi$ is called a 
basic reduction step. In this reduction step $s^\phi$ is called a
{\em redex} (for reducible expression). A term with a hole, i.e.,
an unknown subterm is referred to as a {\em context} and is
denoted as $C[ ]$. A {\em reduction sequence} $t_0 \rightarrow
t_1 \rightarrow \cdots \rightarrow t_n$ means that elementary
reduction steps $t_i \rightarrow t_{i+1}$ are used to obtain $t_n$.
Applying a rewrite rule $r: s \rightarrow t$ in a context $C$
using the substitution $\phi$ is denoted by 
$C[s^\phi] \rightarrow_r C[t^\phi]$.  Also, $t^\phi$ is often
referred to as reduct or contractum.
 
%    something about \asdf 

\subsection{\label{ASDF}\asdf }

  ASF, an acronym for Algebraic Specification Formalism, \cite{BHK89}
  is a formalism supporting modularization and conditional equations.
  SDF, Syntax Definition Formalism, has been developed \cite{HHKR89} to
  support the definition of lexical, context-free and abstract syntax.
  These two formalisms have been combined into one algebraic specification
  formalism called \asdf.

  Specifying in the \asdf\ formalism is supported by the
  \asdf\ {\em system} \cite{Kli93}.
  This system is able to generate parsers from \asdf\ specifications
  and derive term rewriting systems, %for specifications in \asdf,
  thus allowing the execution of \asdf\ specifications.
%  Moreover, it can generate syntax-directed editors for
%  modules in the specification as well as for terms over the signature.
%  The system is able to perform several static and semantic checks on the
%  specifications, and support testing of specifications.


%%\subsection{Lexical Syntax}
%%
%%  SDF allows the definition of lexical syntax, i.e., the definition of 
%%  the elementary ``words'' of the syntax.
%%  There are two important corollaries for the specifier:
%%  \begin{itemize}
%%  \item
%%  The specifier must define layout symbols (white space and
%%  comments recognition) %, for equations and terms,
%%  using the designated sort \ULEX{LAYOUT}.
%%  %First of all the \ULEX{LAYOUT} for the equations must be defined,
%%  %i.e., the specifier himself can (and must) define what symbols constitute
%%  %white space, and how comments can be recognized.
%%  A typical layout definition is:
%%  
%%  \MODULEBEGIN{Layout}
%%  \EXPORTSBEGIN{}
%%  \LEXSYNBEGIN{}
%%  \LEXFUN{\LEX{$[$\ $\backslash$t$\backslash$n$]$}}{\ULEX{LAYOUT}}
%%   \LEXFUN{\QLEX{\%\%} $\sim$\LEX{$[$$\backslash$n$]$}\LEX{$*$}
%%  \QLEX{$\backslash$n}}{\ULEX{LAYOUT}}
%%  \LEXSYNEND{}
%%  \EXPORTSEND{}
%%  \MODULEEND{}
%%
%%  A space, tab, or new line is a layout symbol, as well as everything between
%%  two percent signs and a new line.
%%  If tokens of sort \ULEX{LAYOUT} %{\footnotesize \tt LAYOUT}
%%  are detected in a text, they are ignored.  
%%
%%  \item
%%  Variable declarations are treated as declarations of lexical
%%  syntax. 
%%  This implies that any construct allowed in the lexical syntax definition
%%  is allowed in the variable definition section as well.
%%  Consequently it is possible to define the variables
%%  %{\footnotesize \tt p1}, {\footnotesize \tt p2}, {\footnotesize \tt p3}, 
%%  \SLEX{i_1}, \SLEX{i_2}, \SLEX{i_3}, $\cdots$ all at once:
%%
%%  \begin{tabular}{ll}
%%  \VARDECL{\LEX{i} \LEX{$[$0-9$]*$}}{\ULEX{ELEMENT}}
%%  \end{tabular}
%%
%%  This declares all  words starting with %a {\footnotesize \tt p}
%%  an \SLEX{i} followed
%%  by zero or more characters in the range \LEX{0-9} %{\footnotesize \tt 0-9} 
%%  to be variables of sort \ULEX{ELEMENT}.  %{\footnotesize \tt NAT}.
%%  \end{itemize}

A brief introduction to the \asdf\ formalism is given below,
by means of describing {\em list functions}.

\subsubsection{List functions}

  An important feature of the \asdf\ formalism is the existence
  of list functions and list variables.
  List functions have a variable number of arguments,
  and list variables may range over any number of arguments of a list
  function.

  As an example, suppose we would like to have a function
  [] for the empty set, %{\footnotesize \tt [ ]} for the empty set,
  [ E1 ] for a set with one element, %{\footnotesize \tt [ E1 ]} 
  [ E1, E2 ]  for a set with two elements, %{\footnotesize \tt [ E1, E2 ]} 
  and so on.
  
  The way to define this in \asdf\ is as follows:
  \begin{tabbing}
  m\=m\=m\=m\=m\=m\=m\=m\=\kill
  \+\+\SORTS{\ULEX{ELEMENT} \ULEX{SET}}
   \CFGBEGIN{}
  \CFGFUN{\QLEX{$[$} \{\ULEX{ELEMENT} \QLEX{$,$}\}\LEX{$*$}
  \QLEX{$]$}}{\ULEX{SET}
  }{}
  \CFGEND{}
  \end{tabbing}

  The asterisk * says that we want %{\footnotesize \tt *} says that we want
  zero or more \ULEX{ELEMENT}s, %{\footnotesize \tt ELEMENT}s,
  while the comma says that these should be separated by commas.
  Thus, a set consists of \ULEX{ELEMENT}s, separated by commas
  and the set itself is delimited by [ and ]. 
  %{\footnotesize \tt [} and {\footnotesize \tt ]}.

  This list notation is simply an abbreviation for the declaration of
  infinitely many functions [ $\cdots$ ], %{\footnotesize \tt [ ... ]}, 
  each with a different number of arguments.
  Likewise, the same (concrete) syntax could have been
  obtained without using lists by the following:
  %``normal'' BNF grammar rules:

\begin{tabbing}
m\=m\=m\=m\=m\=m\=m\=m\=\kill
%\EXPORTSBEGIN{}
\+\+\SORTS{\ULEX{ELEMENT} \ULEX{ELEMENTS} \ULEX{SET}}
 \CFGBEGIN{}
\CFGFUN{\QLEX{$[$} \ULEX{ELEMENTS} \QLEX{$]$}}{\ULEX{SET}}{}
 \CFGFUN{}{\ULEX{ELEMENTS}}{}
 \CFGFUN{\ULEX{ELEMENT}}{\ULEX{ELEMENTS}}{}
 \CFGFUN{\ULEX{ELEMENTS} \QLEX{$,$}
\ULEX{ELEMENTS}}{\ULEX{ELEMENTS}}{\{\KW{\LEX{left}}\}}
\CFGEND{}
%\EXPORTSEND{}
\end{tabbing}

%  {\footnotesize \begin{verbatim}
%context-free syntax
%   "[" ELEMENTS "]"      -> SET
%   "[" "]"               -> SET  
%   ELEMENT               -> ELEMENTS
%   ELEMENTS "," ELEMENT  -> ELEMENTS   \end{verbatim}}

%\noindent Attributes, between \{\} 
%      brackets, may be associated with functions, stating for instance
%      that the function is {\bf left} or {\bf right} associative.

  In order to define equations over list functions, we need 
  list variables:
\begin{tabbing}
m\=m\=m\=m\=m\=m\=m\=m\=\kill
\+\+\VARIABLESBEGIN{}
\VARDECL{\LEX{Elts} \LEX{$[$123$]$}}{\{\ULEX{ELEMENT} \QLEX{$,$}\}\LEX{$*$}}
 \VARDECL{\LEX{i}}{\ULEX{ELEMENT}}
\VARIABLESEND{}
\EQUATIONSBEGIN{}
\EQU{\LEX{$[$}\LEX{eq1}
\LEX{$]$}}{\LEX{$[$}\IVAR{\LEX{Elts}}{1},\VAR{\LEX{i}},
\IVAR{\LEX{Elts}}{2},\VAR{\LEX{i}},\IVAR{\LEX{Elts}}{3}\LEX{$]$}}
{\LEX{$[$}
\IVAR{\LEX{Elts}}{1},\VAR{\LEX{i}},\IVAR{\LEX{Elts}}{2},\IVAR{\LEX{Elts}  
  }{3}\LEX{$]$} 
}
\EQUATIONSEND{}
\end{tabbing}

  \SLEX{Elts_1}, \SLEX{Elts_2}, and \SLEX{Elts_3} are  list variables,
  ranging over list of zero or more \ULEX{ELEMENT}s 
  \underline{separated} by commas.

  Here we have specified in one single equation that
  elements of sets do not have multiplicity:
  any set containing element \SLEX{i} %{\footnotesize \tt i} 
  at least two times is 
  equal to  the set containing one occurrence less of \SLEX{i}.
  %{\footnotesize \tt i}.
  However, the alternative way of defining the \ULEX{SET} syntax using
  explicit ``,'' operator for \ULEX{ELEMENTS} sort would require
  many more equations to capture this effect.
  %associativity of \ULEX{SET} elements.




\section{\label{PicoTcOld}Style for identifying errors}

Specifying a type-checker for Pico was done using a table
for type-environments (Sort TENV, see Appendix~\ref{PICOSYN}) 
for keeping track of types associated with
identifiers through declarations. The type-checking was
specified as shown below. The function {\tt tc} returns
a BOOL result. Checking the statement series using the
type-environment results in BOOL, and checking if two types
are same results in BOOL. Extracting the type of an expression
using the type-environment results in TYPE (either {\tt natural}
or {\tt string}).

\MODULEBEGIN{Pico-typecheck-old}
\IMPORTS{\LEX{Type-environments}}
\EXPORTSBEGIN{}
\CFGBEGIN{}
\CFGFUN{\LEX{tc} \ULEX{PROGRAM}}{\ULEX{BOOL}}{}
 \CFGFUN{\QLEX{$[$} \ULEX{DECLS} \QLEX{$]$}}{\ULEX{TENV}}{}
 \CFGFUN{\ULEX{TENV} \ULEX{SERIES}}{\ULEX{BOOL}}{}
 \CFGFUN{\ULEX{TENV} \QLEX{$.$} \ULEX{EXP}}{\ULEX{TYPE}}{}
 \CFGFUN{\LEX{compatible}(\ULEX{TYPE}, \ULEX{TYPE})}{\ULEX{BOOL}}{}
\CFGEND{}
\EXPORTSEND{}
\EQUATIONSBEGIN{}
\CEQU{\LEX{$[$}\LEX{Tc1a}\LEX{$]$}}%
{\COND{\LEX{tc}~\penalty+5\LEX{begin}~\penalty+10\VAR{\LEX{D}}{}~\penalty+10\VAR{\LEX{S}}{}~\penalty+10\LEX{end}}{\CON{true}}}{{\COND{\LEX{$[$}\VAR{\LEX{D}}{}\LEX{$]$}~\penalty+5\VAR{\LEX{S}}{}}{\CON{true}}}}%
\CEQU{\LEX{$[$}\LEX{Tc1b}\LEX{$]$}}%
{\COND{\LEX{tc}~\penalty+5\LEX{begin}~\penalty+10\VAR{\LEX{D}}{}~\penalty+10\VAR{\LEX{S}}{}~\penalty+10\LEX{end}}{\CON{false}}}{{\NCOND{\LEX{$[$}\VAR{\LEX{D}}{}\LEX{$]$}~\penalty+5\VAR{\LEX{S}}{}}{\CON{true}}}}%
\CEQU{\LEX{$[$}\LEX{Tc2}\LEX{$]$}}%
{\COND{\LEX{$[$}\LEX{declare}~\penalty+10\VAR{\LEX{Id}}{}~\penalty+20\LEX{$:$}~\penalty+20\VAR{\LEX{Type}}{},\VAR{\LEX{Id-type-list}}{}\LEX{$;$}\LEX{$]$}}{\LEX{$[$}\VAR{\LEX{Id}}{}~\penalty+15\LEX{$:$}~\penalty+15\VAR{\LEX{Type}}{},\VAR{\LEX{Pair-list}}{}\LEX{$]$}}}{{\COND{\LEX{$[$}\LEX{declare}~\penalty+10\VAR{\LEX{Id-type-list}}{}\LEX{$;$}\LEX{$]$}}{\LEX{$[$}\VAR{\LEX{Pair-list}}{}\LEX{$]$}}}}%
\EQU{\LEX{$[$}\LEX{Tc3}\LEX{$]$}}{\LEX{$[$}\LEX{declare}~\penalty+10\LEX{$;$}\LEX{$]$}}{\LEX{$[$}\LEX{$]$}}
\\
\CEQU{\LEX{$[$}\LEX{Tc4}\LEX{$]$}}%
{\COND{\VAR{\LEX{Tenv}}{}~\penalty+5\VAR{\LEX{Stat}}{};\VAR{\LEX{Stat-list}}{}}{\CON{true}}}{{\COND{\VAR{\LEX{Tenv}}{}~\penalty+5\VAR{\LEX{Stat}}{}}{\CON{true}}}, 
{\COND{\VAR{\LEX{Tenv}}{}~\penalty+5\VAR{\LEX{Stat-list}}{}}{\CON{true}}}}%
\EQU{\LEX{$[$}\LEX{Tc5}\LEX{$]$}}{\VAR{\LEX{Tenv}}{}~\penalty+5}{\CON{true}}
\\
\CEQU{\LEX{$[$}\LEX{Tc6}\LEX{$]$}}%
{\COND{\VAR{\LEX{Tenv}}{}~\penalty+5\VAR{\LEX{Id}}{}~\penalty+15\LEX{$:=$}~\penalty+15\VAR{\LEX{Exp}}{}}{\CON{true}}}{{\COND{\LEX{compatible}\LEX{$($}\VAR{\LEX{Tenv}}{}~\penalty+10\LEX{$.$}~\penalty+10\VAR{\LEX{Id}}{}\LEX{$,$}~\penalty+5\VAR{\LEX{Tenv}}{}~\penalty+10\LEX{$.$}~\penalty+10\VAR{\LEX{Exp}}{}\LEX{$)$}}{\CON{true}}}}%
\CEQU{\LEX{$[$}\LEX{Tc7}\LEX{$]$}}%
{\COND{\VAR{\LEX{Tenv}}{}~\penalty+5\LEX{if}~\penalty+15\VAR{\LEX{Exp}}{}~\penalty+15\LEX{then}~\penalty+15\IVAR{\LEX{S}}{1}{}~\penalty+15\LEX{else}~\penalty+15\IVAR{\LEX{S}}{2}{}~\penalty+15\LEX{fi}}{\CON{true}}}{{\COND{\VAR{\LEX{Tenv}}{}~\penalty+5\LEX{$.$}~\penalty+5\VAR{\LEX{Exp}}{}}{\CON{natural}}}, 
{\COND{\VAR{\LEX{Tenv}}{}~\penalty+5\IVAR{\LEX{S}}{1}{}}{\CON{true}}}, 
{\COND{\VAR{\LEX{Tenv}}{}~\penalty+5\IVAR{\LEX{S}}{2}{}}{\CON{true}}}}%
\CEQU{\LEX{$[$}\LEX{Tc8}\LEX{$]$}}%
{\COND{\VAR{\LEX{Tenv}}{}~\penalty+5\LEX{while}~\penalty+15\VAR{\LEX{Exp}}{}~\penalty+15\LEX{do}~\penalty+15\VAR{\LEX{S}}{}~\penalty+15\LEX{od}}{\CON{true}}}{{\COND{\VAR{\LEX{Tenv}}{}~\penalty+5\LEX{$.$}~\penalty+5\VAR{\LEX{Exp}}{}}{\CON{natural}}}, 
{\COND{\VAR{\LEX{Tenv}}{}~\penalty+5\VAR{\LEX{S}}{}}{\CON{true}}}}%
\CEQU{\LEX{$[$}\LEX{Tc9}\LEX{$]$}}%
{\COND{\VAR{\LEX{Tenv}}{}~\penalty+5\LEX{$.$}~\penalty+5\IVAR{\LEX{Exp}}{1}{}~\penalty+10\LEX{$+$}~\penalty+10\IVAR{\LEX{Exp}}{2}{}}{\CON{natural}}}{{\COND{\VAR{\LEX{Tenv}}{}~\penalty+5\LEX{$.$}~\penalty+5\IVAR{\LEX{Exp}}{1}{}}{\CON{natural}}}, 
{\COND{\VAR{\LEX{Tenv}}{}~\penalty+5\LEX{$.$}~\penalty+5\IVAR{\LEX{Exp}}{2}{}}{\CON{natural}}}}%
\CEQU{\LEX{$[$}\LEX{Tc10}\LEX{$]$}}%
{\COND{\VAR{\LEX{Tenv}}{}~\penalty+5\LEX{$.$}~\penalty+5\IVAR{\LEX{Exp}}{1}{}~\penalty+10\LEX{$-$}~\penalty+10\IVAR{\LEX{Exp}}{2}{}}{\CON{natural}}}{{\COND{\VAR{\LEX{Tenv}}{}~\penalty+5\LEX{$.$}~\penalty+5\IVAR{\LEX{Exp}}{1}{}}{\CON{natural}}}, 
{\COND{\VAR{\LEX{Tenv}}{}~\penalty+5\LEX{$.$}~\penalty+5\IVAR{\LEX{Exp}}{2}{}}{\CON{natural}}}}%
\CEQU{\LEX{$[$}\LEX{Tc11}\LEX{$]$}}%
{\COND{\VAR{\LEX{Tenv}}{}~\penalty+5\LEX{$.$}~\penalty+5\IVAR{\LEX{Exp}}{1}{}~\penalty+10\LEX{$||$}~\penalty+10\IVAR{\LEX{Exp}}{2}{}}{\CON{string}}}{{\COND{\VAR{\LEX{Tenv}}{}~\penalty+5\LEX{$.$}~\penalty+5\IVAR{\LEX{Exp}}{1}{}}{\CON{string}}}, 
{\COND{\VAR{\LEX{Tenv}}{}~\penalty+5\LEX{$.$}~\penalty+5\IVAR{\LEX{Exp}}{2}{}}{\CON{string}}}}%
\EQU{\LEX{$[$}\LEX{Tc12}\LEX{$]$}}{\VAR{\LEX{Tenv}}{}~\penalty+5\LEX{$.$}~\penalty+5\VAR{\LEX{Id}}{}}{\LEX{lookup}~\penalty+5\VAR{\LEX{Id}}{}~\penalty+5\LEX{in}~\penalty+5\VAR{\LEX{Tenv}}{}}
\EQU{\LEX{$[$}\LEX{Tc13}\LEX{$]$}}{\VAR{\LEX{Tenv}}{}~\penalty+5\LEX{$.$}~\penalty+5\VAR{\LEX{Nat-con}}{}}{\CON{natural}}
\EQU{\LEX{$[$}\LEX{Tc14}\LEX{$]$}}{\VAR{\LEX{Tenv}}{}~\penalty+5\LEX{$.$}~\penalty+5\VAR{\LEX{Str-con}}{}}{\CON{string}}
\\
\EQU{\LEX{$[$}\LEX{Tca}\LEX{$]$}}{\LEX{compatible}\LEX{$($}\CON{natural}\LEX{$,$}~\penalty+5\CON{natural}\LEX{$)$}}{\CON{true}}
\EQU{\LEX{$[$}\LEX{Tcb}\LEX{$]$}}{\LEX{compatible}\LEX{$($}\CON{string}\LEX{$,$}~\penalty+5\CON{string}\LEX{$)$}}{\CON{true}}
\EQUATIONSEND{}
\MODULEEND{}


Let us consider the equations [Tc1a] and [Tc1b] in the specification
above. The equation [Tc1a] specifies that type-checking
a program is ``true'' if type-checking the statements using
the type-environment obtained from the declarations results in ``true''.
However, equation [Tc1b] specifies that if type-checking statements 
is {\em not} ``true'', then the result is ``false''.
Equations such as [Tc1b] are solely used in initial algebra
specifications to avoid the so-called non-standard values.
In this case, the goal was that the result of type-checking
be the standard values of sort BOOL (``true'' or ``false'').

The problem with this specification is that generating
error messages (which was ignored in the above specification)
requires modification of the specification to handle the
alternate cases and to keep track of errors (and error propagation).

In a specification of a type-checker, it is desirable to specify 
only the cases for which the program is considered valid and 
somehow from this the error cases be identified.
One approach is to use partial algebra semantics, which
gives ``run-time'' errors, say by identifying the non-standard
values as ``undefined''. This approach however is not what is
desired for type-checking a program, since like in the 
above case a program either checks as ``true'' or else
``false'' or ``undefined''. 
% \begin{verbatim}
% Arie: Interesting -- the reason why one would not want to put all
% non-standard elements in one equivalence class.
% \end{verbatim}
These three result values do not provide the desired information.
While type-checking a program,
we want to know {\em what} went wrong (i.e., why it did
not type-check as ``true'') and {\em where} it went wrong.
Let us first concentrate on extracting the ``why'' information
and then discuss how the ``where'' information can be extracted. 

Our first observation is that, if one avoids using conditions
in the equations then the normal form will not be either ``true'' or 
nothing (nothing: term is already in normal form),
thus the effect of [Tc1b] disappears. 
Avoiding conditions of [Tc1a] and [Tc1b] merges these
equations to [Tc1]:

\begin{tabbing}
mmm\=\kill\pushtabs\\*[-8.5pt]
mmmm\= \=mmmmmmmmmmmmmmmm\= \=mmmmmmmmmmmmmmmm\= \=\kill
\EQU{\LEX{$[$}\LEX{Tc1}\LEX{$]$}}{\LEX{tc}~\penalty+5\LEX{begin}~\penalty+10\VAR
{\LEX{D}}{}~\penalty+10\VAR{\LEX{S}}{}~\penalty+10\LEX{end}}{\LEX{$[$}\VAR{\LEX{
D}}{}\LEX{$]$}~\penalty+5\VAR{\LEX{S}}{}}
\poptabs
\end{tabbing}

Thus avoiding conditions results
in the reduction of the term to something ``smaller'', although it
may not be ``true''. Observing equation [Tc4] above suggests
that the condition could be avoided if Booleans provided for
an operation where ``true'' is the identity.
Thus we could use the conjunction operation ``\&'' and rewrite
the equation:

\begin{tabbing}
mmm\=\kill\pushtabs\\*[-8.5pt]
mmmm\= \=mmmmmmmmmmmmmmmm\= \=mmmmmmmmmmmmmmmm\= \=\kill
\EQU{\LEX{$[$}\LEX{Tc4}\LEX{$]$}}{\VAR{\LEX{Tenv}}{}~\penalty+5\VAR{\LEX{Stat}}{
};\VAR{\LEX{Stat-list}}{}}{\VAR{\LEX{Tenv}}{}~\penalty+10\VAR{\LEX{Stat}
}{}~\penalty+5\LEX{\&}~\penalty+5\VAR{\LEX{Tenv}}{}~\penalty+10\VAR{\LEX{Stat-list}}{}}
\poptabs
\end{tabbing}

The next step is to eliminate conditions in equations [Tc6] --- [Tc8]. 
This requires distribution of the type-environment over
the components of a statement, which results in transforming
the statements to an abstract representation (see equations
[Tc18] --- [Tc21] below).
It is then simple to identify the correct abstract statements.
We could thus {\em inject}\footnote{The \asdf\ formalism
  provides the facility to {\em inject} one sort into another.
  This can be simulated in other formalisms by introducing
  explicit {\em injection functions}.
} statements as non-standard values
of Booleans, which means that the type in-correct statements will be 
non-standard values of sort Booleans.
% \begin{verbatim}
% Arie: Related to order-sorted algebra.
% \end{verbatim}

The need for the compatible predicate (discussion in \cite{BHK89})
in the specification above is basically to avoid accidentally equating
the ``undeclared variable'' cases as type correct.
However, this accident is naturally avoided here since the
abstract representations are explicitly identified are correct.

Note that we are distributing type-environments
over the components of expressions, the result of which
should be sort TYPE.  Thus we need to inject TYPEs into EXPs. 
Also, we can keep the specification simple by generalizing
the assignment statement to EXP := EXP.
This extension to syntax is only available in type-check modules 
and does not affect the syntax of the language Pico.

The following specification evaluates the type correct
statements to ``true'', evaluates the expressions
over their abstract values and avoids the use of 
conditions.
When the tc function is applied to a program,
it either evaluates to true (indicating that the program
is checked as type correct) or it reduces to a normal
form which is a conjunction of type incorrect statements
in their abstract form (all expressions in them are
also reduced to their normal form).

\MODULEBEGIN{Pico-typecheck-new}
\IMPORTS{\LEX{Type-environments}}
\EXPORTSBEGIN{}
\CFGBEGIN{}
\CFGFUN{\LEX{tc} \ULEX{PROGRAM}}{\ULEX{BOOL}}{}
 \CFGFUN{\QLEX{$[$} \ULEX{DECLS} \QLEX{$]$}}{\ULEX{TENV}}{}
 \CFGFUN{\ULEX{TENV} \ULEX{SERIES}}{\ULEX{BOOL}}{}
 \CFGFUN{\ULEX{TENV} \QLEX{$.$} \ULEX{EXP}}{\ULEX{TYPE}}{}
 \\
\CFGFUN{\ULEX{EXP} \QLEX{$:=$} \ULEX{EXP}}{\ULEX{STATEMENT}}{}
 \CFGFUN{\ULEX{STATEMENT}}{\ULEX{BOOL}}{}
 \CFGFUN{\ULEX{TYPE}}{\ULEX{EXP}}{}
\CFGEND{}
\EXPORTSEND{}
\EQUATIONSBEGIN{}
\EQU{\LEX{$[$}\LEX{Tc1}\LEX{$]$}}{\LEX{tc}~\penalty+5\LEX{begin}~\penalty+10\VAR{\LEX{D}}{}~\penalty+10\VAR{\LEX{S}}{}~\penalty+10\LEX{end}}{\LEX{$[$}\VAR{\LEX{D}}{}\LEX{$]$}~\penalty+5\VAR{\LEX{S}}{}}
\\
\CEQU{\LEX{$[$}\LEX{Tc2}\LEX{$]$}}%
{\COND{\LEX{$[$}\LEX{declare}~\penalty+10\VAR{\LEX{Id}}{}~\penalty+20\LEX{$:$}~\penalty+20\VAR{\LEX{Type}}{},\VAR{\LEX{Id-type-list}}{}\LEX{$;$}\LEX{$]$}}{\LEX{$[$}\VAR{\LEX{Id}}{}~\penalty+15\LEX{$:$}~\penalty+15\VAR{\LEX{Type}}{},\VAR{\LEX{Pair-list}}{}\LEX{$]$}}}{{\COND{\LEX{$[$}\LEX{declare}~\penalty+10\VAR{\LEX{Id-type-list}}{}\LEX{$;$}\LEX{$]$}}{\LEX{$[$}\VAR{\LEX{Pair-list}}{}\LEX{$]$}}}}%
\EQU{\LEX{$[$}\LEX{Tc3}\LEX{$]$}}{\LEX{$[$}\LEX{declare}~\penalty+10\LEX{$;$}\LEX{$]$}}{\LEX{$[$}\LEX{$]$}}
\\
\EQU{\LEX{$[$}\LEX{Tc4}\LEX{$]$}}{\VAR{\LEX{Tenv}}{}~\penalty+5\VAR{\LEX{Stat}}{};\VAR{\LEX{Stat-list}}{}}{\VAR{\LEX{Tenv}}{}~\penalty+10\VAR{\LEX{Stat}}{}~\penalty+5\LEX{\&}~\penalty+5\VAR{\LEX{Tenv}}{}~\penalty+10\VAR{\LEX{Stat-list}}{}}
\\
\EQU{\LEX{$[$}\LEX{Tc5}\LEX{$]$}}{\VAR{\LEX{Tenv}}{}~\penalty+5}{\CON{true}}
\\
\EQU{\LEX{$[$}\LEX{Tc6}\LEX{$]$}}{\VAR{\LEX{Tenv}}{}~\penalty+5\VAR{\LEX{Exp}}{}~\penalty+15\LEX{$:=$}~\penalty+15\VAR{\LEX{Exp}}{$\,'$}}{\VAR{\LEX{Tenv}}{}~\penalty+10\LEX{$.$}~\penalty+10\VAR{\LEX{Exp}}{}~\penalty+5\LEX{$:=$}~\penalty+5\VAR{\LEX{Tenv}}{}~\penalty+10\LEX{$.$}~\penalty+10\VAR{\LEX{Exp}}{$\,'$}}
\\
\BIGEQU{\LEX{$[$}\LEX{Tc7}\LEX{$]$}}{\VAR{\LEX{Tenv}}{}~\penalty+5\LEX{if}~\penalty+15\VAR{\LEX{Exp}}{}~\penalty+15\LEX{then}~\penalty+15\IVAR{\LEX{S}}{1}{}~\penalty+15\LEX{else}~\penalty+15\IVAR{\LEX{S}}{2}{}~\penalty+15\LEX{fi}}{\LEX{if}~\penalty+15\VAR{\LEX{Tenv}}{}~\penalty+20\LEX{$.$}~\penalty+20\VAR{\LEX{Exp}}{}~\penalty+15\LEX{then}~\penalty+15~\penalty+15\LEX{else}~\penalty+15~\penalty+15\LEX{fi}~\penalty+10\LEX{\&}~\penalty+10\VAR{\LEX{Tenv}}{}~\penalty+15\IVAR{\LEX{S}}{1}{}~\penalty+5\LEX{\&}~\penalty+5\VAR{\LEX{Tenv}}{}~\penalty+10\IVAR{\LEX{S}}{2}{}}
\\
\EQU{\LEX{$[$}\LEX{Tc8}\LEX{$]$}}{\VAR{\LEX{Tenv}}{}~\penalty+5\LEX{while}~\penalty+15\VAR{\LEX{Exp}}{}~\penalty+15\LEX{do}~\penalty+15\VAR{\LEX{S}}{}~\penalty+15\LEX{od}}{\LEX{while}~\penalty+10\VAR{\LEX{Tenv}}{}~\penalty+15\LEX{$.$}~\penalty+15\VAR{\LEX{Exp}}{}~\penalty+10\LEX{do}~\penalty+10~\penalty+10\LEX{od}~\penalty+5\LEX{\&}~\penalty+5\VAR{\LEX{Tenv}}{}~\penalty+10\VAR{\LEX{S}}{}}
\\
\EQU{\LEX{$[$}\LEX{Tc9a}\LEX{$]$}}{\VAR{\LEX{Tenv}}{}~\penalty+5\LEX{$.$}~\penalty+5\VAR{\LEX{Id}}{}}{\LEX{lookup}~\penalty+5\VAR{\LEX{Id}}{}~\penalty+5\LEX{in}~\penalty+5\VAR{\LEX{Tenv}}{}}
\EQU{\LEX{$[$}\LEX{Tc9b}\LEX{$]$}}{\VAR{\LEX{Tenv}}{}~\penalty+5\LEX{$.$}~\penalty+5\VAR{\LEX{Type}}{}}{\VAR{\LEX{Type}}{}}
\\
\EQU{\LEX{$[$}\LEX{Tc10}\LEX{$]$}}{\VAR{\LEX{Nat-con}}{}}{\CON{natural}}
\EQU{\LEX{$[$}\LEX{Tc11}\LEX{$]$}}{\VAR{\LEX{Str-con}}{}}{\CON{string}}
\\
\EQU{\LEX{$[$}\LEX{Tc12}\LEX{$]$}}{\VAR{\LEX{Tenv}}{}~\penalty+5\LEX{$.$}~\penalty+5\IVAR{\LEX{Exp}}{1}{}~\penalty+10\LEX{$+$}~\penalty+10\IVAR{\LEX{Exp}}{2}{}}{\VAR{\LEX{Tenv}}{}~\penalty+10\LEX{$.$}~\penalty+10\IVAR{\LEX{Exp}}{1}{}~\penalty+5\LEX{$+$}~\penalty+5\VAR{\LEX{Tenv}}{}~\penalty+10\LEX{$.$}~\penalty+10\IVAR{\LEX{Exp}}{2}{}}
\\
\EQU{\LEX{$[$}\LEX{Tc13}\LEX{$]$}}{\VAR{\LEX{Tenv}}{}~\penalty+5\LEX{$.$}~\penalty+5\IVAR{\LEX{Exp}}{1}{}~\penalty+10\LEX{$-$}~\penalty+10\IVAR{\LEX{Exp}}{2}{}}{\VAR{\LEX{Tenv}}{}~\penalty+10\LEX{$.$}~\penalty+10\IVAR{\LEX{Exp}}{1}{}~\penalty+5\LEX{$-$}~\penalty+5\VAR{\LEX{Tenv}}{}~\penalty+10\LEX{$.$}~\penalty+10\IVAR{\LEX{Exp}}{2}{}}
\\
\EQU{\LEX{$[$}\LEX{Tc14}\LEX{$]$}}{\VAR{\LEX{Tenv}}{}~\penalty+5\LEX{$.$}~\penalty+5\IVAR{\LEX{Exp}}{1}{}~\penalty+10\LEX{$||$}~\penalty+10\IVAR{\LEX{Exp}}{2}{}}{\VAR{\LEX{Tenv}}{}~\penalty+10\LEX{$.$}~\penalty+10\IVAR{\LEX{Exp}}{1}{}~\penalty+5\LEX{$||$}~\penalty+5\VAR{\LEX{Tenv}}{}~\penalty+10\LEX{$.$}~\penalty+10\IVAR{\LEX{Exp}}{2}{}}
\\
\EQU{\LEX{$[$}\LEX{Tc15}\LEX{$]$}}{\CON{natural}~\penalty+5\LEX{$+$}~\penalty+5\CON{natural}}{\CON{natural}}
\EQU{\LEX{$[$}\LEX{Tc16}\LEX{$]$}}{\CON{natural}~\penalty+5\LEX{$-$}~\penalty+5\CON{natural}}{\CON{natural}}
\EQU{\LEX{$[$}\LEX{Tc17}\LEX{$]$}}{\CON{string}~\penalty+5\LEX{$||$}~\penalty+5\CON{string}}{\CON{string}}
\\
\EQU{\LEX{$[$}\LEX{Tc18}\LEX{$]$}}{\CON{natural}~\penalty+5\LEX{$:=$}~\penalty+5\CON{natural}}{\CON{true}}
\EQU{\LEX{$[$}\LEX{Tc19}\LEX{$]$}}{\CON{string}~\penalty+5\LEX{$:=$}~\penalty+5\CON{string}}{\CON{true}}
\EQU{\LEX{$[$}\LEX{Tc20}\LEX{$]$}}{\LEX{if}~\penalty+5\CON{natural}~\penalty+5\LEX{then}~\penalty+5~\penalty+5\LEX{else}~\penalty+5~\penalty+5\LEX{fi}}{\CON{true}}
\EQU{\LEX{$[$}\LEX{Tc21}\LEX{$]$}}{\LEX{while}~\penalty+5\CON{natural}~\penalty+5\LEX{do}~\penalty+5~\penalty+5\LEX{od}}{\CON{true}}
\\
\CEQU{\LEX{$[$}\LEX{Ext}\LEX{$]$}}%
{\COND{\VAR{\LEX{Id}}{}~\penalty+5\LEX{$:=$}~\penalty+5\VAR{\LEX{Exp}}{}}{\VAR{\LEX{Exp}}{$\,'$}~\penalty+5\LEX{$:=$}~\penalty+5\VAR{\LEX{Exp}}{}}}{{\COND{\VAR{\LEX{Exp}}{$\,'$}}{\VAR{\LEX{Id}}{}}}}%
\EQUATIONSEND{}
\MODULEEND{}


The basic techniques that are identifiable in the above specification,
in order to conform to our suggested style, are to:
\begin{itemize}
\item Avoid conditions in the equations.
\item Distribute the type-environment over statements (equations [Tc4]--[Tc8]).
\item Distribute the type-environment over expressions ([Tc12]--[Tc14]).
\item Evaluate the expressions at abstract (type) level ([Tc15]--[Tc17]).
\item Identify the abstract type-correct statements ([Tc18]--[Tc21]).
\item Transform to extended syntax when needed (equation [Ext]).
%Arie: [Ext] is rather cryptic.
\item Change the constants to their abstract representation ([Tc10] and [Tc11]).
\end{itemize}

The equations [Tc15]--[Tc21]
%, along with equations [Tc10] and [Tc11]
are the crux of this style of specification. The other equations
help transform the source program into its abstract form. The
equations [Tc15]--[Tc21] then identify the type-correct
constructs, while anything that is not reducible by these
equations become {\em structured error} messages.

Thus for the program in the appendix, the result of applying
the tc function is:

\begin{verbatim}
natural := natural - string
\end{verbatim}

This structured error message indicates the following:\\
(1) the program has a type error (did not evaluate to true)
(2) the error is in an assignment statement
(3) the type of the left side of the assignment is incompatible
    with the right side
and (4) the right argument for a subtraction operation has string operand.

This error information can easily be further processed by
a separate module which takes the normal form as input
and issues human readable error messages. For the structured error
above, all or some of the 4 reasons can be used to generate 
human readable error messages.
Generation of such messages in the classical style of
specification (module Pico-typecheck-old) would involve
considerable modification to the equations and is non-trivial
to specify type-checking so that errors can be specified
as a nice modular extension.

The obvious lack of information in the above message, 
however, is an indication for where in the source
program the errors are located. 
It is mandatory to have information on location of errors in a large program.
Generation of this information
is often done by keeping track of line numbers in the input program.
We claim that this information can be automatically
generated by a programming environment generator and
discuss how this is done currently in the \asdf\ system
in Section~\ref{IERR}.

In \cite{DT92}, there is an illustration as to  how this
style of specification also allows type-checking
over effectively incomplete programs. Incomplete
programs can be written in the \asdf\ system
using meta-variables in the input term.

In the next section we illustrate an example of 
using such a technique in a non-toy Pascal
like language called \clax, with a snapshot 
while using origin tracking in the \asdf\ system
to locate a type-error.

%\input{origins}

\section{\label{ORIGIN}Origin Tracking}

Origin tracking is a generic technique for relating parts
of intermediate terms, which occur during term rewriting, to parts
of the initial term. For a detailed description of origin tracking
and its applications, the reader is referred to \cite{DKT92}.
In the ASF+SDF system \cite{Kli93}, algebraic specifications
are executed as term rewriting systems. Consequently, origin tracking 
can be applied to the specification defined in this document. 
In this section, we illustrate how origin tracking is used in
determining the source positions of errors found by the type-checker.

In Section~\ref{PicoTcOld}, we suggested a alternate way
of specifying the type-checker for Pico.
The picture below demonstrates the effectiveness of this style 
for a non-toy, Pascal relative, language \clax.

\vspace{0.5 cm}
\input{psfig}
\centerline{\psfig{figure=snapshot.ps,width=14cm}}
%\newpage

The function ``errors \_'' --- which uses the result of ``tc \_'',
is applied to the program in the window. 
The result of which could be either ``no-errors'' or a list of errors.
These error messages, albeit useful, provide no information
regarding the specific constructs of the program that
caused it or the position where it originated.

Maintaining the relation between the program
construct that caused the error and the error message, could
lead to unwieldy specifications and demand a lot of additional
work on specification/implementation of the type-checker.
Origin tracking automatically maintains certain forms of
relations between the source and result that we can
exploit here. 
Origin tracking in the system provides one with the ability 
to identify the culprit program
constructs by clicking on the desired error and requesting
the system to show its origin. 


\section{\label{IERR}On identifying error location}

% ***Discussion of Nat-con = natural ****
The new type-checker specification discussed in
Section~\ref{PicoTcOld} also has information on error locations, which can
be harnessed by a helpful system as shown in Section~\ref{ORIGIN}. 

To simply explain the idea of automating the process of
error location identification, we will consider a simple
origin tracking mechanism and modify our 
specification so that enough origins can be tracked
to determine complete information on where the errors
are located.

We consider equation [Tc13] from Pico-typecheck-old specification
for our discussion.

\begin{tabbing}
mmm\=\kill\pushtabs\\*[-8.5pt]
mmmm\= \=mmmmmmmmmmmmmmmm\= \=mmmmmmmmmmmmmmmm\= \=\kill
\EQU{\LEX{$[$}\LEX{Tc13}\LEX{$]$}}{\VAR{\LEX{Tenv}}{}~\penalty+5\LEX{$.$
}~\penalty+5\VAR{\LEX{Nat-con}}{}}{\CON{natural}}
\poptabs
\end{tabbing}

If we look at this equation for relationships between the
left hand side and right hand side, we can hardly see any
other than the obvious one indicated by the $=$ symbol.

\catcode`@=11
\expandafter\ifx\csname graph\endcsname\relax \alloc@4\box\chardef\insc@unt\graph\fi
\expandafter\ifx\csname graphtemp\endcsname\relax \alloc@1\dimen\dimendef\insc@unt\graphtemp\fi
\catcode`@=12
\setbox\graph=\vtop{%
  \vbox to0pt{\hbox{%
    \special{pn 8}%
    \special{pa 0 900}%
    \special{pa 0 700}%
    \special{pa 200 700}%
    \special{pa 200 900}%
    \special{pa 0 900}%
    \special{ip}%
    \graphtemp=.6ex \advance\graphtemp by 0.8in
    \rlap{\kern 0.1in\lower\graphtemp\hbox to 0pt{\hss \specvariable{Tenv}\hss}}%
    \special{pa 600 900}%
    \special{pa 600 700}%
    \special{pa 800 700}%
    \special{pa 800 900}%
    \special{pa 600 900}%
    \special{ip}%
    \graphtemp=.6ex \advance\graphtemp by 0.8in
    \rlap{\kern 0.7in\lower\graphtemp\hbox to 0pt{\hss \specvariable{Nat-con}\hss}}%
    \special{pa 300 500}%
    \special{pa 300 300}%
    \special{pa 500 300}%
    \special{pa 500 500}%
    \special{pa 300 500}%
    \special{ip}%
    \graphtemp=.6ex \advance\graphtemp by 0.4in
    \rlap{\kern 0.4in\lower\graphtemp\hbox to 0pt{\hss \specparent{ .}\hss}}%
    \special{pa 100 700}%
    \special{pa 400 500}%
    \special{fp}%
    \special{pa 700 700}%
    \special{pa 400 500}%
    \special{fp}%
    \special{pa 1899 700}%
    \special{pa 1899 500}%
    \special{pa 2099 500}%
    \special{pa 2099 700}%
    \special{pa 1899 700}%
    \special{ip}%
    \graphtemp=.6ex \advance\graphtemp by 0.6in
    \rlap{\kern 1.999in\lower\graphtemp\hbox to 0pt{\hss natural\hss}}%
    \graphtemp=.6ex \advance\graphtemp by 0.6in
    \rlap{\kern 1.399in\lower\graphtemp\hbox to 0pt{\hss \reductionarrow\hss}}%
    \special{pa 1899 500}%
    \special{pa 1400 0}%
    \special{pa 501 299}%
    \special{sp -0.05}%
    \special{sh 1}%
    \special{pa 500 300}%
    \special{pa 587 245}%
    \special{pa 602 292}%
    \special{pa 500 300}%
    \special{fp}%
    \kern 2.099in
  }\vss}%
  \kern 0.9in
}

\centerline{\box\graph}

\smallskip

Since the possible terms which match the left side are
not program terms (Pico syntax terms), there would be no (transitive)
relation from the reduct to any program term.

Now let use consider the following equations instead of [Tc13]
in the module Pico-typecheck-old. Equations [Tc9b] and [Tc10]
are from module Pico-typecheck-new.

\begin{tabbing}
mmm\=\kill\pushtabs\\*[-8.5pt]
mmmm\= \=mmmmmmmmmmmmmmmm\= \=mmmmmmmmmmmmmmmm\= \=\kill
\EQU{\LEX{$[$}\LEX{Tc9b}\LEX{$]$}}{\VAR{\LEX{Tenv}}{}~\penalty+5\LEX{$.$
}~\penalty+5\VAR{\LEX{Type}}{}}{\VAR{\LEX{Type}}{}}
\\
\EQU{\LEX{$[$}\LEX{Tc10}\LEX{$]$}}{\VAR{\LEX{Nat-con}}{}}{\CON{natural}}
\poptabs
\end{tabbing}

The equation [Tc9b] suggests not only a relationship
indicated by the $=$ symbol, but also one about the variable {\em Type};
the variable {\em Type} on left-hand side and right-hand side are same.

\catcode`@=11
\expandafter\ifx\csname graph\endcsname\relax \alloc@4\box\chardef\insc@unt\graph\fi
\expandafter\ifx\csname graphtemp\endcsname\relax \alloc@1\dimen\dimendef\insc@unt\graphtemp\fi
\catcode`@=12
\setbox\graph=\vtop{%
  \vbox to0pt{\hbox{%
    \special{pn 8}%
    \special{pa 0 600}%
    \special{pa 0 400}%
    \special{pa 200 400}%
    \special{pa 200 600}%
    \special{pa 0 600}%
    \special{ip}%
    \graphtemp=.6ex \advance\graphtemp by 0.5in
    \rlap{\kern 0.1in\lower\graphtemp\hbox to 0pt{\hss \specvariable{Tenv}\hss}}%
    \special{pa 400 600}%
    \special{pa 400 400}%
    \special{pa 600 400}%
    \special{pa 600 600}%
    \special{pa 400 600}%
    \special{ip}%
    \graphtemp=.6ex \advance\graphtemp by 0.5in
    \rlap{\kern 0.5in\lower\graphtemp\hbox to 0pt{\hss \specvariable{Type}\hss}}%
    \special{pa 200 200}%
    \special{pa 200 0}%
    \special{pa 400 0}%
    \special{pa 400 200}%
    \special{pa 200 200}%
    \special{ip}%
    \graphtemp=.6ex \advance\graphtemp by 0.1in
    \rlap{\kern 0.3in\lower\graphtemp\hbox to 0pt{\hss \specparent{ .}\hss}}%
    \special{pa 100 400}%
    \special{pa 300 200}%
    \special{fp}%
    \special{pa 500 400}%
    \special{pa 300 200}%
    \special{fp}%
    \special{pa 1700 400}%
    \special{pa 1700 200}%
    \special{pa 1900 200}%
    \special{pa 1900 400}%
    \special{pa 1700 400}%
    \special{ip}%
    \graphtemp=.6ex \advance\graphtemp by 0.3in
    \rlap{\kern 1.8in\lower\graphtemp\hbox to 0pt{\hss \specvariable{Type}\hss}}%
    \graphtemp=.6ex \advance\graphtemp by 0.3in
    \rlap{\kern 1.2in\lower\graphtemp\hbox to 0pt{\hss \reductionarrow\hss}}%
    \special{pa 1700 400}%
    \special{pa 1201 599}%
    \special{pa 602 599}%
    \special{sp -0.05}%
    \special{sh 1}%
    \special{pa 600 600}%
    \special{pa 700 575}%
    \special{pa 700 625}%
    \special{pa 600 600}%
    \special{fp}%
    \kern 1.9in
  }\vss}%
  \kern 0.724in
}


\centerline{\box\graph}

\smallskip

Thus, when this rule is used to rewrite a term,
the reduct term could be related to its redex in two ways.
First as in the previous case, would not lead to relating
the reduct to a source program sub-term for our specification, 
but another that could potentially be useful! 
In our case {\tt natural} and {\tt string} are two
words that can be found in program source. The second
relation could thus help the system in tracking the reduct to a part
of the source program.

% \begin{verbatim}
% Arie: The following should be provided by an adaptation of the
% origin notion rather than by an adaptation of the language/specification.
% \end{verbatim}
Next we can consider the case of constants of type {natural}
that equation [Tc13] was used for. 

\catcode`@=11
\expandafter\ifx\csname graph\endcsname\relax \alloc@4\box\chardef\insc@unt\graph\fi
\expandafter\ifx\csname graphtemp\endcsname\relax \alloc@1\dimen\dimendef\insc@unt\graphtemp\fi
\catcode`@=12
\setbox\graph=\vtop{%
  \vbox to0pt{\hbox{%
    \special{pn 8}%
    \special{pa 0 700}%
    \special{pa 0 500}%
    \special{pa 200 500}%
    \special{pa 200 700}%
    \special{pa 0 700}%
    \special{ip}%
    \graphtemp=.6ex \advance\graphtemp by 0.6in
    \rlap{\kern 0.1in\lower\graphtemp\hbox to 0pt{\hss \specvariable{Nat-con}\hss}}%
    \special{pa 1300 700}%
    \special{pa 1300 500}%
    \special{pa 1500 500}%
    \special{pa 1500 700}%
    \special{pa 1300 700}%
    \special{ip}%
    \graphtemp=.6ex \advance\graphtemp by 0.6in
    \rlap{\kern 1.4in\lower\graphtemp\hbox to 0pt{\hss natural\hss}}%
    \graphtemp=.6ex \advance\graphtemp by 0.6in
    \rlap{\kern 0.8in\lower\graphtemp\hbox to 0pt{\hss \reductionarrow\hss}}%
    \special{pa 1300 500}%
    \special{pa 801 0}%
    \special{pa 202 499}%
    \special{sp -0.05}%
    \special{sh 1}%
    \special{pa 200 500}%
    \special{pa 261 417}%
    \special{pa 293 455}%
    \special{pa 200 500}%
    \special{fp}%
    \kern 1.5in
  }\vss}%
  \kern 0.7in
}

\centerline{\box\graph}

\smallskip

It is now easy to observe
that the necessary relation between a constant %of type 
and its reduct the word {\tt natural} 
(type name in our case) exists as desired.
Handling the constants like this (equations [Tc10] and [Tc11]
of module Pico-typecheck-new) seems to
provide enough information to the system to show
the error location information automatically if a constant is involved 
in causing the error message. 

Another situation that is, vaguely, similar to the case of 
constants is the syntax constructs of the program themselves ---
e.g., the need to know which ``{\tt :=}'' or ``{\tt -}''
in the source program appeared in error message discussed earlier.
The current specification of the module Pico-syntax
abstracts away such tokens, i.e., the origin relation
for equation [Tc13] in Pico-typecheck-new is (no
relations for the {\tt -} token):

\catcode`@=11
\expandafter\ifx\csname graph\endcsname\relax \alloc@4\box\chardef\insc@unt\graph\fi
\expandafter\ifx\csname graphtemp\endcsname\relax \alloc@1\dimen\dimendef\insc@unt\graphtemp\fi
\catcode`@=12
\setbox\graph=\vtop{%
  \vbox to0pt{\hbox{%
    \special{pn 8}%
    \special{pa 0 600}%
    \special{pa 0 400}%
    \special{pa 200 400}%
    \special{pa 200 600}%
    \special{pa 0 600}%
    \special{ip}%
    \graphtemp=.6ex \advance\graphtemp by 0.5in
    \rlap{\kern 0.1in\lower\graphtemp\hbox to 0pt{\hss \specvariable{Tenv}\hss}}%
    \special{pa 600 1000}%
    \special{pa 600 800}%
    \special{pa 800 800}%
    \special{pa 800 1000}%
    \special{pa 600 1000}%
    \special{ip}%
    \graphtemp=.6ex \advance\graphtemp by 0.9in
    \rlap{\kern 0.7in\lower\graphtemp\hbox to 0pt{\hss \specvariable{$E_1$}\hss}}%
    \special{pa 1200 1000}%
    \special{pa 1200 800}%
    \special{pa 1400 800}%
    \special{pa 1400 1000}%
    \special{pa 1200 1000}%
    \special{ip}%
    \graphtemp=.6ex \advance\graphtemp by 0.9in
    \rlap{\kern 1.3in\lower\graphtemp\hbox to 0pt{\hss \specvariable{$E_2$}\hss}}%
    \special{pa 900 600}%
    \special{pa 900 400}%
    \special{pa 1100 400}%
    \special{pa 1100 600}%
    \special{pa 900 600}%
    \special{ip}%
    \graphtemp=.6ex \advance\graphtemp by 0.5in
    \rlap{\kern 1in\lower\graphtemp\hbox to 0pt{\hss \specparent{ -}\hss}}%
    \special{pa 700 800}%
    \special{pa 1000 600}%
    \special{fp}%
    \special{pa 1300 800}%
    \special{pa 1000 600}%
    \special{fp}%
    \special{pa 600 200}%
    \special{pa 600 0}%
    \special{pa 800 0}%
    \special{pa 800 200}%
    \special{pa 600 200}%
    \special{ip}%
    \graphtemp=.6ex \advance\graphtemp by 0.1in
    \rlap{\kern 0.7in\lower\graphtemp\hbox to 0pt{\hss \specparent{ .}\hss}}%
    \special{pa 100 400}%
    \special{pa 700 200}%
    \special{fp}%
    \special{pa 1000 400}%
    \special{pa 700 200}%
    \special{fp}%
    \special{pa 2199 1000}%
    \special{pa 2199 800}%
    \special{pa 2399 800}%
    \special{pa 2399 1000}%
    \special{pa 2199 1000}%
    \special{ip}%
    \graphtemp=.6ex \advance\graphtemp by 0.9in
    \rlap{\kern 2.299in\lower\graphtemp\hbox to 0pt{\hss \specvariable{Tenv}\hss}}%
    \special{pa 2599 1000}%
    \special{pa 2599 800}%
    \special{pa 2799 800}%
    \special{pa 2799 1000}%
    \special{pa 2599 1000}%
    \special{ip}%
    \graphtemp=.6ex \advance\graphtemp by 0.9in
    \rlap{\kern 2.699in\lower\graphtemp\hbox to 0pt{\hss \specvariable{$E_1$}\hss}}%
    \special{pa 2399 600}%
    \special{pa 2399 400}%
    \special{pa 2599 400}%
    \special{pa 2599 600}%
    \special{pa 2399 600}%
    \special{ip}%
    \graphtemp=.6ex \advance\graphtemp by 0.5in
    \rlap{\kern 2.499in\lower\graphtemp\hbox to 0pt{\hss \specparent{ .}\hss}}%
    \special{pa 2299 800}%
    \special{pa 2499 600}%
    \special{fp}%
    \special{pa 2699 800}%
    \special{pa 2499 600}%
    \special{fp}%
    \special{pa 3199 1000}%
    \special{pa 3199 800}%
    \special{pa 3399 800}%
    \special{pa 3399 1000}%
    \special{pa 3199 1000}%
    \special{ip}%
    \graphtemp=.6ex \advance\graphtemp by 0.9in
    \rlap{\kern 3.299in\lower\graphtemp\hbox to 0pt{\hss \specvariable{Tenv}\hss}}%
    \special{pa 3599 1000}%
    \special{pa 3599 800}%
    \special{pa 3799 800}%
    \special{pa 3799 1000}%
    \special{pa 3599 1000}%
    \special{ip}%
    \graphtemp=.6ex \advance\graphtemp by 0.9in
    \rlap{\kern 3.699in\lower\graphtemp\hbox to 0pt{\hss \specvariable{$E_2$}\hss}}%
    \special{pa 3399 600}%
    \special{pa 3399 400}%
    \special{pa 3599 400}%
    \special{pa 3599 600}%
    \special{pa 3399 600}%
    \special{ip}%
    \graphtemp=.6ex \advance\graphtemp by 0.5in
    \rlap{\kern 3.499in\lower\graphtemp\hbox to 0pt{\hss \specparent{ .}\hss}}%
    \special{pa 3299 800}%
    \special{pa 3499 600}%
    \special{fp}%
    \special{pa 3699 800}%
    \special{pa 3499 600}%
    \special{fp}%
    \special{pa 2899 200}%
    \special{pa 2899 0}%
    \special{pa 3099 0}%
    \special{pa 3099 200}%
    \special{pa 2899 200}%
    \special{ip}%
    \graphtemp=.6ex \advance\graphtemp by 0.1in
    \rlap{\kern 2.999in\lower\graphtemp\hbox to 0pt{\hss \specparent{ -}\hss}}%
    \special{pa 2499 400}%
    \special{pa 2999 200}%
    \special{fp}%
    \special{pa 3499 400}%
    \special{pa 2999 200}%
    \special{fp}%
    \graphtemp=.6ex \advance\graphtemp by 0.5in
    \rlap{\kern 1.8in\lower\graphtemp\hbox to 0pt{\hss \reductionarrow\hss}}%
    \special{pa 2199 1000}%
    \special{pa 1300 1299}%
    \special{pa 701 1299}%
    \special{pa 102 599}%
    \special{sp -0.05}%
    \special{sh 1}%
    \special{pa 100 600}%
    \special{pa 184 659}%
    \special{pa 146 692}%
    \special{pa 100 600}%
    \special{fp}%
    \special{pa 3199 1000}%
    \special{pa 1300 1299}%
    \special{pa 701 1299}%
    \special{pa 102 599}%
    \special{sp -0.05}%
    \special{sh 1}%
    \special{pa 100 600}%
    \special{pa 184 659}%
    \special{pa 146 692}%
    \special{pa 100 600}%
    \special{fp}%
    \special{pa 2599 1000}%
    \special{pa 2100 1599}%
    \special{pa 1301 1599}%
    \special{pa 802 999}%
    \special{sp -0.05}%
    \special{sh 1}%
    \special{pa 800 1000}%
    \special{pa 883 1061}%
    \special{pa 845 1093}%
    \special{pa 800 1000}%
    \special{fp}%
    \special{pa 3599 1000}%
    \special{pa 2700 1749}%
    \special{pa 1401 999}%
    \special{sp -0.05}%
    \special{sh 1}%
    \special{pa 1400 1000}%
    \special{pa 1499 1028}%
    \special{pa 1474 1071}%
    \special{pa 1400 1000}%
    \special{fp}%
    \kern 3.799in
  }\vss}%
  \kern 1.849in
}

\centerline{\box\graph}

\smallskip

An immediate result of this is that the syntax could be
defined slightly differently so that the rest comes for free.
The following modification to syntax, effectively re-does
the implicit structure of operations and statements without
asking for changes in the specification of the type-checker.
Thus using the Pico-syntax-new module below, with origin
tracking provides us enough of the location information for 
tracking the origin of 
error message (See Section~\ref{ORIGIN} for how the system is used).

\newpage

\MODULEBEGIN{Pico-tokens}
\EXPORTSBEGIN{}
\SORTS{\ULEX{AOP} \ULEX{SOP} \ULEX{IF} \ULEX{THEN} \ULEX{ELSE} \ULEX{FI} \ULEX{WHILE} \ULEX{DO} \ULEX{OD} \ULEX{ASGN}}
 \CFGBEGIN{}
\CFGFUN{\QLEX{$+$}}{\ULEX{AOP}}{}
 \CFGFUN{\QLEX{$-$}}{\ULEX{AOP}}{}
 \CFGFUN{\QLEX{$||$}}{\ULEX{SOP}}{}
 \CFGFUN{\LEX{if}}{\ULEX{IF}}{}
 \CFGFUN{\LEX{then}}{\ULEX{THEN}}{}
 \CFGFUN{\LEX{else}}{\ULEX{ELSE}}{}
 \CFGFUN{\LEX{fi}}{\ULEX{FI}}{}
 \CFGFUN{\LEX{while}}{\ULEX{WHILE}}{}
 \CFGFUN{\LEX{do}}{\ULEX{DO}}{}
 \CFGFUN{\LEX{od}}{\ULEX{OD}}{}
 \CFGFUN{\QLEX{$:=$}}{\ULEX{ASGN}}{}
\CFGEND{}
\EXPORTSEND{}
\MODULEEND{}

\MODULEBEGIN{Pico-syntax-new}
\IMPORTS{\LEX{Layout} \LEX{Identifiers} \LEX{Integers} \LEX{Strings} \LEX{Booleans} \LEX{Pico-tokens}}
\EXPORTSBEGIN{}
\SORTS{\ULEX{PROGRAM} \ULEX{DECLS} \ULEX{ID-TYPE} \ULEX{SERIES} \ULEX{STATEMENT} \ULEX{EXP} \ULEX{TYPE}}
 \CFGBEGIN{}
\CFGFUN{\LEX{begin} \ULEX{DECLS} \ULEX{SERIES} \LEX{end}}{\ULEX{PROGRAM}}{}
 \CFGFUN{\LEX{declare} \{\ULEX{ID-TYPE} \QLEX{$,$}\}\LEX{$*$} \QLEX{$;$}}{\ULEX{DECLS}}{}
 \CFGFUN{\ULEX{ID} \QLEX{$:$} \ULEX{TYPE}}{\ULEX{ID-TYPE}}{}
 \CFGFUN{\{\ULEX{STATEMENT} \QLEX{$;$}\}\LEX{$*$}}{\ULEX{SERIES}}{}
 \CFGFUN{\ULEX{ID} \ULEX{ASGN} \ULEX{EXP}}{\ULEX{STATEMENT}}{}
 \CFGFUN{\ULEX{IF} \ULEX{EXP} \ULEX{THEN} \ULEX{SERIES} \ULEX{ELSE} \ULEX{SERIES} \ULEX{FI}}{\ULEX{STATEMENT}}{}
 \CFGFUN{\ULEX{WHILE} \ULEX{EXP} \ULEX{DO} \ULEX{SERIES} \ULEX{OD}}{\ULEX{STATEMENT}}{}
 \\
\CFGFUN{\ULEX{EXP} \ULEX{AOP} \ULEX{EXP}}{\ULEX{EXP}}{\{\KW{\LEX{left}}\}}
 \CFGFUN{\ULEX{EXP} \ULEX{SOP} \ULEX{EXP}}{\ULEX{EXP}}{\{\KW{\LEX{left}}\}}
 \CFGFUN{\ULEX{ID}}{\ULEX{EXP}}{}
 \CFGFUN{\ULEX{NAT-CON}}{\ULEX{EXP}}{}
 \CFGFUN{\ULEX{STR-CON}}{\ULEX{EXP}}{}
 \CFGFUN{\QLEX{$($} \ULEX{EXP} \QLEX{$)$}}{\ULEX{EXP}}{\{\KW{\LEX{bracket}}\}}
 \\
\CFGFUN{\LEX{natural}}{\ULEX{TYPE}}{}
 \CFGFUN{\LEX{string}}{\ULEX{TYPE}}{}
\CFGEND{}
 \VARIABLESBEGIN{}
\VARDECL{\LEX{D}}{\ULEX{DECLS}}
 \VARDECL{\LEX{Id-type-list}}{\{\ULEX{ID-TYPE} \QLEX{$,$}\}\LEX{$*$}}
 \VARDECL{\LEX{S} \LEX{$[$12$]$}\LEX{$*$}}{\ULEX{SERIES}}
 \VARDECL{\LEX{Stat}}{\ULEX{STATEMENT}}
 \VARDECL{\LEX{Stat-list}}{\{\ULEX{STATEMENT} \QLEX{$;$}\}\LEX{$+$}}
 \VARDECL{\LEX{Exp} \LEX{$[$12$\,']$}\LEX{$*$}}{\ULEX{EXP}}
\VARIABLESEND{}
\EXPORTSEND{}
\MODULEEND{}


\newpage
The origin relations for equation [Tc13] now become:

\catcode`@=11
\expandafter\ifx\csname graph\endcsname\relax \alloc@4\box\chardef\insc@unt\graph\fi
\expandafter\ifx\csname graphtemp\endcsname\relax \alloc@1\dimen\dimendef\insc@unt\graphtemp\fi
\catcode`@=12
\setbox\graph=\vtop{%
  \vbox to0pt{\hbox{%
    \special{pn 8}%
    \special{pa 0 700}%
    \special{pa 0 500}%
    \special{pa 200 500}%
    \special{pa 200 700}%
    \special{pa 0 700}%
    \special{ip}%
    \graphtemp=.6ex \advance\graphtemp by 0.6in
    \rlap{\kern 0.1in\lower\graphtemp\hbox to 0pt{\hss \specvariable{Tenv}\hss}}%
    \special{pa 400 1100}%
    \special{pa 400 900}%
    \special{pa 600 900}%
    \special{pa 600 1100}%
    \special{pa 400 1100}%
    \special{ip}%
    \graphtemp=.6ex \advance\graphtemp by 1in
    \rlap{\kern 0.5in\lower\graphtemp\hbox to 0pt{\hss \specvariable{$E_1$}\hss}}%
    \special{pa 800 1100}%
    \special{pa 800 900}%
    \special{pa 1000 900}%
    \special{pa 1000 1100}%
    \special{pa 800 1100}%
    \special{ip}%
    \graphtemp=.6ex \advance\graphtemp by 1in
    \rlap{\kern 0.9in\lower\graphtemp\hbox to 0pt{\hss \specparent{-}\hss}}%
    \special{pa 1200 1100}%
    \special{pa 1200 900}%
    \special{pa 1400 900}%
    \special{pa 1400 1100}%
    \special{pa 1200 1100}%
    \special{ip}%
    \graphtemp=.6ex \advance\graphtemp by 1in
    \rlap{\kern 1.3in\lower\graphtemp\hbox to 0pt{\hss \specvariable{$E_2$}\hss}}%
    \special{pa 800 700}%
    \special{pa 800 500}%
    \special{pa 1000 500}%
    \special{pa 1000 700}%
    \special{pa 800 700}%
    \special{ip}%
    \graphtemp=.6ex \advance\graphtemp by 0.6in
    \rlap{\kern 0.9in\lower\graphtemp\hbox to 0pt{\hss \specparent{ @}\hss}}%
    \special{pa 500 900}%
    \special{pa 900 700}%
    \special{fp}%
    \special{pa 900 900}%
    \special{pa 900 700}%
    \special{fp}%
    \special{pa 1300 900}%
    \special{pa 900 700}%
    \special{fp}%
    \special{pa 600 300}%
    \special{pa 600 100}%
    \special{pa 800 100}%
    \special{pa 800 300}%
    \special{pa 600 300}%
    \special{ip}%
    \graphtemp=.6ex \advance\graphtemp by 0.2in
    \rlap{\kern 0.7in\lower\graphtemp\hbox to 0pt{\hss \specparent{ .}\hss}}%
    \special{pa 100 500}%
    \special{pa 700 300}%
    \special{fp}%
    \special{pa 900 500}%
    \special{pa 700 300}%
    \special{fp}%
    \special{pa 2100 1100}%
    \special{pa 2100 900}%
    \special{pa 2300 900}%
    \special{pa 2300 1100}%
    \special{pa 2100 1100}%
    \special{ip}%
    \graphtemp=.6ex \advance\graphtemp by 1in
    \rlap{\kern 2.2in\lower\graphtemp\hbox to 0pt{\hss \specvariable{Tenv}\hss}}%
    \special{pa 2500 1100}%
    \special{pa 2500 900}%
    \special{pa 2700 900}%
    \special{pa 2700 1100}%
    \special{pa 2500 1100}%
    \special{ip}%
    \graphtemp=.6ex \advance\graphtemp by 1in
    \rlap{\kern 2.6in\lower\graphtemp\hbox to 0pt{\hss \specvariable{$E_1$}\hss}}%
    \special{pa 2300 700}%
    \special{pa 2300 500}%
    \special{pa 2500 500}%
    \special{pa 2500 700}%
    \special{pa 2300 700}%
    \special{ip}%
    \graphtemp=.6ex \advance\graphtemp by 0.6in
    \rlap{\kern 2.4in\lower\graphtemp\hbox to 0pt{\hss \specparent{ .}\hss}}%
    \special{pa 2200 900}%
    \special{pa 2400 700}%
    \special{fp}%
    \special{pa 2600 900}%
    \special{pa 2400 700}%
    \special{fp}%
    \special{pa 2900 700}%
    \special{pa 2900 500}%
    \special{pa 3100 500}%
    \special{pa 3100 700}%
    \special{pa 2900 700}%
    \special{ip}%
    \graphtemp=.6ex \advance\graphtemp by 0.6in
    \rlap{\kern 3in\lower\graphtemp\hbox to 0pt{\hss \specparent{-}\hss}}%
    \special{pa 3300 1100}%
    \special{pa 3300 900}%
    \special{pa 3500 900}%
    \special{pa 3500 1100}%
    \special{pa 3300 1100}%
    \special{ip}%
    \graphtemp=.6ex \advance\graphtemp by 1in
    \rlap{\kern 3.4in\lower\graphtemp\hbox to 0pt{\hss \specvariable{Tenv}\hss}}%
    \special{pa 3700 1100}%
    \special{pa 3700 900}%
    \special{pa 3900 900}%
    \special{pa 3900 1100}%
    \special{pa 3700 1100}%
    \special{ip}%
    \graphtemp=.6ex \advance\graphtemp by 1in
    \rlap{\kern 3.8in\lower\graphtemp\hbox to 0pt{\hss \specvariable{$E_2$}\hss}}%
    \special{pa 3500 700}%
    \special{pa 3500 500}%
    \special{pa 3700 500}%
    \special{pa 3700 700}%
    \special{pa 3500 700}%
    \special{ip}%
    \graphtemp=.6ex \advance\graphtemp by 0.6in
    \rlap{\kern 3.6in\lower\graphtemp\hbox to 0pt{\hss \specparent{ .}\hss}}%
    \special{pa 3400 900}%
    \special{pa 3600 700}%
    \special{fp}%
    \special{pa 3800 900}%
    \special{pa 3600 700}%
    \special{fp}%
    \special{pa 2900 300}%
    \special{pa 2900 100}%
    \special{pa 3100 100}%
    \special{pa 3100 300}%
    \special{pa 2900 300}%
    \special{ip}%
    \graphtemp=.6ex \advance\graphtemp by 0.2in
    \rlap{\kern 3in\lower\graphtemp\hbox to 0pt{\hss \specparent{ @}\hss}}%
    \special{pa 2400 500}%
    \special{pa 3000 300}%
    \special{fp}%
    \special{pa 3000 500}%
    \special{pa 3000 300}%
    \special{fp}%
    \special{pa 3600 500}%
    \special{pa 3000 300}%
    \special{fp}%
    \graphtemp=.6ex \advance\graphtemp by 0.6in
    \rlap{\kern 1.8in\lower\graphtemp\hbox to 0pt{\hss \reductionarrow\hss}}%
    \special{pa 2100 1100}%
    \special{pa 1301 1399}%
    \special{pa 502 1399}%
    \special{pa 103 699}%
    \special{sp -0.05}%
    \special{sh 1}%
    \special{pa 100 700}%
    \special{pa 171 774}%
    \special{pa 128 799}%
    \special{pa 100 700}%
    \special{fp}%
    \special{pa 3300 1100}%
    \special{pa 1301 1399}%
    \special{pa 502 1399}%
    \special{pa 103 699}%
    \special{sp -0.05}%
    \special{sh 1}%
    \special{pa 100 700}%
    \special{pa 171 774}%
    \special{pa 128 799}%
    \special{pa 100 700}%
    \special{fp}%
    \special{pa 2500 1100}%
    \special{pa 2001 1699}%
    \special{pa 1302 1699}%
    \special{pa 603 1099}%
    \special{sp -0.05}%
    \special{sh 1}%
    \special{pa 600 1100}%
    \special{pa 692 1146}%
    \special{pa 660 1184}%
    \special{pa 600 1100}%
    \special{fp}%
    \special{pa 3700 1100}%
    \special{pa 2601 1849}%
    \special{pa 1402 1099}%
    \special{sp -0.05}%
    \special{sh 1}%
    \special{pa 1400 1100}%
    \special{pa 1498 1132}%
    \special{pa 1472 1174}%
    \special{pa 1400 1100}%
    \special{fp}%
    \special{pa 2900 500}%
    \special{pa 1801 0}%
    \special{pa 1002 899}%
    \special{sp 0.05}%
    \special{sh 1}%
    \special{pa 1000 900}%
    \special{pa 1048 809}%
    \special{pa 1085 842}%
    \special{pa 1000 900}%
    \special{fp}%
    \kern 3.9in
  }\vss}%
  \kern 1.85in
}

\centerline{\box\graph}

\smallskip

Rest of the  specification is not altered in any manner but
some of the concrete syntax is converted to abstract
syntax for the program constructs, which can now be
handled by the origin-tracking mechanism to indicate
the location of the error.

We note that it is undesirable to force this {\em tokenization} 
on the specifier. But on the brighter side, we also observed
that such tokenization does not call for modification
of rest of the specification. Thus, tokenization can be
automatically performed by the system and the specifier
need not know about it.

Alternative methods to get an effect similar to that of
tokenization are of interest. A notion of {\em decreasing
origins} discussed in \cite{V93.horig} appears to provide
origin information needed for the cases of interest here.


%\input{concl}	

\section{Concluding Remarks}

The style of type--checking in this document concentrates on specifying
only the necessary information, while still providing reasonably good
error messages. 
The style used makes use of the so called non-standard values of an
initial algebra specification to generate errors.
The result of type--checking is to effectively
form a conjunction of the abstract meanings of statements of the
language. All type correct statements evaluate to ``true'' while
an incorrect statement reduces to a structured error. 
This structured error can be used by a separate module
to generate human readble error messages.
Together with origin tracking, this provides 
information on the location of error in the source program automatically. 

%This is a typical example of how the specification
%style and the system can help process the declarative
%knowledge available in algebraic specification.

% This study also shows that the ASF+SDF meta-environment is
% a useful tool, which can be used to generate sophisticated programming
% environments and many necessary tools automatically.
% In particular, syntax-directed program construction, rapid prototyping,
% origin-tracking and effective source-level program analysis techniques
% such as visualization are available.

$$$$
{\bf Acknowledgements:} I would like to thank Arie van Deursen
for many discussions. Arie van Deursen and Susan Uskudarli
provided comments and constructive help with the drafts of the paper. 

%\bibliography{/ufs/gipe/lib/tex/bib,/ufs/gipe/spec/clax/doc/bib,bib}
\bibliography{bib,/ufs/gipe/lib/tex/bib}

\newpage
\appendix

\section{Pico Language}
\label{PICOSYN}

A Pico program consists of declarations followed by statements 
and is defined in Chapter~9 of \cite{BHK89}.
All variables are declared to be either of type natural or
of type string. Statements may be assignment statements,
if-statements and while-statements.
Expressions may be a single identifier, addition or subtraction
of natural numbers, or concatenation of strings.

The imported modules define the lexicals identifiers,
constants for natural numbers and strings used in the language.

The variables defined in this module are used in the
equations of modules that import Pico-syntax 
(e.g., Pico-typecheck modules).

\MODULEBEGIN{Pico-syntax}
\IMPORTS{\LEX{Layout} \LEX{Identifiers} \LEX{Integers} \LEX{Strings} \LEX{Booleans}}
\EXPORTSBEGIN{}
\SORTS{\ULEX{PROGRAM} \ULEX{DECLS} \ULEX{ID-TYPE} \ULEX{SERIES} \ULEX{STATEMENT} \ULEX{EXP} \ULEX{TYPE}}
 \CFGBEGIN{}
\CFGFUN{\LEX{begin} \ULEX{DECLS} \ULEX{SERIES} \LEX{end}}{\ULEX{PROGRAM}}{}
 \CFGFUN{\LEX{declare} \{\ULEX{ID-TYPE} \QLEX{$,$}\}\LEX{$*$} \QLEX{$;$}}{\ULEX{DECLS}}{}
 \CFGFUN{\ULEX{ID} \QLEX{$:$} \ULEX{TYPE}}{\ULEX{ID-TYPE}}{}
 \CFGFUN{\{\ULEX{STATEMENT} \QLEX{$;$}\}\LEX{$*$}}{\ULEX{SERIES}}{}
 \CFGFUN{\ULEX{ID} \QLEX{$:=$} \ULEX{EXP}}{\ULEX{STATEMENT}}{}
 \CFGFUN{\LEX{if} \ULEX{EXP} \LEX{then} \ULEX{SERIES} \LEX{else} \ULEX{SERIES} \LEX{fi}}{\ULEX{STATEMENT}}{}
 \CFGFUN{\LEX{while} \ULEX{EXP} \LEX{do} \ULEX{SERIES} \LEX{od}}{\ULEX{STATEMENT}}{}
 \\
\CFGFUN{\ULEX{EXP} \QLEX{$+$} \ULEX{EXP}}{\ULEX{EXP}}{\{\KW{\LEX{left}}\}}
 \CFGFUN{\ULEX{EXP} \QLEX{$-$} \ULEX{EXP}}{\ULEX{EXP}}{\{\KW{\LEX{left}}\}}
 \CFGFUN{\ULEX{EXP} \QLEX{$||$} \ULEX{EXP}}{\ULEX{EXP}}{\{\KW{\LEX{left}}\}}
 \CFGFUN{\ULEX{ID}}{\ULEX{EXP}}{}
 \CFGFUN{\ULEX{NAT-CON}}{\ULEX{EXP}}{}
 \CFGFUN{\ULEX{STR-CON}}{\ULEX{EXP}}{}
 \CFGFUN{\QLEX{$($} \ULEX{EXP} \QLEX{$)$}}{\ULEX{EXP}}{\{\KW{\LEX{bracket}}\}}
 \\
\CFGFUN{\LEX{natural}}{\ULEX{TYPE}}{}
 \CFGFUN{\LEX{string}}{\ULEX{TYPE}}{}
\CFGEND{}
 \VARIABLESBEGIN{}
\VARDECL{\LEX{D}}{\ULEX{DECLS}}
 \VARDECL{\LEX{Id-type-list}}{\{\ULEX{ID-TYPE} \QLEX{$,$}\}\LEX{$*$}}
 \VARDECL{\LEX{S} \LEX{$[$12$]$}\LEX{$*$}}{\ULEX{SERIES}}
 \VARDECL{\LEX{Stat}}{\ULEX{STATEMENT}}
 \VARDECL{\LEX{Stat-list}}{\{\ULEX{STATEMENT} \QLEX{$;$}\}\LEX{$+$}}
 \VARDECL{\LEX{Exp} \LEX{$[$12$\,']$}\LEX{$*$}}{\ULEX{EXP}}
 \VARDECL{\LEX{Type}}{\ULEX{TYPE}}
\VARIABLESEND{}
\EXPORTSEND{}
\MODULEEND{}


\newpage

Example Pico program:

\verbatimfile{small.pico}

Specifying a type-checker for Pico was done using a table
for type-environments TENV, similar to list representation used below:

\MODULEBEGIN{Type-environments}
\IMPORTS{\LEX{Identifiers} \LEX{Pico-syntax}}
\EXPORTSBEGIN{}
\SORTS{\ULEX{TENV} \ULEX{PAIR}}
 \CFGBEGIN{}
\CFGFUN{\ULEX{ID} \QLEX{$:$} \ULEX{TYPE}}{\ULEX{PAIR}}{}
 \CFGFUN{\QLEX{$[$} \{\ULEX{PAIR} \QLEX{$,$}\}\LEX{$*$} \QLEX{$]$}}{\ULEX{TENV}}{}
 \CFGFUN{\LEX{lookup} \ULEX{ID} \LEX{in} \ULEX{TENV}}{\ULEX{TYPE}}{}
\CFGEND{}
 \VARIABLESBEGIN{}
\VARDECL{\LEX{Pair-list}}{\{\ULEX{PAIR} \QLEX{$,$}\}\LEX{$*$}}
 \VARDECL{\LEX{Tenv}}{\ULEX{TENV}}
\VARIABLESEND{}
\EXPORTSEND{}
\EQUATIONSBEGIN{}
\EQU{\LEX{$[$}\LEX{T1a}\LEX{$]$}}{\LEX{lookup}~\penalty+5\VAR{\LEX{Id}}{}~\penalty+5\LEX{in}~\penalty+5\LEX{$[$}\VAR{\LEX{Id}}{}~\penalty+20\LEX{$:$}~\penalty+20\VAR{\LEX{Type}}{},\VAR{\LEX{Pair-list}}{}\LEX{$]$}}{\VAR{\LEX{Type}}{}}
\\
\CEQU{\LEX{$[$}\LEX{T1b}\LEX{$]$}}%
{\COND{\LEX{lookup}~\penalty+5\VAR{\LEX{Id}}{}~\penalty+5\LEX{in}~\penalty+5\LEX{$[$}\VAR{\LEX{Id}}{$\,'$}~\penalty+20\LEX{$:$}~\penalty+20\VAR{\LEX{Type}}{},\VAR{\LEX{Pair-list}}{}\LEX{$]$}}{\LEX{lookup}~\penalty+5\VAR{\LEX{Id}}{}~\penalty+5\LEX{in}~\penalty+5\LEX{$[$}\VAR{\LEX{Pair-list}}{}\LEX{$]$}}}{{\NCOND{\VAR{\LEX{Id}}{}}{\VAR{\LEX{Id}}{$\,'$}}}}%
\EQUATIONSEND{}
\MODULEEND{}


\end{document}
