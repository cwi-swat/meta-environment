\section{\label{AlgSpec}Algebraic specifications and term rewriting systems}

%something about algebraic specifications,

Algebraic specifications are typically used as formalisms to
specify abstract data types. An abstract data type (ADT) is an
{\em algebra}, i.e., a collection of {\em carrier sets}
and associated {\em functions}. 
In an algebraic specification formalism, the properties of the data type to
be specified have to be expressed in terms of {\em equations}. 

ASF is an algebraic specification formalism that allows modular
construction of specifications \cite{BHK89}. 
A {\em basic} ASF module consists of
\begin{itemize}
\item 
      A set of {\em sorts} and {\em function} declarations.
      Together these constitute the so-called {\em signature} of
      the module. Sorts are names of the carrier sets and functions
      correspond to functions in the algebra described by the module.
\item 
      A set of {\em Variable declarations}, which together with the
      signature define a language of terms.
\item 
      A set of {\em equations} over the terms defined by the language.
\end{itemize}
An ADT specification describes many algebras, all
of them satisfying the equations it contains -- the {\em models} of an ADT.
The model closest to ordinary programming practice is the so-called
{\em initial} algebra. Characteristic properties of initial algebra 
are 1) no junk (only contains objects expressible using its signature)
and 2) no confusion (all equational properties can be deduced from
its equations).
In ASF, a typical specification is a sequence of modules. A module
can be normalized in the context of the specification in which it 
belongs by eliminating all imports. The semantics of a module is
the initial algebra of its normal form.

%and their implementation as term rewriting systems

An interesting consequence of algebraic specifications is that
they can be executed if they can be interpreted as {\em term
rewriting systems} (TRS). For a specification to be a TRS, it
must obey two rules:
\begin{itemize}
\item 
      No variable may occur on the right-hand side of an equation
      which does not occur on the left-hand side.
\item
      The left-hand side of an equation cannot be a sole variable.
\end{itemize}

%basic terminology: redex, contractum, substitutions, contexts

In such a specification (TRS), an equation $s = t$ is interpreted
as a rewrite rule $s\rightarrow t$, meaning that $s$
can be rewritten to $t$. A term is said to be in its {\em normal form}
if it cannot be further rewritten.
A term containing variables is called {\em open},
otherwise it is called a {\em ground term}. A {\em substitution}
assigns values (terms) to the variables in term. An {\em instantiation}
of an open term is obtained by substituting, for each of its variables,
an assigned value.

If $s = t$ is an equation in ASF, it is interpreted as a reduction
rule $s \rightarrow t$. If $s^\phi$ is an instantiation of $s$ using 
substitution $\phi$, then $s^\phi \rightarrow t^\phi$ is called a 
basic reduction step. In this reduction step $s^\phi$ is called a
{\em redex} (for reducible expression). A term with a hole, i.e.,
an unknown subterm is referred to as a {\em context} and is
denoted as $C[ ]$. A {\em reduction sequence} $t_0 \rightarrow
t_1 \rightarrow \cdots \rightarrow t_n$ means that elementary
reduction steps $t_i \rightarrow t_{i+1}$ are used to obtain $t_n$.
Applying a rewrite rule $r: s \rightarrow t$ in a context $C$
using the substitution $\phi$ is denoted by 
$C[s^\phi] \rightarrow_r C[t^\phi]$.  Also, $t^\phi$ is often
referred to as reduct or contractum.
 
%    something about \asdf 

\subsection{\label{ASDF}\asdf }

  ASF, an acronym for Algebraic Specification Formalism, \cite{BHK89}
  is a formalism supporting modularization and conditional equations.
  SDF, Syntax Definition Formalism, has been developed \cite{HHKR89} to
  support the definition of lexical, context-free and abstract syntax.
  These two formalisms have been combined into one algebraic specification
  formalism called \asdf.

  Specifying in the \asdf\ formalism is supported by the
  \asdf\ {\em system} \cite{Kli93}.
  This system is able to generate parsers from \asdf\ specifications
  and derive term rewriting systems, %for specifications in \asdf,
  thus allowing the execution of \asdf\ specifications.
%  Moreover, it can generate syntax-directed editors for
%  modules in the specification as well as for terms over the signature.
%  The system is able to perform several static and semantic checks on the
%  specifications, and support testing of specifications.


%%\subsection{Lexical Syntax}
%%
%%  SDF allows the definition of lexical syntax, i.e., the definition of 
%%  the elementary ``words'' of the syntax.
%%  There are two important corollaries for the specifier:
%%  \begin{itemize}
%%  \item
%%  The specifier must define layout symbols (white space and
%%  comments recognition) %, for equations and terms,
%%  using the designated sort \ULEX{LAYOUT}.
%%  %First of all the \ULEX{LAYOUT} for the equations must be defined,
%%  %i.e., the specifier himself can (and must) define what symbols constitute
%%  %white space, and how comments can be recognized.
%%  A typical layout definition is:
%%  
%%  \MODULEBEGIN{Layout}
%%  \EXPORTSBEGIN{}
%%  \LEXSYNBEGIN{}
%%  \LEXFUN{\LEX{$[$\ $\backslash$t$\backslash$n$]$}}{\ULEX{LAYOUT}}
%%   \LEXFUN{\QLEX{\%\%} $\sim$\LEX{$[$$\backslash$n$]$}\LEX{$*$}
%%  \QLEX{$\backslash$n}}{\ULEX{LAYOUT}}
%%  \LEXSYNEND{}
%%  \EXPORTSEND{}
%%  \MODULEEND{}
%%
%%  A space, tab, or new line is a layout symbol, as well as everything between
%%  two percent signs and a new line.
%%  If tokens of sort \ULEX{LAYOUT} %{\footnotesize \tt LAYOUT}
%%  are detected in a text, they are ignored.  
%%
%%  \item
%%  Variable declarations are treated as declarations of lexical
%%  syntax. 
%%  This implies that any construct allowed in the lexical syntax definition
%%  is allowed in the variable definition section as well.
%%  Consequently it is possible to define the variables
%%  %{\footnotesize \tt p1}, {\footnotesize \tt p2}, {\footnotesize \tt p3}, 
%%  \SLEX{i_1}, \SLEX{i_2}, \SLEX{i_3}, $\cdots$ all at once:
%%
%%  \begin{tabular}{ll}
%%  \VARDECL{\LEX{i} \LEX{$[$0-9$]*$}}{\ULEX{ELEMENT}}
%%  \end{tabular}
%%
%%  This declares all  words starting with %a {\footnotesize \tt p}
%%  an \SLEX{i} followed
%%  by zero or more characters in the range \LEX{0-9} %{\footnotesize \tt 0-9} 
%%  to be variables of sort \ULEX{ELEMENT}.  %{\footnotesize \tt NAT}.
%%  \end{itemize}

A brief introduction to the \asdf\ formalism is given below,
by means of describing {\em list functions}.

\subsubsection{List functions}

  An important feature of the \asdf\ formalism is the existence
  of list functions and list variables.
  List functions have a variable number of arguments,
  and list variables may range over any number of arguments of a list
  function.

  As an example, suppose we would like to have a function
  [] for the empty set, %{\footnotesize \tt [ ]} for the empty set,
  [ E1 ] for a set with one element, %{\footnotesize \tt [ E1 ]} 
  [ E1, E2 ]  for a set with two elements, %{\footnotesize \tt [ E1, E2 ]} 
  and so on.
  
  The way to define this in \asdf\ is as follows:
  \begin{tabbing}
  m\=m\=m\=m\=m\=m\=m\=m\=\kill
  \+\+\SORTS{\ULEX{ELEMENT} \ULEX{SET}}
   \CFGBEGIN{}
  \CFGFUN{\QLEX{$[$} \{\ULEX{ELEMENT} \QLEX{$,$}\}\LEX{$*$}
  \QLEX{$]$}}{\ULEX{SET}
  }{}
  \CFGEND{}
  \end{tabbing}

  The asterisk * says that we want %{\footnotesize \tt *} says that we want
  zero or more \ULEX{ELEMENT}s, %{\footnotesize \tt ELEMENT}s,
  while the comma says that these should be separated by commas.
  Thus, a set consists of \ULEX{ELEMENT}s, separated by commas
  and the set itself is delimited by [ and ]. 
  %{\footnotesize \tt [} and {\footnotesize \tt ]}.

  This list notation is simply an abbreviation for the declaration of
  infinitely many functions [ $\cdots$ ], %{\footnotesize \tt [ ... ]}, 
  each with a different number of arguments.
  Likewise, the same (concrete) syntax could have been
  obtained without using lists by the following:
  %``normal'' BNF grammar rules:

\begin{tabbing}
m\=m\=m\=m\=m\=m\=m\=m\=\kill
%\EXPORTSBEGIN{}
\+\+\SORTS{\ULEX{ELEMENT} \ULEX{ELEMENTS} \ULEX{SET}}
 \CFGBEGIN{}
\CFGFUN{\QLEX{$[$} \ULEX{ELEMENTS} \QLEX{$]$}}{\ULEX{SET}}{}
 \CFGFUN{}{\ULEX{ELEMENTS}}{}
 \CFGFUN{\ULEX{ELEMENT}}{\ULEX{ELEMENTS}}{}
 \CFGFUN{\ULEX{ELEMENTS} \QLEX{$,$}
\ULEX{ELEMENTS}}{\ULEX{ELEMENTS}}{\{\KW{\LEX{left}}\}}
\CFGEND{}
%\EXPORTSEND{}
\end{tabbing}

%  {\footnotesize \begin{verbatim}
%context-free syntax
%   "[" ELEMENTS "]"      -> SET
%   "[" "]"               -> SET  
%   ELEMENT               -> ELEMENTS
%   ELEMENTS "," ELEMENT  -> ELEMENTS   \end{verbatim}}

%\noindent Attributes, between \{\} 
%      brackets, may be associated with functions, stating for instance
%      that the function is {\bf left} or {\bf right} associative.

  In order to define equations over list functions, we need 
  list variables:
\begin{tabbing}
m\=m\=m\=m\=m\=m\=m\=m\=\kill
\+\+\VARIABLESBEGIN{}
\VARDECL{\LEX{Elts} \LEX{$[$123$]$}}{\{\ULEX{ELEMENT} \QLEX{$,$}\}\LEX{$*$}}
 \VARDECL{\LEX{i}}{\ULEX{ELEMENT}}
\VARIABLESEND{}
\EQUATIONSBEGIN{}
\EQU{\LEX{$[$}\LEX{eq1}
\LEX{$]$}}{\LEX{$[$}\IVAR{\LEX{Elts}}{1},\VAR{\LEX{i}},
\IVAR{\LEX{Elts}}{2},\VAR{\LEX{i}},\IVAR{\LEX{Elts}}{3}\LEX{$]$}}
{\LEX{$[$}
\IVAR{\LEX{Elts}}{1},\VAR{\LEX{i}},\IVAR{\LEX{Elts}}{2},\IVAR{\LEX{Elts}  
  }{3}\LEX{$]$} 
}
\EQUATIONSEND{}
\end{tabbing}

  \SLEX{Elts_1}, \SLEX{Elts_2}, and \SLEX{Elts_3} are  list variables,
  ranging over list of zero or more \ULEX{ELEMENT}s 
  \underline{separated} by commas.

  Here we have specified in one single equation that
  elements of sets do not have multiplicity:
  any set containing element \SLEX{i} %{\footnotesize \tt i} 
  at least two times is 
  equal to  the set containing one occurrence less of \SLEX{i}.
  %{\footnotesize \tt i}.
  However, the alternative way of defining the \ULEX{SET} syntax using
  explicit ``,'' operator for \ULEX{ELEMENTS} sort would require
  many more equations to capture this effect.
  %associativity of \ULEX{SET} elements.


